\chapter{Background}
\label{chap:background}

\section{Basics of Fully Homomorphic Encryption}
\gls{he} makes it possible to operate on data without knowing it.
One can distinguish three flavors of it, Partial-, Semi- and \gls{fhe}.

\subsection{HE using RSA}
\subsection{Learning with Errors (LWE)}
\subsection{Ring-LWE}
\subsection{The BFV scheme}
\subsection{The CKKS scheme}
The CKKS scheme allows us?? to perform approximate arithmetic on floating point numbers.

\section{Machine Learning}
\subsection{Linear Regression?}
\subsection{Gradient Descent?}
\subsection{The Backpropagation Algorithm}
\subsection{Multi-Layered Neural Networks}
Matrix -> Activation Function
\begin{itemize}
    \item Matrix Multiplication (Dense Layer)
    \item Convolutional Layer
    \item Sigmoid Activation
    \item Max Pooling
\end{itemize}

\section{Demo}
In this chapter, we?? provide some usage examples for
glossaries and acronym lists with \texttt{glossaries} (Section \ref{sec:gloss}),
bibliography and citations with \texttt{biblatex} (Section \ref{sec:bib}), and more.

\begin{figure}[H]
    \centering
    \includegraphics[width=0.8\linewidth]{figures/taylor-relu.png}
    \caption{Comparison of the Relu activation function vs. its Taylor expansion}
\end{figure}

\section{Notation and Acronyms}
\label{sec:gloss}
Symbols and acronyms are defined in the preamble, after loading the \texttt{glossaries} package, and used as follows.

In this chapter, we introduce the necessary background on the \gls{aes}.
We denote binary exclusive-or by \gls{xor}.

\section{Citations}
\label{sec:bib}
This is an example of how to specify and cite
a book \cite{AESbook},
a journal article \cite{bstjShannon49},
a conference article \cite{spKocherHFGGHHLM019},
and an informal report \cite{iacrSchneierFKR15}.
We can also add the authors' names to the citation:
\Gls{aes} is a block cipher defined by \textcite{AESbook}.
