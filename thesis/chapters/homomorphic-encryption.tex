\chapter{Homorphic Encryption}

\section{Basics of Fully Homomorphic Encryption}
\gls{he} makes it possible to operate on data without knowing it.
One can distinguish three flavors of it, Partial-, Somewhat- and \gls{fhe}.

For \Gls{fhe}, there exist a few schemes in use today with existing implementations.
\begin{itemize}
  \item \gls{bfv} scheme for integer arithmetic
        (\cite{2012-fv-original}, \cite{2012-brakerski}).
  \item \gls{bgv} scheme for integer arithmetic \parencite{2012-bgv-original}.
  \item \gls{ckks} scheme for (complex) floating point arithmetic \parencite{2017-ckks-original}.
  \item Ducas-Micciancio (FHEW) and Chillotti-Gama-Georgieva-Izabachene (TFHE) schemes for Boolean circuit evaluation
        \parencite{2019-tfhe-original}.
\end{itemize}

We will first introduce the BFV scheme (integer arithmetic) as it represents a fundamental building block behind CKKS.
Due to the inherent applications, this thesis will focus on the CKKS scheme to perform homomorphic operations on (complex-valued) floating point numbers and vectors.

\section{HE using RSA}
In order to illustrate the basic idea behind \Gls{he}, without distancing ourselves too far from the original goal of introducing basic \gls{he} operations used in practice, this short section aims to motivate the definition of ring homomorphisms (cf. \autoref{def:ring-homomorphism}) behind a cryptographic background.

With unpadded RSA, some arithmetic can be performed on the ciphertext - % TODO: find RSA citation
looking at the encrypted ciphertext $\mathcal{E}(m_1) = (m_1)^r \mod n$
of the message $m_1$ and $m_2$ respectively, the following holds:
\begin{align*}
  \mathcal{E}(m_1) \cdot \mathcal{E}(m_2)
   & \equiv (m_1)^r (m_2)^r \mod n     \\
   & \equiv (m_1 m_2)^r \mod n         \\
   & \equiv \mathcal{E}(m_1 \cdot m_2)
\end{align*}

The encryption therefore partially fulfills the properties of a ring homomorphism, which in general terms is defined as follows:

\begin{definition}{Ring Homomorphism}{ring-homomorphism}
  Given two \hyperref[def:ring]{rings} $(R, +, \cdot)$ and $(S, \oplus, \otimes)$, we call a mapping $\varphi: R \rightarrow S$
  a ring homomorphism when it satisfies the following conditions:
  $$\forall a, b \in R: \varphi(a + b) = \varphi(a) \oplus \varphi(b) \wedge \varphi(a \cdot b) =
    \varphi(a) \otimes \varphi(b)$$
\end{definition}

\section{Gentry's FHE-Scheme and BGV}
\cite{2009-gentry-fhe-original}
% TODO: historical introduction

\pagebreak
\section{The BFV scheme}
\cite{2012-fv-original}
\cite{2012-brakerski}
BFV is based on BGV and they are very similar in their core ideas, one can even convert a BFV ciphertext to an equivalent BGV ciphertext \parencite{2021-he-revisiting}.

In this section, we will focus on a slightly altered implementation introduced in \cite{2014-fv-comparison}.

For two tuples $(\cdot, \cdot)$ defined over the same ring, denote their element-wise addition as $(\cdot, \cdot) + (\cdot, \cdot)$, element-wise multiplication by a scalar $u$ as $u \cdot (\cdot, \cdot)$ and element-wise rounding as $\lfloor (\cdot, \cdot) \rceil$.

\begin{definition}{The BFV-Scheme}{bfv-scheme}
  Let $R = \Z[X] / \Phi_d(X)$ be a polynomial ring with $\Phi_d(X)$ the $d$\textsuperscript{th} \hyperref[def:cyclotomic-poly]{cyclotomic polynomial}
  ($\rightarrow d \in \N$) for ciphertexts $c \in R \times R$.
  Introduce $R / qR$ the associated quotient ring of the $q$\textsuperscript{th} coset of $R$ with the modulus $q \in \N$.
  Further let $t \in \N$ denote the message modulus with $1<t<q$
  for plain messages $m \in R/tR$ and define $\delta = \lfloor \frac{q}{t} \rfloor$,
  $\delta\inv = \frac{t}{q}$.

  Introduce three bounded discrete probability distributions $\chi_{key}$, $\chi_{enc}$ and $\chi_{error}$ over $R/qR$, one which is only used once for key generation, another used for \cryptop{BFV.Encrypt} and another (usually Gaussian-like) error distribution for manually inserted error terms (confer the \hyperref[def:lwe-search-problem]{LWE-problem}). For BFV, usually $\chi_{key} = \chi_{enc}$.

  For a polynomial $a \in R/qR$, consider the decomposition $a = \sum_{i=0}^{l-1} a_i w^i$ into base $w \in \N$ obtained by $\cryptop{WordDecomp}: R \mapsto R^l, \cryptop{WordDecomp}(a) = ([a_i]_w)_{i=0}^{l-1}$. \\
  Further let $\cryptop{PowersOf}: R \mapsto R^l$ be defined as $\cryptop{PowersOf}(a) = ([a w^i]_q)_{i=0}^{l-1}$.

  Let the parameters $\mathbb{P} = (d, q, t, \chi_{key}, \chi_{error}, w)$ and $l = \lfloor \log_w(q) \rfloor + 1$.
  \vspace{0.2cm}

  \cryptop{BFV.} \\
  \begin{tblr}{Q[l,h]Q[l,h,\textwidth - 3.5cm]}
    \cryptop{ParamGen}$(\lambda)$ & {
        Choose parameters as defined above, given the
        security parameter $\lambda$, such that $1 < t < q$, $w \geq 2$,
        initialize distributions $\chi_{key}$, $\chi_{enc}$ and $\chi_{error}$
        $\quad\rightarrow \mathbb{P}$} \\
    \cryptop{KeyGen}$(\mathbb{P})$ & {
        Generate the secret key $s \leftarrow \chi_{key}$, sample $\vec{\mu} \in (R/qR)^l$
        from $\chi_{error}$ and choose some $\vec{a} \in (R/qR)^l$ uniformly
        at random, compute the relinearization key
        $\vec{\gamma} = (\cryptop{PowersOf}(s^2) - (\vec{\mu} + \vec{a} \cdot s), \vec{a})$
        and finally output the public key for uniformly random
        $a \in (R/qR)$ and $\mu \leftarrow \chi_{error}$ with $b =-(a \cdot s + \mu)$
        as $\vec{p} = (b, a)$.
        $\quad\rightarrow \vec{p}, s, \vec{\gamma}$} \\
    \cryptop{Encrypt}$(\vec{p}, m)$ & {
        Let $(b,a) = \vec{p}$, $u \leftarrow \chi_{enc}$, $\mu_1, \mu_2 \leftarrow \chi_{error}$,
        then the ciphertext is $\vec{c} = u \cdot \vec{p} + (\delta m + \mu_1, \mu_2) = (\delta m + bu + \mu_1, au + \mu_2)$
        $\quad\rightarrow \vec{c}$} \\
    \cryptop{Decrypt}$(s, \vec{c})$ & {
        Decrypt $\vec{c} = (c_0, c_1)$ as
        $m = \lfloor \delta\inv [c_0 + c_1 s]_q \rceil \in R/tR$
        $\quad\rightarrow m$} \\
    \cryptop{Add}$(\vec{c}_1, \vec{c}_2)$ & {
        Let $(c_0^1, c_1^1) = \vec{c}_1$ and $(c_0^2, c_1^2) = \vec{c}_2$
        then $\vec{c}_3 = (c_0^1 + c_0^2, c_1^1 + c_1^2) = \vec{c}_1 + \vec{c}_2$
        $\quad\rightarrow \vec{c}_3$} \\
    \cryptop{Mult}$(\vec{c}_1, \vec{c}_2)$ & {
        Output $\overline{\vec{c}} = (
          \lfloor \delta\inv c_0^1 c_0^2 \rceil,
          \lfloor \delta\inv(c_0^1 c_1^2 + c_1^1 c_0^2) \rceil,
          \lfloor \delta\inv c_1^1 c_1^2 \rceil
          )$
        $\quad\rightarrow \overline{\vec{c}}$} \\
    \cryptop{ReLin}$(\overline{\vec{c}}, \vec{\gamma})$ & {
        Using the relin key $\vec{\gamma} = (\vec{b}, \vec{a})$,
        relinearize from $\overline{\vec{c}} = (c_0, c_1, c_2)$ as
        $\vec{c} = (c_0 + \cryptop{WordDecomp}(c_2) \cdot \vec{b}, c_1 + \cryptop{WordDecomp}(c_2) \cdot \vec{a})$
        $\quad\rightarrow \vec{c}$} \\
  \end{tblr}

  \parencite{2012-fv-original, 2012-brakerski}
\end{definition}

To summarise the parameters and variables, a brief overview of all used symbols is provided in \autoref{tab:bfv-symbols}.
\begin{table}[H]
  \centering
  \caption{Summary of the parameters and symbols in BFV.}
  \begin{tblr}{rll}
    \hline
    \textbf{Symbol} & \textbf{Space} & \textbf{Explanation} \\
    \hline
    $\lambda$ & $\in \R$ & Security parameter \\
    $d$ & $\in \N$ & Index of the cyclotomic polynomial used in $R$ \\
    $q$ & $\in \N$ & Modulus of the ciphertext space $R/qR$ \\
    $t$ & $\in \N$ & Modulus of the plaintext message space $R/tR$ \\
    $\delta$ & $\in \N$ & Ratio between ciphertext and plaintext modulus \\
    $\delta\inv$ & $\in \R$ & Inversion coefficient of the effect of $\delta$ \\
    $w$ & $\in \N$ & Word size used as basis, e.g. $w = 2$ for bits \\
    $l$ & $\in \N$ & Number of words of size $w$ required to encode $q$ \\
    $s$ & $\in R$ & Secret Key \\
    $\vec{p}$ & $\in R/qR \times R/qR$ & Public Key $(b, a)$ \\
    $\vec{\gamma}$ & $\in (R/qR)^l \times (R/qR)^l$ & Relinearization Key \\
    $m$ & $\in R/tR$ & Plaintext Message\\
    $\vec{c}$ & $\in R \times R$ & Ciphertext \\
    $\overline{\vec{c}}$ & $\in R \times R \times R$ & Slightly larger ciphertext resulting from multiplication \\
  \end{tblr}
  \label{tab:bfv-symbols}
\end{table}

\tikzstyle{cleartext} = [rectangle, rounded corners,minimum width=3cm, minimum height=1cm, text centered, draw=black, fill=orange!30]
\tikzstyle{plaintext} = [rectangle, rounded corners,minimum width=3cm, minimum height=1cm, text centered, draw=black, fill=purple!30]
\tikzstyle{ciphertext} = [rectangle, rounded corners,minimum width=3cm, minimum height=1cm, text centered, draw=black, fill=themecolor!30]
\tikzstyle{arrow} = [thick,->,>=stealth]

\begin{tikzpicture}
  \node[plaintext, align = center] (p) at (6,0) {plaintext \\ $m(X)$};
  \node[ciphertext, align = center] (c) at (12,0) {ciphertext \\ $\vec{c} = (c_0(X), c_1(X))$};

  \node[plaintext, align = center] (p2) at (6,-2.5) {plaintext \\ $\tilde{m}(X) = f(m(X))$};
  \node[ciphertext, align = center] (c2) at (12,-2.5) {ciphertext \\ $\tilde{\vec{c}} = f(\vec{c})$};

  \draw (p.north) node[above]{$\Z[X]/(X^N + 1)$};
  \draw (p2.south) node[below]{$\Z[X]/(X^N + 1)$};
  \draw (c.north) node[above]{$(\Z_q[X]/(X^N + 1))^2$};
  \draw (c2.south) node[below]{$(\Z_q[X]/(X^N + 1))^2$};

  \draw [arrow] (p) -- node[anchor=south] {encryption} (c);
  \draw [arrow] (c) -- node[anchor=east] {computing $f$} (c2);
  \draw [arrow] (c2) -- node[anchor=south] {decryption} (p2);
\end{tikzpicture}


\begin{theorem}{BFV encryption is homomorphic with respect to addition}{bfv-enc-is-homomorphic}
  \cryptop{BFV.Encrypt} should encrypt in such a way that the addition algebra can be retained even in the transformed space, showing that we can indeed refer to it as \hyperref[def:ring-homomorphism]{homomorphic} encryption.
\end{theorem}

\textit{Microsoft SEAL} implements the scheme, enabled using \cpp{seal::scheme_type::bfv}.

\section{The CKKS scheme}
The CKKS scheme allows us to perform approximate arithmetic on floating point numbers.
Essentially, the idea is to extend BFV which allows us to operate on vectors $\vec{y} \in \Z_t^n$, by an embedding approach that allows us to encode a (complex) floating point number vector $\vec{x} \in \R^n (\C^n)$ as an integer vector. A na\"ive approach would be to use a fixed-point embedding:
\newcommand{\embed}{\mathrm{embed}}
$$\embed(\vec{x}) = \vec{x} \cdot F$$
with $F \in \Z$. In decimal form, for instance with $F = 1000$, we could effectively encode three decimal places of the original vector $\vec{x}$.
% ... TODO: scale explodes, confer Roman's PETS lecture -> motivation for CKKS.

Introduce $d, R, R/qR$ as in \autoref{def:bfv-scheme} and further define $S = \R[X] / \Phi_d(X)$ a similar polynomial ring to $R$, but over the reals instead of the integers.
Let $N = \varphi(d)$ be the degree of the reducing cyclotomic polynomial of $S$, confer \autoref{def:bounded-polynomials}.
For convenience, we usually choose $d$ a power of $2$ and then, by \autoref{thm:power-of-2-cyclo-poly}, $N = \varphi(d) = \frac{d}{2}$ which yields very efficiently multipliable polynomials because the homomorphic multiplication operation can be performed using a \gls{dft} and further optimized using the \gls{fft}, which in its unmodified form only accepts power-of-2 vector sizes \parencite{2017-ckks-original}.

\begin{definition}{Canonical Embedding $\underline{\sigma}$}{sigma-transform}
  For a real-valued polynomial $p \in S$, define the canonical embedding of $S$ in $\C^N$ as a mapping $\underline{\sigma}: S \mapsto \C^N$ with $$\underline{\sigma}(p) := \big(p(e^{-2\pi i j / N})\big)_{j \in \Z_d^*}$$
  with $\Z_d^* := \{x \in \Z / d\Z \,|\, \gcd(x, d) = 1\}$ the set of all integers smaller than $d$ that do not share a factor $> 1$ with $d$.
  The image of $\underline{\sigma}$ given a set of inputs $R$ shall be denoted as $\underline{\sigma}(R) \subseteq \C^N$.
  Let the inverse of $\underline{\sigma}$ be denoted by $\underline{\sigma}\inv: \C^N \mapsto S$.
\end{definition}

Define the commutative subring $(H, +, \cdot)$ of $(\C^N, +, \cdot)$ on the set
$$H = \{\vec{z} = (z_j)_{j \in \Z_d^*} \in \C^N : z_j = \overline{z_{-j}} \;\forall j \in \Z_d^*\}$$
of all complex-valued vectors $\vec{z}$ where the first half equals the reversed complex-conjugated second half.

\begin{definition}{Natural Projection $\underline{\pi}$}{pi-transform}
  Let $T$ be a mulitplicative subgroup of $\Z_d^*$ with $\Z_d^* / T = \{-1, 1\}$, then the natural projection $\underline{\pi}: C^N \mapsto ??$ is defined as % TODO
  Let its inverse be denoted by $\underline{\pi}\inv: ? \mapsto ?$.
\end{definition}
\begin{definition}{Discretisation to an element of $\underline{\sigma}(R)$}{}
  Using one of several round-off algorithms (c.f. \cite{2013-rlwe-toolkit}), given an element of $H$, define a rounding operation $\underline{\rho}: H \mapsto \underline{\sigma}(R)$ that maps an $h \in H$ to its closest element in $\underline{\sigma}(R)$, also denoted as
  $$\underline{\rho}(h) = \lfloor h \rceil_{\underline{\sigma}(R)}\,.$$
  Further let $\underline{\rho_\delta}(h) = \lfloor \delta \cdot h \rceil_{\underline{\sigma}(R)}$ denote the same rounding operation but with prior scaling by a scalar factor $\delta$.
  Note that $\underline{\rho}\inv$ is given directly as the identity operation because all elements of its domain are already elements of its image. Similarly, $\underline{\rho_\delta}\inv(\vec{y}) = \delta\inv \cdot \vec{y}$.
\end{definition}

Because it is not essential to understanding the encryption scheme, we will skip over concrete implementations of the rounding procedure $\underline{\rho}$.
Note that for choosing a 'close' element $g \in H$, we must first introduce a sense of proximity, in this case done by the $l_\infty$-norm $||g - h||_\infty$ of the difference between $h \in H$ and $g$.

\begin{definition}{The CKKS Scheme}{ckks-scheme}
  Define $R, R/q_L R$ as in \autoref{def:bfv-scheme}.
  \vspace{0.2cm}

  \cryptop{CKKS.} \\
  \begin{tblr}{Q[l,h]Q[l,h,\textwidth - 3.5cm]}
    \cryptop{ParamGen}$(\lambda)$ & {
        Choose parameters as defined above, given the security parameter $\lambda$ and space modulus $q_L$, choose $d \in \N$ a power of $2$, $P, h \in \Z$, $\sigma \in \R$ and initialize distributions $\chi_{key}$, $\chi_{enc}$ and $\chi_{error}$.
        $\quad\rightarrow \mathbb{P}$} \\
    \cryptop{KeyGen}$(\mathbb{P})$ & {
        Sample the secret key $s \leftarrow \chi_{key}$, $a \in R_{q_L}$ uniformly at random, $\mu \leftarrow \chi_{error}$ and obtain the public key $\vec{p} = (b, a)$
        with $b = -a \cdot s + \mu$.
        Sample $a' \in R_{P \cdot q_L}$ uniformly at random, $\mu' \leftarrow \chi_{error}$
        and obtain the evaluation key $\vec{\gamma} = (b', a')$
        with $b' = -a' \cdot s + \mu' + Ps^2$.
        $\quad\rightarrow \vec{p}, s, \vec{\gamma}$} \\
    \cryptop{Encode}$(\vec{z})$ & {For a given input vector $\vec{z}$, output
        $m = (\underline{\sigma}\inv \circ \underline{\rho_\delta} \circ \underline{\pi}\inv)(\vec{z}) = \underline{\sigma}\inv(\lfloor \delta \cdot \underline{\pi}\inv(\vec{z})\rceil_{\underline{\sigma}(R)})$ $\quad\rightarrow m$} \\
    \cryptop{Decode}$(m)$ & {Decode plaintext $m$ as
        $\vec{z} = (\underline{\pi} \circ \underline{\rho_\delta}\inv \circ \underline{\sigma})(m) = (\underline{\pi} \circ \underline{\sigma})(\delta\inv m)$
        $\quad\rightarrow \vec{z}$} \\
    \cryptop{Encrypt}$(\vec{p}, m)$ & {
        Let $(b,a) = \vec{p}$, $u \leftarrow \chi_{enc}$, $\mu_1, \mu_2 \leftarrow \chi_{error}$,
        then the ciphertext is $\vec{c} = u \cdot \vec{p} + (m + \mu_1, \mu_2) = (m + bu + \mu_1, au + \mu_2)$
        $\quad\rightarrow \vec{c}$} \\
    \cryptop{Decrypt}$(s, \vec{c})$ & {
        Decrypt the ciphertext $\vec{c} = (c_0, c_1)$ as $m = \lbrack c_0 + c_1 s\rbrack_{q_L}$
        $\quad\rightarrow m$} \\
    \cryptop{Add}$(\vec{c}_1, \vec{c}_2)$ & {
        Output $\vec{c}_3 = \vec{c}_1 + \vec{c}_2$
        $\quad\rightarrow \vec{c}_3$} \\
    \cryptop{Mult}$(\vec{c}_1, \vec{c}_2)$ & {
        Output $\overline{\vec{c}} = (
          c_0^1 c_0^2,
          c_0^1 c_1^2 + c_1^1 c_0^2,
          c_1^1 c_1^2
          )$
        $\quad\rightarrow \overline{\vec{c}}$} \\
    \cryptop{ReLin}$(\overline{\vec{c}}, \vec{\gamma})$ & {
        Using the evaluation key $\vec{\gamma}$,
        relinearize from $\overline{\vec{c}} = (c_0, c_1, c_2)$ to
        $\vec{c} = (c_0, c_1) + \lfloor P\inv c_2 \vec{\gamma} \rceil$
        $\quad\rightarrow \vec{c}$} \\
    \cryptop{ReScale}$(\vec{c})$ & {
    In order to rescale a ciphertext from level $l_{old}$ to $l_{new}$,
    multiply by a factor $\frac{q_{l_{new}}}{q_{l_{old}}} \in \Q$ and round to the nearest
    element of $(R/q_{l_{new}} R) \times (R/q_{l_{new}} R)$:
    $\vec{c}_{new} = \big\lfloor \frac{q_{l_{new}}}{q_{l_{old}}} \vec{c} \big\rceil$
    $\quad\rightarrow \vec{c_{new}}$} \\
  \end{tblr}

  \parencite{2017-ckks-original}
\end{definition}

For more details on the probability distributions, refer to the original CKKS paper \parencite{2017-ckks-original}, with the following naming relations:
$\chi_{key} = \mathcal{H}WT(h)$ over $\{0,\pm 1\}^N$,
$\chi_{error} = \mathcal{DG}(\sigma^2)$ over $\Z^N$ and
$\chi_{enc} = \mathcal{ZO}(0.5)$ another distribution over $\{0,\pm 1\}^N$.

It should also be noted that the encoding procedure represents an isometric ring isomorphism between its domain and image, as does the decoding procedure.
This reflects in the observation that the plaintext sizes and errors are preserved under the transformations \parencite{2017-ckks-original}.

To summarise the parameters and variables, a brief overview of all used symbols is provided in \autoref{tab:ckks-symbols}.
\begin{table}[H]
  \centering
  \caption{Summary of the parameters and symbols in CKKS.}
  \begin{tblr}{rll}
    \hline
    \textbf{Symbol} & \textbf{Space} & \textbf{Explanation} \\
    \hline
    $\lambda$ & $\in \R$ & Security parameter \\
    $d$ & $\in \N$ & Index of the cyclotomic polynomial used in $R$ \\
    $P$ & $\in \Z$ & Hmm... \\ % TODO
    $h$ & $\in \Z$ & Hamming weight of the secret key (used by $\chi_{key}$) \\
    $\sigma$ & $\in \R$ & Standard deviation of the Gaussian $\chi_{error}$ \\
    $q_L$ & $\in \N$ & Modulus of $R/q_L R$ at level $L$ \\
    $\delta$ & $\in \N$ & Scaling factor used when encoding \\
    $\delta\inv$ & $\in \R$ & Inversion coefficient of the effect of $\delta$ \\
    $s$ & $\in \{0,\pm 1\}^N$ & Secret Key \\
    $\vec{p}$ & $\in R/q_L R \times R/q_L R$ & Public Key $(b, a)$ \\
    $\vec{\gamma}$ & $\in R/(P \cdot q_L)R \times R/(P \cdot q_L)R$ & Relinearization Key \\
    $m$ & $\in R$ & Plaintext Message \\
    $\vec{c}$ & $\in R/q_L R \times R/q_L R$ & Ciphertext Message \\
    $\overline{\vec{c}}$ & $\in R/q_L R \times R/q_L R \times R/q_L R$ & Slightly larger ciphertext from multiplication \\
  \end{tblr}
  \label{tab:ckks-symbols}
\end{table}

\tikzstyle{cleartext} = [rectangle, rounded corners,minimum width=3cm, minimum height=1cm, text centered, draw=black, fill=orange!30]
\tikzstyle{plaintext} = [rectangle, rounded corners,minimum width=3cm, minimum height=1cm, text centered, draw=black, fill=purple!30]
\tikzstyle{ciphertext} = [rectangle, rounded corners,minimum width=3cm, minimum height=1cm, text centered, draw=black, fill=themecolor!30]
\tikzstyle{arrow} = [thick,->,>=stealth]

\begin{tikzpicture}
  \node[cleartext, align = center] (m) at (0,0) {message \\ $\vec{z}$};
  \node[plaintext, align = center] (p) at (6,0) {plaintext \\ $m(X)$};
  \node[ciphertext, align = center] (c) at (12,0) {ciphertext \\ $\vec{c} = (c_0(X), c_1(X))$};

  \node[cleartext, align = center] (m2) at (0,-2.5) {message \\ $\tilde{\vec{z}}=f(\vec{z})$};
  \node[plaintext, align = center] (p2) at (6,-2.5) {plaintext \\ $\tilde{m}(X) = f(m(X))$};
  \node[ciphertext, align = center] (c2) at (12,-2.5) {ciphertext \\ $\tilde{\vec{c}} = f(\vec{c})$};

  \draw (m.north) node[above]{$\C^{N/2}$};
  \draw (m2.south) node[below]{$\C^{N/2}$};
  \draw (p.north) node[above]{$\Z[X]/(X^N + 1)$};
  \draw (p2.south) node[below]{$\Z[X]/(X^N + 1)$};
  \draw (c.north) node[above]{$(\Z_{q_L}[X]/(X^N + 1))^2$};
  \draw (c2.south) node[below]{$(\Z_{q_L}[X]/(X^N + 1))^2$};

  \draw [arrow] (m) -- node[anchor=south] {encoding} (p) ;
  \draw [arrow] (p) -- node[anchor=south] {encryption} (c);
  \draw [arrow] (c) -- node[anchor=east] {computing $f$} (c2);
  \draw [arrow] (c2) -- node[anchor=south] {decryption} (p2);
  \draw [arrow] (p2) -- node[anchor=south] {decoding} (m2);
\end{tikzpicture}


\begin{theorem}{CKKS encryption is homomorphic with respect to addition}{ckks-enc-is-homomorphic}
  \cryptop{CKKS.Encrypt} should encrypt in such a way that the addition algebra can be retained even in the transformed space, showing that we can indeed refer to it as \hyperref[def:ring-homomorphism]{homomorphic} encryption.
\end{theorem}

\textit{Microsoft SEAL} implements the scheme, enabled using \cpp{seal::scheme_type::ckks}.
