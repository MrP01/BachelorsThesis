\chapter{Homomorphic Encryption}
\label{chap:homomorphic-encryption}

\gls{he} makes it possible to operate on data without knowing it.
One can distinguish three flavors of it, Partial-, Somewhat-, Levelled- and \gls{fhe}.

For \Gls{fhe}, there exist a few schemes in use today with existing implementations.
\begin{itemize}
  \item \gls{bfv} scheme for integer arithmetic \parencite{2012-fv-original, 2012-brakerski}.
  \item \gls{bgv} scheme for integer arithmetic \parencite{2012-bgv-original}.
  \item \gls{ckks} scheme for (complex) floating point arithmetic \parencite{2017-ckks-original}.
  \item \gls{fhew} scheme for Boolean circuit evaluation \parencite{2015-fhew-original}.
  \item \gls{tfhe} scheme for Boolean circuit evaluation \parencite{2019-tfhe-original}.
\end{itemize}

We will first introduce the BFV scheme (integer arithmetic) as it represents a fundamental building block behind CKKS.
Due to the inherent applications, this thesis will focus on the CKKS scheme to perform homomorphic operations on (complex-valued) floating point numbers and vectors.

To alleviate upcoming notation, for two tuples $(\cdot, \cdot)$ defined over the same ring, denote their element-wise addition as $(\cdot, \cdot) + (\cdot, \cdot)$, element-wise multiplication by a scalar $u$ as $u \cdot (\cdot, \cdot)$ and element-wise rounding as $\lfloor (\cdot, \cdot) \rceil$.

\section{Homomorphic Encryption using RSA}
In order to illustrate the basic idea behind \Gls{he}, without distancing ourselves too far from the original goal of introducing basic \gls{he} operations used in practice, this short section aims to motivate the definition of ring homomorphisms (cf. \cref{def:ring-homomorphism}) behind a cryptographic background.

With unpadded RSA \parencite{1983-rsa}, some arithmetic can be performed on the ciphertext - looking at the encrypted ciphertext $\mathcal{E}: \Z/q\Z \mapsto \Z/q\Z,\, \mathcal{E}(m) := m^r \mod q$ ($r, q \in \N$) of the message $m_1, m_2 \in \Z/q\Z$ respectively, the following holds:
\begin{align*}
  \mathcal{E}(m_1) \cdot \mathcal{E}(m_2)
   & \equiv (m_1)^r (m_2)^r \mod q            \\
   & \equiv (m_1 m_2)^r \mod q                \\
   & \equiv \mathcal{E}(m_1 \cdot m_2) \mod q
\end{align*}

\gls{rsa} encryption (even supporting an unbounded number of modular multiplications) therefore fulfils the properties of a multiplicative ring homomorphism, which in general terms is defined as follows:

\begin{definition}{Ring Homomorphism}{ring-homomorphism}
  Given two \hyperref[def:ring]{rings} $(R, +, \cdot)$ and $(S, \oplus, \otimes)$, we call a mapping $\varphi: R \rightarrow S$ a ring homomorphism when it satisfies the following conditions:
  $$\forall a, b \in R: \varphi(a + b) = \varphi(a) \oplus \varphi(b) \wedge \varphi(a \cdot b) = \varphi(a) \otimes \varphi(b)$$
\end{definition}

As we can see, the term \glsdesc{he} originates from the ability to perform computations on encrypted data while ensuring the same results are obtained when the same operations are applied to the original data.

\section{Gentry's FHE-Scheme and BGV}
Homomorphic encryption was envisioned by \citeauthor{1978-he-envisioned} as early as the 70s but remained a phantasm for almost three decades and was since referred to as the 'holy grail' of cryptography.
The first fully homomorphic encryption scheme was introduced in Craig \name{Gentry}'s PhD thesis, based on lattice problems \parencite{2009-gentry-fhe-original}.
In earlier schemes, each \gls{he} operation increases noise, making it partially homomorphic instead of fully homomorphic encryption.
\name{Gentry} devised a technique called \textit{bootstrapping} that evaluates the \textit{decryption} circuit homomorphically and thereby resets the noise introduced by previous operations, enabling true fully homomorphic encryption (\gls{fhe}).
Follow-up schemes improved his blueprint, Gentry's work is clearly a landmark achievement \parencite{2010-first-glimpse-of-fhe}.

\cite{2012-bgv-original} developed a generalisation of \gls{rlwe} that enables interpolation between \gls{lwe} and \gls{rlwe}, allowing for many improvements on earlier schemes, though mainly relying on \cite{2009-gentry-fhe-original}.
The resulting scheme, referred to as \gls{bgv}, allows for integer arithmetic (addition and multiplication on $\Z/q\Z$). It also employs a modulus reduction technique, greatly extending the homomorphic capacity to a broader class of homomorphic circuits.
\cite{seal-4.0} implements the scheme, enabled using \texttt{seal::scheme\_type::bgv}.

\pagebreak
\section{The BFV Scheme}
\label{sec:bfv}
This scheme was developed in two separate publications, whose authors initials it is named after, \cite{2012-brakerski} and \cite{2012-fv-original}.
\gls{bfv} is based on \gls{bgv} and they are really similar in their core ideas, one can even convert a BFV ciphertext to an equivalent BGV ciphertext \parencite{2021-he-revisiting}.
In this section, we will focus on a slightly altered implementation introduced in \cite{2014-fv-comparison}.

\subsection{Scheme Definition}
The \gls{bfv} scheme is a tuple of algorithms, introduced in \cref{def:bfv-scheme}.
To summarise the occurring parameters and variables, a brief overview of all used symbols is provided in \cref{tab:bfv-symbols}.

\begin{definition}{The BFV-Scheme}{bfv-scheme}
  Let $R := \Z[X] / \Phi_d(X)$ be a polynomial ring with $\Phi_d(X)$ the $d$\th \hyperref[def:cyclotomic-poly]{cyclotomic polynomial}
  ($\rightarrow d \in \N$) for ciphertexts $c \in R \times R$.
  Introduce $R / qR$ the associated quotient ring of the $q$\th coset of $R$ with the modulus $q \in \N$.
  Further let $t \in \N$ denote the message modulus with $1<t<q$
  for plain messages $m \in R/tR$ and define $\delta := \lfloor \frac{q}{t} \rfloor$,
  $\delta\inv = \frac{t}{q}$.

  Introduce three bounded discrete probability distributions $\chi_{key}$, $\chi_{enc}$ and $\chi_{error}$ over $R/qR$, one which is only used once for key generation, another used for \cryptop{BFV.Encrypt} and another (usually Gaussian-like) error distribution for manually inserted error terms (confer the \hyperref[def:lwe-search-problem]{LWE-problem}). For BFV, usually $\chi_{key} = \chi_{enc}$.

  For a polynomial $a \in R/qR$, consider the decomposition $a = \sum_{i=0}^{l-1} a_i w^i$ into base $w \in \N$ obtained by $\cryptop{WordDecomp}: R \mapsto R^l, \cryptop{WordDecomp}(a) := ([a_i]_w)_{i=0}^{l-1}$. \\
  Further let $\cryptop{PowersOf}: R \mapsto R^l$ be defined as $\cryptop{PowersOf}(a) := ([a w^i]_q)_{i=0}^{l-1}$.

  Let the parameters $\mathbb{P} := (d, q, t, \chi_{key}, \chi_{error}, w)$, $N := \varphi(d)$ and $l := \lfloor \log_w(q) \rfloor + 1$.
  \vspace{0.2cm}

  \cryptop{BFV.} \\
  \begin{tblr}{Q[l,h]Q[l,h,\textwidth - 3.5cm]}
    \cryptop{ParamGen}$(\lambda)$ & {
        Choose parameters as defined above, given the
        security parameter $\lambda$, such that $1 < t < q$, $w \geq 2$,
        initialize distributions $\chi_{key}$, $\chi_{enc}$ and $\chi_{error}$
        $\quad\rightarrow \mathbb{P}$} \\
    \cryptop{KeyGen}$(\mathbb{P})$ & {
        Generate the secret key $s \leftarrow \chi_{key}$, sample $\vec{\mu} \in (R/qR)^l$
        from $\chi_{error}$ and choose some $\vec{a} \in (R/qR)^l$ uniformly
        at random, compute the relinearisation key
        $\vec{\gamma} = (\cryptop{PowersOf}(s^2) - (\vec{\mu} + \vec{a} \cdot s), \vec{a})$
        and finally output the public key for uniformly random
        $a \in (R/qR)$ and $\mu \leftarrow \chi_{error}$ with $b =-(a \cdot s + \mu)$
        as $\vec{p} = (b, a)$.
        $\quad\rightarrow \vec{p}, s, \vec{\gamma}$} \\
    \cryptop{Encrypt}$(\vec{p}, m)$ & {
        Let $(b,a) = \vec{p}$, $u \leftarrow \chi_{enc}$, $\mu_1, \mu_2 \leftarrow \chi_{error}$,
        then the ciphertext is $\vec{c} = u \cdot \vec{p} + (\delta m + \mu_1, \mu_2) = (\delta m + bu + \mu_1, au + \mu_2)$
        $\quad\rightarrow \vec{c}$} \\
    \cryptop{Decrypt}$(s, \vec{c})$ & {
        Decrypt $\vec{c} = (c_0, c_1)$ as
        $m = \lfloor \delta\inv [c_0 + c_1 s]_t \rceil \in R/tR$
        $\quad\rightarrow m$} \\
    \cryptop{Add}$(\vec{c}_1, \vec{c}_2)$ & {
        Let $(c_0^1, c_1^1) = \vec{c}_1$ and $(c_0^2, c_1^2) = \vec{c}_2$
        then $\vec{c}_3 = (c_0^1 + c_0^2, c_1^1 + c_1^2) = \vec{c}_1 + \vec{c}_2$
        $\quad\rightarrow \vec{c}_3$} \\
    \cryptop{Mult}$(\vec{c}_1, \vec{c}_2)$ & {
        Output $\overline{\vec{c}} = (
          \lfloor \delta\inv c_0^1 c_0^2 \rceil,
          \lfloor \delta\inv(c_0^1 c_1^2 + c_1^1 c_0^2) \rceil,
          \lfloor \delta\inv c_1^1 c_1^2 \rceil
          )$
        $\quad\rightarrow \overline{\vec{c}}$} \\
    \cryptop{ReLin}$(\overline{\vec{c}}, \vec{\gamma})$ & {
        Using the relin key $\vec{\gamma} = (\vec{b}, \vec{a})$,
        relinearise from $\overline{\vec{c}} = (c_0, c_1, c_2)$ as
        $\vec{c} = (c_0 + \cryptop{WordDecomp}(c_2) \cdot \vec{b}, c_1 + \cryptop{WordDecomp}(c_2) \cdot \vec{a})$
        $\quad\rightarrow \vec{c}$} \\
  \end{tblr}

  \parencite{2012-fv-original, 2012-brakerski}
\end{definition}

\begin{table}[H]
  \centering
  \caption[Summary of the parameters and symbols in BFV]{Summary of the parameters and symbols in BFV.}
  \SetTblrInner{rowsep=0pt}
  \begin{tblr}{rll}
    \hline
    \textbf{Symbol} & \textbf{Space} & \textbf{Explanation} \\
    \hline
    $\lambda$ & $\in \R$ & Security parameter \\
    $d$ & $\in \N$ & Index of the cyclotomic polynomial used in $R$ \\
    $q$ & $\in \N$ & Modulus of the ciphertext space $R/qR$ \\
    $t$ & $\in \N$ & Modulus of the plaintext message space $R/tR$ \\
    $\delta$ & $\in \N$ & Ratio between ciphertext and plaintext modulus \\
    $\delta\inv$ & $\in \R$ & Inversion coefficient of the effect of $\delta$ \\
    $w$ & $\in \N$ & Word size used as basis, e.g. $w = 2$ for bits \\
    $l$ & $\in \N$ & Number of words of size $w$ required to encode $q$ \\
    $s$ & $\in R$ & Secret Key \\
    $\vec{p}$ & $\in (R/qR)^2$ & Public Key $(b, a)$ \\
    $\vec{\gamma}$ & $\in [(R/qR)^l]^2$ & Relinearisation Key \\
    $m$ & $\in R/tR$ & Plaintext Message\\
    $\vec{c}$ & $\in (R/qR)^2$ & Ciphertext \\
    $\overline{\vec{c}}$ & $\in (R/qR)^3$ & Slightly larger ciphertext resulting from multiplication \\
  \end{tblr}
  \label{tab:bfv-symbols}
\end{table}

Parameters in $\mathbb{P}$ described above need to be carefully chosen in order to provide for a certain security level $\lambda$ \footnote{for example, using \url{https://github.com/malb/lattice-estimator}}.
Also note that $q$ or $t$ do not need to be prime and could be chosen e.g. as powers of 2.
Encryption requires the public key, decryption the private key as usual in public-key encryption schemes.
The public key depends on the secret key and is chosen in such a way that the corresponding term cancels out when decrypting, as can be seen in \cref{subsec:bfv-verification}.

Homomorphic Addition, by design, works by simple addition of the corresponding ciphertexts.
Multiplication is based on a similar procedure, but tries to prevent an explosion of the scale by dividing through $\delta$ in all three terms.
Otherwise, the original input would be proportional to $\delta^2$ after multiplication, instead of $\delta$ as expected when decrypting.
The \cryptop{BFV.ReLin} operation then takes care of merging the three-term tuple back into a ciphertext made of two polynomials using the relinearisation key $\vec{\gamma}$.

The diagram in \cref{fig:bfv-overview} shows how a typical encryption process works and which ring each object is part of, also compare \cref{tab:bfv-symbols}.

\begin{figure}[H]
  \centering
  \inputtikz{figures/bfv-schematic}
  \caption[Schematic overview of the BFV scheme]{
    Schematic overview of the BFV scheme, adapted from \cite{2020-cryptotree}.
    A plaintext polynomial $m(X)$ is encrypted to the ciphertext $\vec{c} = \cryptop{BFV.Encrypt}(\vec{p}, m)$ using the public key $\vec{p}$, operated on using a combination of \cryptop{BFV.\{Add, Mult, ReLin\}} ciphertext operations and finally decrypted to a new $\tilde{m} = \cryptop{BFV.Decrypt}(s, \tilde{\vec{c}})$ using the secret key $s$.
    \vspace{0.8cm}
  }
  \label{fig:bfv-overview}
\end{figure}

\subsection{Verification of the Additive Homomorphism}
\label{subsec:bfv-verification}
\begin{theorem}{BFV encryption is homomorphic with respect to addition}{bfv-enc-is-homomorphic}
  \cryptop{BFV.Encrypt} should encrypt in such a way that the addition algebra can be retained even in the transformed space, showing that we can indeed refer to it as \hyperref[def:ring-homomorphism]{homomorphic} encryption.
\end{theorem}

\begin{proof}
  Starting out with two messages $m, m' \in R/tR$, two polynomials of degree $N-1$ with $N$ coefficients modulo $t$, we check whether addition of two ciphertexts $\vec{c} := \cryptop{BFV.Encrypt}(\vec{p}, m)$ and $\vec{c}' := \cryptop{BFV.Encrypt}(\vec{p}, m')$ indeed decrypts as $m + m'$.

  The client first creates a secret key $s$ and public key $\vec{p} = (b, a)$ with $b = -(as + \tilde{\mu})$ using $\cryptop{BFV.ParamGen}(\lambda)$ and $\cryptop{BFV.KeyGen}(\mathbb{P})$.
  Encrypting $m$ and $m'$ using the public key, we obtain
  $$\vec{c} = (c_0, c_1) = \begin{pmatrix}
      \delta m + b u + \mu_1 \\
      a u + \mu_2
    \end{pmatrix}^T \quad\text{and}\quad \vec{c}' = (c_0', c_1') = \begin{pmatrix}
      \delta m' + b u' + \mu_1' \\
      a u' + \mu_2'
    \end{pmatrix}^T \,.$$
  Evaluating $\overline{\vec{c}} := \cryptop{BFV.Add}(\vec{c}, \vec{c}') = \vec{c} + \vec{c}'$,
  $$\overline{\vec{c}} = \begin{pmatrix}
      \delta (m + m') + b (u + u') + (\mu_1 + \mu_1') \\
      a (u + u') + (\mu_2 + \mu_2')
    \end{pmatrix}^T = \begin{pmatrix}
      \delta \overline{m} + b \overline{u} + \overline{\mu_1} \\
      a \overline{u} + \overline{\mu_2}
    \end{pmatrix}^T$$
  we obtain a ciphertext that decrypts to the correct sum.
  Indeed,
  \begin{align*}
    \cryptop{BFV.Decrypt}(s, \overline{\vec{c}})
     & = \lfloor \delta\inv [\overline{c_0} + \overline{c_1} s]_t \rceil                                                                                                                                                         \\
     & = \big\lfloor \delta\inv [\delta \overline{m} + b \overline{u} + \overline{\mu_1} + (a \overline{u} + \overline{\mu_2}) s]_t \big\rceil                                                                                   \\
     & = \big\lfloor [(\delta\inv\delta) \overline{m} + \delta\inv b \overline{u} + \delta\inv \overline{\mu_1} + \delta\inv a s \overline{u} + \delta\inv \overline{\mu_2} s]_t \big\rceil                                      \\
     & = \big\lfloor [\overline{m} - \delta\inv (as + \tilde{\mu}) \overline{u} + \delta\inv \overline{\mu_1} + \delta\inv a s \overline{u} + \delta\inv \overline{\mu_2} s]_t \big\rceil                                        \\
     & = \big\lfloor [\overline{m} - \cancel{\delta\inv as \overline{u}} - \delta\inv \tilde{\mu} \overline{u} + \delta\inv \overline{\mu_1} + \cancel{\delta\inv as \overline{u}} + \delta\inv \overline{\mu_2} s]_t \big\rceil \\
     & = \big\lfloor [\overline{m} + \underbrace{\delta\inv (\overline{\mu_1} + \overline{\mu_2} s - \tilde{\mu} \overline{u})}_{:= \epsilon \,, ||\epsilon|| \ll 1}]_t \big\rceil
    \approx \big\lfloor [\overline{m}]_t \big\rceil = \lfloor \overline{m} \rceil \approx \overline{m}
  \end{align*}
  we arrive at the desired result $\overline{m} = m + m'$ after rounding ($\lfloor \cdot \rceil$) the (real) polynomial to a close element in $R/tR$ using one of several round-off algorithms  (cf. \cite{2013-rlwe-toolkit}).
  Of course, the influx of $\epsilon$ is only negligible if all parameters are carefully chosen as described in \cref{def:bfv-scheme} and the error terms are sufficiently small.
  $$t \ll q \Longrightarrow \delta\inv = t/q \ll 1$$ should be given while also ensuring that the spread of the distributions $\chi_{key}$, $\chi_{enc}$ and $\chi_{error}$ is not too large so that $\overline{\mu_{1,2}}, \overline{u}$ and $\tilde{\mu}$ do not lead to a large $\epsilon$ distorting our final result.
\end{proof}

As the public key $\vec{p} = (b, a)$ corresponds to a sample from the RLWE distribution (\cref{def:rlwe-dist}), the implied security of the BFV scheme is given by the hardness assumption of LWE (\cref{thm:lwe-hardness}).
An attacker trying to decrypt a ciphertext $\vec{c}$, given only the public key $\vec{p}$, would thereby need to solve the RLWE search problem (\cref{def:search-rlwe}) which is known to be hard \parencite{2010-rlwe-original}.

\cite{seal-4.0} implements the scheme, enabled using \texttt{seal::scheme\_type::bfv}.

\pagebreak
\section{The CKKS Scheme}
\label{sec:ckks}
The \gls{ckks} scheme allows us to perform approximate arithmetic on floating point numbers.
Essentially, the idea is to extend \gls{bgv} which allows us to operate on polynomials $p \in R/q_L R$, by an embedding approach that allows us to encode a (complex) floating point number vector $\vec{z} \in \R^n\, (\C^n)$ as an integer polynomial, similar to what is used in \gls{bfv}.
A main contribution of \gls{ckks} is the homomorphic rounding operation which allows us to reduce the scaling factors after multiplication.
The remaining scheme operations are extremely similar to the encryption, decryption, addition and multiplication in \gls{bfv} and \gls{bgv}.

Introduce $d, R, R/qR$ as in \cref{def:bfv-scheme} and further define $\mathcal{S} := \R[X] / \Phi_d(X) \subset R$ a similar polynomial ring to $R$, but over the reals instead of the integers.
Let $N := \varphi(d)$ be the degree of the reducing cyclotomic polynomial of $\mathcal{S}$, confer \cref{def:bounded-polynomials}.
For convenience, we usually choose $d$ a power of $2$ and then, by \cref{thm:power-of-2-cyclo-poly}, $N = \varphi(d) = \frac{d}{2}$ which yields efficiently multipliable polynomials because the homomorphic multiplication operation can be performed using a \gls{dft} and further optimized using the \gls{fft}, which in its unmodified form only accepts power-of-2 vector sizes \parencite{2017-ckks-original}.

\subsection{Encoding and Decoding}
In addition to encryption and decryption, the \gls{ckks} scheme also defines the \cryptop{CKKS.Encode} and \cryptop{CKKS.Decode} operations, extending possible plain inputs from polynomials $m \in R$ (as in BFV) to complex-valued vectors $\vec{z} \in \C^{N/2}$
\footnote{Many implementations of \gls{bfv} provide similar encoding and decoding procedures, extending the original BFV scheme \parencite{2012-fv-original} to facilitate encrypted vector arithmetic.}.
When encoding a vector of $N/2$ elements into a polynomial, a main goal is of course to ensure that addition and multiplication then correspond to elementwise vector addition and multiplication.
Furthermore, these vectors can be rotated (i.e. shifting the elements by an offset) using the \textit{Galois automorphism}.
We will not discuss it in detail here, nevertheless Galois rotations are heavily used in the implementation to facilitate effective matrix multiplications (confer \cref{sec:matmul}).
In total, the encoding and decoding steps consist of three transformations, $\underline{\pi}$, $\underline{\rho}$ and $\underline{\sigma}$.

\begin{definition}{Canonical Embedding $\underline{\sigma}$}{sigma-transform}
  For a real-valued polynomial $p \in \mathcal{S}$, define the canonical embedding of $\mathcal{S}$ in $\C^N$ as a mapping $\underline{\sigma}: \mathcal{S} \mapsto \C^N$ with $$\underline{\sigma}(p) := \big(p(e^{-2\pi i j / N})\big)_{j \in \Z_d^*}$$
  with $\Z_d^* := \{x \in \Z / d\Z \,|\, \gcd(x, d) = 1\}$ the set of all integers smaller than $d$ that do not share a factor $> 1$ with $d$.
  The image of $\underline{\sigma}$ given a set of inputs $R$ shall be denoted as $\underline{\sigma}(R) \subseteq \C^N$.
  Let the inverse of $\underline{\sigma}$ be denoted by $\underline{\sigma}\inv: \C^N \mapsto \mathcal{S}$.
\end{definition}

All elements of $R$ are also elements of $\mathcal{S}$ since $\Z \subset \mathcal{S}$ which results in $\underline{\sigma}(R) \subset \underline{\sigma}(\mathcal{S})$, every plaintext polynomial $m \in R$ can be encoded into $\underline{\sigma}(R)$.
Also note that evaluating a polynomial on the $n$\th roots of unity corresponds to performing a \name{Fourier}-Transform.

Define the commutative subring $(\mathbb{H}, +, \cdot)$ of $(\C^N, +, \cdot)$ on the set
$$\mathbb{H} := \{\vec{z} = (z_j)_{j \in \Z_d^*} \in \C^N : z_j = \overline{z_{-j}} \;\forall j \in \Z_d^*\} \subseteq \C^N$$
of all complex-valued vectors $\vec{z}$ where the first half equals the reversed complex-conjugated second half.

\begin{definition}{Natural Projection $\underline{\pi}$}{pi-transform}
  Let $T$ be a multiplicative subgroup of $\Z_d^*$ with $\Z_d^* / T = \{\pm 1\} = \{1T, -1T\}$, then the natural projection $\underline{\pi}: \mathbb{H} \mapsto \C^{N/2}$ is defined as
  $$\underline{\pi}\big((z_j)_{j \in \Z_M^*}\big) := (z_j)_{j \in T}$$
  Let its inverse be denoted by $\underline{\pi}\inv: \C^{N/2} \mapsto \mathbb{H}$ and consequently defined as
  $$\underline{\pi}\inv\big((z_j)_{j \in T}\big) := \big(\nu(z_j)\big)_{j \in \Z_M^*} \; \mathrm{with} \; \nu(z_j) = \begin{cases}
      z_j            & \text{if } j \in T \\
      \overline{z_j} & \text{otherwise}
    \end{cases}$$
\end{definition}

The natural projection $\underline{\pi}$ simply halves a vector $\vec{z} \in \mathbb{H}$ to all elements where $j \in T$ to only contain its essential information (the first half), since the second half can easily be reconstructed by element-wise conjugation using $\nu$.
The exact structure of $T$ is given by $\Z_d^* / T = \{\pm 1T\}$ with $+1T$ and $-1T$ denoting multiplicative left cosets of $T$, together forming the \hyperref[def:quotient-group]{quotient group} $(\Z_d^* / T, \cdot)$ over multiplication (denoted $\cdot$ instead of $+$ as in the quotient group definition in the previous chapter).

\paragraph{Further studying $T$.}
We first notice that by \name{Lagrange}'s theorem on finite groups, the number of elements in $T$ is exactly $N / 2$ since $$\frac{|\Z_d^*|}{|T|} = |\{\pm 1\}| \Leftrightarrow \frac{N}{|T|} = 2 \Leftrightarrow |T| = \frac{N}{2}$$ leading to $\underline{\pi}(\mathbb{H}) \subseteq \C^{N/2}$.
Rephrased, we seek a $T \subseteq \Z_d^*$ with $1 \in T$ such that we can fully construct $\Z_d^*$ by the union of the cosets $1T$ and $-1T$, i.e. $\Z_d^* = (1T) \cup (-1T)$.
Note that $T$ is not unique, we can find multiple sets $T$ for which the above holds, for instance by brute force computation:

\begin{minted}{python}
  import itertools, math, numpy as np
  d = 16; Zdstar = [z for z in range(d) if math.gcd(d, z) == 1]
  possible_T = [T for T in itertools.combinations(Zdstar, len(Zdstar) // 2)
    if 1 in T and list(np.unique(list(T) + [(-1*t) % d for t in T])) == Zdstar]
\end{minted}

\paragraph{Example.}
Let $d = 16$, then $\Z_d^* = \{1, 3, 5, 7, 9, 11, 13, 15\}$ and $N = |\Z_d^*| = 8$ and by \name{Lagrange}'s theorem, $|T| = 4$.
Since $(T, \cdot)$ forms a normal subgroup under multiplication, we must have that $1 \in T$ and we can identify all possible subgroups $T$ satisfying $\Z_d^* / T = \{\pm 1T\}$ to be one of
\begin{align*}
  \{1, 3, 5, 7\},
  \{1, 3, 5, 9\},
  \{1, 3, 7, 11\},
  \{1, 3, 9, 11\}, \\
  \{1, 5, 7, 13\},
  \{1, 5, 9, 13\},
  \{1, 7, 11, 13\},
  \{1, 9, 11, 13\}
\end{align*}
using the above Python code. An example of an invalid subset $T$ that does cover the whole original set $\Z_d^*$ would be $T = \{1, 7, 9, 15\}$.

For the purposes of \gls{ckks}, we simply choose a global $T$ that is constant for our encoding and decoding procedure and a given $d$.
The inverse natural projection $\underline{\pi}\inv$ then uniquely constructs a vector in $\mathbb{H}$ by filling in elements $\overline{z_j}$ for $j \notin T$ into $\vec{z}$.
For simplicity, we commonly choose $T$ as the 'first half' of $\Z_d^*$ when sorting in an ascending manner as it is always a valid choice
\footnote{
  This can be seen from the coset $-1T$ which exactly equals the 'missing' half in $\Z_d^*$ when the first half is covered by $1T = T = \{1, 3, 5, ..., N-1\}$ since $-1T = \{-1, -3, -5, ..., -(N-1)\} \equiv \{d-1, d-3, d-5, ..., d-N+1\}$ $(\text{mod}\, d)$ when $d$ a power of 2.
  Then, $(1T) \cup (-1T) = \{1, 3, 5, ..., N-1\} \cup \{d-1, d-3, d-5, ..., d-N+1\} = \{1, 3, 5, ..., N-1, N+1, ..., d-5, d-3, d-1\} = \Z_d^*$.
}.

\begin{definition}{Discretisation to an element of $\underline{\sigma}(R)$}{}
  Using one of several round-off algorithms (cf. \cite{2013-rlwe-toolkit}), given an element of $\mathbb{H}$, define a rounding operation $\underline{\rho}\inv: \mathbb{H} \mapsto \underline{\sigma}(R)$ that maps an $\vec{h} \in \mathbb{H}$ to its closest element in $\underline{\sigma}(R) \subset \mathbb{H}$, also denoted as
  $$\underline{\rho}\inv(\vec{h}) := \lfloor \vec{h} \rceil_{\underline{\sigma}(R)}\,.$$
  Further let $\underline{\rho_\delta}\inv(\vec{h}) = \lfloor \delta \cdot \vec{h} \rceil_{\underline{\sigma}(R)}$ denote the same rounding operation but with prior scaling by a scalar factor $\delta$.
  Note that $\underline{\rho}$ is given directly as the identity operation because all elements of its domain are already elements of its image. Similarly, $\underline{\rho_\delta}(\vec{y}) = \delta\inv \cdot \vec{y}$.
\end{definition}

Because it is not essential to understanding the encryption scheme, we will skip over concrete implementations of the rounding procedure $\underline{\rho}\inv$.
For choosing a \textit{close} element $\vec{g} \in \mathbb{H}$, we must first introduce a sense of proximity, in this case done by the $l_\infty$-norm $||\vec{g} - \vec{h}||_\infty$ of the difference between $\vec{h} \in \mathbb{H}$ and $\vec{g}$.

All in all, $m = \cryptop{CKKS.Encode}(\vec{z}),\, \vec{z} \in \C^{N/2}$ applies all inverse transformations $\underline{\pi}\inv$ (first), $\underline{\rho}\inv$ (second) and $\underline{\sigma}\inv$ (third) to an input vector $\vec{z}$ in order to finally arrive at a plaintext polynomial $m \in R/q_L R$ (equally stated as $m \in \Z_{q_L} / (X^N + 1)$ as long as $N$ is a power of 2).
Although $\underline{\sigma}$ is defined over $\mathcal{S}$ instead of $\mathcal{R}$, all elements of $\underline{\sigma}(R)$ can indeed be mapped back to an element in $R$ using $\underline{\sigma}\inv$.
Summarised,
$$\C^{N/2} \xlongrightarrow{\;\underline{\pi}\inv\;} \mathbb{H} \xlongrightarrow{\;\underline{\rho}\inv\;} \sigma(R) \xlongrightarrow{\;\underline{\sigma}\inv\;} R \,.$$
The decoding procedure $\vec{z} = \cryptop{CKKS.Decode}(m),\, m \in R$ does the opposite to re-obtain the input vector $\vec{z} \in \C^{N/2}$.

It should also be noted that the encoding procedure represents an isometric ring isomorphism (a linear bijection that preserves distance) between its domain and image, as does the decoding procedure.
This reflects in the observation that the plaintext sizes and errors are preserved under the transformations \parencite{2017-ckks-original}.

\pagebreak
\subsection{Scheme Definition}
The \gls{ckks} scheme is a tuple of algorithms, introduced in \cref{def:ckks-scheme}.
To summarise the occurring parameters and variables, a brief overview of all used symbols is provided in \cref{tab:ckks-symbols}.

\begin{definition}{The CKKS Scheme}{ckks-scheme}
  Define $R, R/q_L R$ as in \cref{def:bfv-scheme}.
  Introduce three bounded discrete probability distributions $\chi_{key}$, $\chi_{enc}$ and $\chi_{error}$ over $R/q_L R$.
  \vspace{0.2cm}

  \cryptop{CKKS.} \\
  \begin{tblr}{Q[l,h]Q[l,h,\textwidth - 3.5cm]}
    \cryptop{ParamGen}$(\lambda)$ & {
        Choose parameters as defined above, given the security parameter $\lambda$ and space modulus $q_L$, choose $d \in \N$ a power of $2$, $P, h \in \Z$, $\sigma \in \R$ and initialize distributions $\chi_{key}$, $\chi_{enc}$ and $\chi_{error}$.
        $\quad\rightarrow \mathbb{P}$} \\
    \cryptop{KeyGen}$(\mathbb{P})$ & {
        Sample the secret key $s \leftarrow \chi_{key}$, $a \in R_{q_L}$ uniformly at random, $\mu \leftarrow \chi_{error}$ and obtain the public key $\vec{p} = (b, a)$
        with $b = -a \cdot s + \mu$.
        Sample $a' \in R_{P \cdot q_L}$ uniformly at random, $\mu' \leftarrow \chi_{error}$
        and obtain the evaluation key $\vec{\gamma} = (b', a')$
        with $b' = -a' \cdot s + \mu' + Ps^2$.
        $\quad\rightarrow \vec{p}, s, \vec{\gamma}$} \\
    \cryptop{Encode}$(\vec{z})$ & {For a given input vector $\vec{z}$, output
        $m = (\underline{\sigma}\inv \circ \underline{\rho_\delta}\inv \circ \underline{\pi}\inv)(\vec{z}) = \underline{\sigma}\inv(\lfloor \delta \cdot \underline{\pi}\inv(\vec{z})\rceil_{\underline{\sigma}(R)})$ $\quad\rightarrow m$} \\
    \cryptop{Decode}$(m)$ & {Decode plaintext $m$ as
        $\vec{z} = (\underline{\pi} \circ \underline{\rho_\delta} \circ \underline{\sigma})(m) = (\underline{\pi} \circ \underline{\sigma})(\delta\inv m)$
        $\quad\rightarrow \vec{z}$} \\
    \cryptop{Encrypt}$(\vec{p}, m)$ & {
        Let $(b,a) = \vec{p}$, $u \leftarrow \chi_{enc}$, $\mu_1, \mu_2 \leftarrow \chi_{error}$,
        then the ciphertext is $\vec{c} = u \cdot \vec{p} + (m + \mu_1, \mu_2) = (m + bu + \mu_1, au + \mu_2)$
        $\quad\rightarrow \vec{c}$} \\
    \cryptop{Decrypt}$(s, \vec{c})$ & {
        Decrypt the ciphertext $\vec{c} = (c_0, c_1)$ as $m = \lbrack c_0 + c_1 s\rbrack_{q_L}$
        $\quad\rightarrow m$} \\
    \cryptop{Add}$(\vec{c}_1, \vec{c}_2)$ & {
        Output $\vec{c}_3 = \vec{c}_1 + \vec{c}_2$
        $\quad\rightarrow \vec{c}_3$} \\
    \cryptop{Mult}$(\vec{c}_1, \vec{c}_2)$ & {
        Output $\overline{\vec{c}} = (
          c_0^1 c_0^2,
          c_0^1 c_1^2 + c_1^1 c_0^2,
          c_1^1 c_1^2
          )$
        $\quad\rightarrow \overline{\vec{c}}$} \\
    \cryptop{ReLin}$(\overline{\vec{c}}, \vec{\gamma})$ & {
        Using the evaluation key $\vec{\gamma}$,
        relinearise from $\overline{\vec{c}} = (c_0, c_1, c_2)$ to
        $\vec{c} = (c_0, c_1) + \lfloor P\inv c_2 \vec{\gamma} \rceil$
        $\quad\rightarrow \vec{c}$} \\
    \cryptop{ReScale}$(\vec{c})$ & {
    In order to rescale a ciphertext from level $l_{old}$ to $l_{new}$, multiply by a factor $\frac{q_{l_{new}}}{q_{l_{old}}} \in \Q$ and round to the nearest element of $(R/q_{l_{new}} R) \times (R/q_{l_{new}} R)$:
    $\vec{c}_{new} = \big\lfloor \frac{q_{l_{new}}}{q_{l_{old}}} \vec{c} \big\rceil$ $\quad\rightarrow \vec{c_{new}}$
    } \\
  \end{tblr}

  \parencite{2017-ckks-original}
\end{definition}

For more details on the probability distributions, refer to the original CKKS paper \parencite{2017-ckks-original}, with the following naming relations:
$\chi_{key} = \mathcal{H}WT(h)$ over $\{0,\pm 1\}^N$,
$\chi_{error} = \mathcal{DG}(\sigma^2)$ over $\Z^N$ and
$\chi_{enc} = \mathcal{ZO}(0.5)$ another distribution over $\{0,\pm 1\}^N$.

Encryption, decryption, addition and multiplication work similarly as in \cref{def:bfv-scheme}.
Unlike \gls{bfv} however, \gls{ckks} works with different moduli $q_L$ at each level $L$.
This also casts off the need for up- and downscaling by $\delta$ when multiplying ciphertexts.
The \cryptop{CKKS.ReLin} operation serves the same purpose as in \gls{bfv}, the rescaling operation is new however and takes care of the scale management.
\cryptop{CKKS.ReScale} can be employed whenever we want to work with ciphertexts at different levels, modifying a given ciphertext to the scale of the other enables the other operations to work just as usual.
\Cref{fig:ckks-overview} shows the extended course of action in \gls{ckks} with a preceding encoding and decoding step.

\begin{table}[H]
  \centering
  \caption[Summary of the parameters and symbols in CKKS]{Summary of the parameters and symbols in CKKS.}
  \SetTblrInner{rowsep=0pt}
  \begin{tblr}{rll}
    \hline
    \textbf{Symbol} & \textbf{Space} & \textbf{Explanation} \\
    \hline
    $\lambda$ & $\in \R$ & Security parameter \\
    $d$ & $\in \N$ & Index of the cyclotomic polynomial used in $R$ \\
    $P$ & $\in \Z$ & Factor used during relinearisation \\
    $h$ & $\in \Z$ & Hamming weight of the secret key (used by $\chi_{key}$) \\
    $\sigma$ & $\in \R$ & Standard deviation of the Gaussian $\chi_{error}$ \\
    $q_L$ & $\in \N$ & Modulus of $R/q_L R$ at level $L$ \\
    $\delta$ & $\in \N$ & Scaling factor used when encoding \\
    $\delta\inv$ & $\in \R$ & Inversion coefficient of the effect of $\delta$ \\
    $s$ & $\in \{0,\pm 1\}^N$ & Secret Key \\
    $\vec{p}$ & $\in (R/q_L R)^2$ & Public Key $(b, a)$ \\
    $\vec{\gamma}$ & $\in (R/(P q_L)R)^2$ & Relinearisation Key \\
    $\vec{z}$ & $\in \C^{N/2}$ & Plain input vector \\
    $m$ & $\in R$ & Plaintext Message \\
    $\vec{c}$ & $\in (R/q_L R)^2$ & Ciphertext Message \\
    $\overline{\vec{c}}$ & $\in (R/q_L R)^3$ & Slightly larger ciphertext from multiplication \\
  \end{tblr}
  \label{tab:ckks-symbols}
\end{table}

\begin{figure}[H]
  \centering
  \inputtikz{figures/ckks-schematic}
  \caption[Schematic overview of the CKKS scheme]{
    Schematic overview of CKKS, adapted from \cite{2020-cryptotree}.
    A plain vector $\vec{z} \in \C^{N/2}$ is encoded to a plaintext polynomial $m = \cryptop{CKKS.Encode}(\vec{z})$, encrypted to the ciphertext $\vec{c} = \cryptop{CKKS.Encrypt}(\vec{p}, m)$ using the public key $\vec{p}$, operated on using a combination of \cryptop{CKKS.\{Add, Mult, ReLin, ReScale\}} ciphertext operations and finally decrypted and decoded to a new $\tilde{\vec{z}} = \cryptop{CKKS.Decode}(\cryptop{CKKS.Decrypt}(s, \tilde{\vec{c}}))$ using the secret key $s$.
  }
  \label{fig:ckks-overview}
\end{figure}

\subsection{Verification of the Additive Homomorphism}
\begin{theorem}{CKKS encryption is homomorphic with respect to addition}{ckks-enc-is-homomorphic}
  \cryptop{CKKS.Encode} and \cryptop{CKKS.Encrypt} should encrypt in such a way that the addition algebra can be retained even in the transformed space, showing that we can indeed refer to it as \hyperref[def:ring-homomorphism]{homomorphic} encryption.
\end{theorem}

\begin{proof}
  Similar to the BFV scheme proof (\cref{thm:bfv-enc-is-homomorphic}), we aim to show that two input vectors $\vec{z}, \vec{z}' \in \C^{N/2}$ can be encoded, encrypted and added - and finally decrypted back to $\overline{\vec{z}} = \vec{z} + \vec{z}'$.

  Due to the extremely high similarity of the BFV and CKKS schemes, they are even identical in their encryption, decryption and adding procedures, the only thing that remains to be shown is the additivity (or even linearity) of \cryptop{CKKS.Encode}.

  % As a first step, the client generates the secret key $s$ and public key $\vec{p} = (b, a)$ with $b = -(as + \overline{\mu})$ using $\cryptop{CKKS.ParamGen}(\lambda)$ and $\cryptop{CKKS.KeyGen}(\mathbb{P})$.
  Encoding $\vec{z}$ and $\vec{z}'$ into $m$ and $m'$, we obtain
  $$m := \cryptop{CKKS.Encode}(\vec{z}) = (\underline{\sigma}\inv \circ \underline{\rho_\delta}\inv \circ \underline{\pi}\inv)(\vec{z})$$
  comprised of three transformations $\underline{\sigma}\inv$, $\underline{\rho_\delta}\inv$ and $\underline{\pi}\inv$ which can be studied separately for their approximate additivity.
  If a function is linear and bijective (turning it into an isomorphism), its inverse will also be linear.
  We will utilize this below by only showing the additivity of $\underline{\sigma}$, $\underline{\rho_\delta}$ and $\underline{\pi}$, assuming their (approximate) bijectivity.
  Especially the bijectivity of $\underline{\sigma}$ is cumbersome to show and for more details we refer the reader to \cite{2017-ckks-original}.

  \begin{itemize}
    \item The canonical embedding $\underline{\sigma}$ evaluates an input polynomial on the $N$\th roots of unity $\{\xi_j\}_{j \in \Z_d^*}$.
          For any two polynomials $p_1, p_2 \in \mathcal{S}$, $$\underline{\sigma}(p_1 + p_2) = \big((p_1 + p_2)(\xi_j)\big)_{j \in \Z_d^*} = \big(p_1(\xi_j) + p_2(\xi_j)\big)_{j \in \Z_d^*} = \underline{\sigma}(p_1) + \underline{\sigma}(p_2) \,.$$
    \item The rounding operation $\underline{\rho_\delta}\inv$ is only approximately additive due to its nature\footnote{For an illustrative counterexample of the additivity of rounding, refer to \url{https://math.stackexchange.com/questions/58239/linear-functions-with-rounding}.}.
          For any two vectors $\vec{h_1}, \vec{h_2} \in \mathbb{H}$,
          \begin{align*}
            \underline{\rho_\delta}\inv(\vec{h_1} + \vec{h_2})
             & = \lfloor \delta \cdot (\vec{h_1} + \vec{h_2}) \rceil_{\underline{\sigma}(R)}
            = \lfloor \delta \vec{h_1} + \delta\vec{h_2} \rceil_{\underline{\sigma}(R)}
            \approx \lfloor \delta \vec{h_1} \rceil_{\underline{\sigma}(R)} + \lfloor \delta\vec{h_2} \rceil_{\underline{\sigma}(R)} \\
             & \approx \underline{\rho_\delta}\inv(\vec{h_1}) + \underline{\rho_\delta}\inv(\vec{h_2})\,.
          \end{align*}
          Its inverse $\underline{\rho_\delta}$ is fully additive nevertheless, since it acts as the identity (up to a scalar factor) of the subset $\underline{\sigma}(R)$ back to an element of $\mathbb{H}$.
    \item The natural projection $\underline{\pi}$ halves a vector in $\mathbb{H}$ to an element of $\C^{N/2}$ which is naturally linear.
          Consider any $\vec{h_1}, \vec{h_2} \in \mathbb{H}$, then
          $$\underline{\pi}(\vec{h_1} + \vec{h_2}) = ({h_1}_j + {h_2}_j)_{j \in T} = ({h_1}_j)_{j \in T} + ({h_2}_j)_{j \in T} = \underline{\pi}(\vec{h_1}) + \underline{\pi}(\vec{h_2})\,.$$
  \end{itemize}

  As every step in the full encoding process $\underline{\sigma}\inv \circ \underline{\rho_\delta}\inv \circ \underline{\pi}\inv$ is additive, \cryptop{CKKS.Encode} indeed acts additively.
  \cryptop{CKKS.Decode} on the other hand is only approximately additive due to the rounding operation required in between.

  The rest follows from \cref{thm:bfv-enc-is-homomorphic} as encryption and addition of the BFV scheme are identical.
  All in all for \cryptop{CKKS}, using the secret key $s$ and public key $\vec{p}$,
  $$\cryptop{Decode}(
    \cryptop{Decrypt}(s,
    \cryptop{Add}(
    \cryptop{Encrypt}(\vec{p}, \cryptop{Encode}(\vec{z})),
    \cryptop{Encrypt}(\vec{p}, \cryptop{Encode}(\vec{z}'))
    ))) \approx \vec{z} + \vec{z}' \,.$$
  We can conclude that encoding \textit{and} encryption in CKKS are indeed homomorphic with respect to addition.
\end{proof}

As the public key $\vec{p} = (b, a)$ corresponds to a sample from the RLWE distribution (\cref{def:rlwe-dist}), the implied security of the CKKS scheme is given by the hardness assumption of LWE (\cref{thm:lwe-hardness}).
An attacker trying to decrypt a ciphertext $\vec{c}$, given only the public key $\vec{p}$, would thereby need to solve the RLWE search problem (\cref{def:search-rlwe}) which is known to be hard \parencite{2010-rlwe-original}.

\cite{seal-4.0} implements the scheme, enabled using \texttt{seal::scheme\_type::ckks}.
