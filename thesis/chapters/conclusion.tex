\chapter{Conclusion}
\label{chap:conclusion}

In the present thesis, we explored the interesting realm of homomorphic encryption in a machine-learning context while considering various security aspects of it.
Next to the written part, the reader may find a server-client ('as a service') based demonstrator implementation of the homomorphically encrypted machine learning service, including a web-based frontend.

\section{Summary}
\Cref{chap:background} introduced mathematical preliminaries of the lattice-based homomorphic encryptions schemes used in \cref{chap:homomorphic-encryption}.
As the main focus lies on thoroughly understanding the notation used in the encryption schemes, we first concentrated on the algebraic ring $\Z_q[X]/(X^N+1)$ with coefficients modulo $q$ and the cyclotomic polynomial $(X^N+1)$ as its modulus.
Forming the basis of the upcoming cryptographic schemes, we introduced the \gls{lwe} problem and its descendant, \gls{rlwe}.
The multi-layered neural network is trained using the backpropagation algorithm of the Tensorflow library, an approach similar to gradient descent - which was all introduced in the next section of \cref{chap:background}.
The final \cref{sec:post-quantum-sec} introduced the reader to basic quantum mechanics and most of the necessary background to understand a simple quantum-mechanical system and the underlying principles of modern quantum computers.

\Cref{chap:homomorphic-encryption}
\Cref{chap:implementation}
\Cref{chap:results}

\Cref{chap:appendix}

Considering the implications of mass surveilance, the importance of privacy-preserving/enhancing technologies should not be forgotten.


\section{Outlook}
\todo{To be written: describe existing solutions, approaches, current research, etc.}
% describe existing solutions, approaches, current research, etc.
% -> list the papers in library/ folder?
% (include Fabians master thesis about splitting Relin, Galois keys using SPDZ
% to support multiple data providers (clients) and one server, using normal HE algorithms)

\section{Related Works}
\todo{Vielleicht als kleiner Teaser für mehr Literatur?}

Gazelle (inferred ML) as described by \cite{2018-gazelle}.

Random Forests (RF) on HE as described by \cite{2020-cryptotree}.
