\chapter{Conclusion}
\label{chap:conclusion}

In the present thesis, we explored the interesting realm of homomorphic encryption in a machine-learning context while considering various security aspects of it.
Next to the written part, the reader may find a server-client ('as a service') based demonstrator implementation of the homomorphically encrypted machine learning service for classifying handwritten digits, including a web-based frontend.

\section{Summary}
\Cref{chap:background} introduced mathematical preliminaries of the lattice-based homomorphic encryptions schemes used in \cref{chap:homomorphic-encryption}.
As the focus lies on thoroughly understanding the notation used in the encryption schemes, we first concentrated on the algebraic ring $\Z_q[X]/(X^N+1)$ with coefficients modulo $q$ and the cyclotomic polynomial $(X^N+1)$ as its modulus.
Forming the basis of the upcoming cryptographic schemes, we introduced the \gls{lwe} problem and its descendant, \gls{rlwe}.
The multi-layered neural network is trained using the backpropagation algorithm of the Tensorflow library, an approach similar to gradient descent - which was all introduced in the next section of \cref{chap:background}.
The final \cref{sec:post-quantum-sec} introduced the reader to basic quantum mechanics and most of the necessary background to understand a simple quantum-mechanical system and the underlying principles of modern quantum computers.

\Cref{chap:homomorphic-encryption} starts out with one of the first homomorphic schemes there is, unpadded \gls{rsa}, motivating the need for a such structure.
The first \glsdesc{fhe} scheme, invented by Craig \name{Gentry} in his PhD thesis, strongly inspired further developments in the field:
Based on lattice problems and the \gls{rlwe} cryptosystem, \citeauthor{2012-brakerski,2012-fv-original} introduced the \gls{bfv} scheme for integer arithmetic.
Extending it, we arrived at \gls{ckks} scheme developed by \citeauthor{2017-ckks-original} with an extensive encoding and decoding procedure.
For both schemes, we verified the additive homomorphisms of the corresponding encryption (and encoding) operations \cryptop{BFV.Encrypt} and $\cryptop{CKKS.Encrypt} \circ \cryptop{CKKS.Encode}$.

\Cref{chap:implementation} then presents details about the implementation of the web-based demonstrator, concerning the deployment and software architectural specifics.
The focus however lies on the technical challenges with \glsdesc{he} in practice.
Four different matrix multiplication techniques were introduced in detail, gradually increasing the complexity but also the efficiency.
The next challenge ahead was evaluating the activation function in between the two layers, due to technical reasons, only approximated by a series expansion.
\Cref{chap:results} presents the network's training results and a detailed statistical analysis of the network performance.
To explicitly show the 'garbling' property of the encryption, the first polynomial of the ciphertext is visualised for a few test samples (\cref{fig:ciphertext-visualisation}).
Finally, we analysed the runtime, network communication and numerical performance of the algorithms for different parameters, summarised in \cref{tab:performance-benchmarks}.

Two longer proofs were outhoused to the \Cref{chap:appendix}, one on the power-of-2 cyclotomic polynomials and a second one on the \gls{bsgs} matrix multiplication method.

\section{Related Works and Outlook}
There are many existing approaches of \gls{ppml} schemes or structures and even more applications - some related publications are, for the curious reader:
\begin{itemize}
  \item CryptoNets (inferred \gls{ml}), confer \cite{2016-cryptonets}.
  \item Gazelle (inferred \gls{ml}) as described by \cite{2018-gazelle}.
  \item \gls{tfhe} neural network inference, confer \cite{2019-tfhe-original}.
  \item Random Forests using \gls{he} as described by \cite{2020-cryptotree}.
  \item Pasta for hybrid homomorphic encryption as described in \cite{2021-pasta}.
\end{itemize}

Next to Microsoft \gls{seal}, there are also other existing libraries for \gls{he}:
\begin{itemize}
  \item Microsoft's PyEVA, an interface to SEAL's CKKS implementation.
  \item Halevi \& Shoup's HELib, a separate implementation of \gls{he} algorithms using NTL (the number theoretic library).
  \item Palisade, another collection of many implemented \gls{he} algorithms, supposed to be a general-purpose tool.
  \item TenSEAL from OpenMined, a high-level Python interface to SEAL.
  \item PyGrid and PySyft from OpenMined, libraries to publish encrypted learning data (PyGrid) and utilize it (PySyft) for \gls{ml}.
\end{itemize}

Considering the implications of mass surveilance and the thereby growing mindset for data privacy, the relevance of privacy-preserving/enhancing technologies will definitely grow in the coming years.
The relatively new class of lattice cryptosystems looks immensely promising, given the homomorphic properties of many such systems, and the inherent quantum security that will become even more relevant in the near future.
