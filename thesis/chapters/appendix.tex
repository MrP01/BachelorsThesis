\chapter*{Appendix}
\addcontentsline{toc}{chapter}{Appendix}

\begin{proof}[Proof of \autoref{thm:power-of-2-cyclo-poly}]
  With $k \in \N$ a positive integer, we want to show that
  $$\Phi_{2^k} (x) = x^{2^{k-1}} + 1\,.$$

  A polynomial $p \in \Z[X]$ with $$p(x) = x^n - a$$ of degree $n$ has $n$ roots
  $$\{x_j\} = \{a^\frac{1}{n} e^{2\pi i \frac{j}{n}} \,|\, j \in \N\}$$
  related by a factor $a^\frac{1}{n}$ to the
  \hyperref[lemma:nth-roots-of-unity]{$n$\textsuperscript{th} roots of unity} given by powers of
  $\xi = e^{2\pi i \frac{1}{n}}$.

  It is clear from the fundamental theorem of algebra that the polynomial $p$ with roots $\{x_k\}$
  can be factorised as
  $$p(x) = \prod_{j=1}^{n} (x - x_j) = \prod_{j=1}^{n} (x - a^\frac{1}{n} e^{2\pi i \frac{j}{n}})\,.$$

  Fixing $a = -1$, we obtain $p(x) = x^n + 1$ with roots given by
  $$x_j = (-1)^\frac{1}{n} e^{2\pi i \frac{j}{n}}
    = (e^{i\pi})^\frac{1}{n} e^{2\pi i \frac{j}{n}}
    = e^{\frac{i\pi (2j + 1)}{n}}$$
  and according factorisation
  $$p(x) = \prod_{j=1}^{n} (x - e^{\frac{i\pi}{n} (2j + 1)})\,.$$

  Further letting $n = 2^{k-1}$ and observing that
  $$\gcd(2^k, l) = \begin{cases}
      1 & \text{if } l \text{ odd}  \\
      2 & \text{if } l \text{ even}
    \end{cases} \quad l, k \in \N$$
  since a number $2^k$ that can only be decomposed into multiples of $2$
  never shares a factor with an odd number, in accordance with \autoref{lemma:nth-roots-of-unity}
  we can conclude that the set of all odd roots of unity is exactly the set of all primitive roots.

  Following from above,
  \begin{align*}
    p(x) & = \prod_{j=1}^{2^{k-1}} (x - e^{\frac{i\pi}{n} (2j + 1)})                \\
         & = \prod_{\stackrel{l=1}{l \text{ odd}}}^{2^k} (x - e^{\frac{i\pi}{n} l}) \\
         & = \prod_{\stackrel{l=1}{x_l \text{ primitive}}}^{2^k} (x - \xi^l)
    = \Phi_{2^k}(x)
  \end{align*}
  we arrive exactly at the definition of a cyclotomic polynomial (\autoref{def:cyclotomic-poly}). \\
  \parencite{power-of-2-cyclo-poly}
\end{proof}
