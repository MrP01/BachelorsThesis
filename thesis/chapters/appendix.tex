\appendix
\titleformat{\chapter}[block]{\normalfont\LARGE\bfseries}{Appendix \thechapter \;\textendash\;}{0ex}{\vspace{-4cm}}[\vspace{4.5cm}]
\titlespacing{\chapter}{0cm}{0cm}{0cm}

\chapter{Missing Proofs}
\label{chap:appendix}

\section{Power-of-2 Cyclotomic Polynomials}
\label{proof:power-of-2-cyclo-poly}
\begin{proof}[Proof of \autoref{thm:power-of-2-cyclo-poly}]
  With $k \in \N$ a positive integer, we want to show that
  $$\Phi_{2^k} (x) = x^{2^{k-1}} + 1\,.$$

  A polynomial $p \in \Z[X]$ with $$p(x) = x^n - a$$ of degree $n$ has $n$ roots
  $$\{x_j\} = \{a^\frac{1}{n} e^{2\pi i \frac{j}{n}} \,|\, j \in \N, j \leq n\}$$
  related by a factor $a^\frac{1}{n}$ to the \hyperref[lemma:nth-roots-of-unity]{$n$\textsuperscript{th} roots of unity} given by powers of $\xi = e^{2\pi i \frac{1}{n}}$.

  It is clear from the fundamental theorem of algebra that the polynomial $p$ with roots $\{x_j\}$ can be factorised as
  $$p(x) = \prod_{j=1}^{n} (x - x_j) = \prod_{j=1}^{n} (x - a^\frac{1}{n} e^{2\pi i \frac{j}{n}})\,.$$

  Fixing $a = -1$, we obtain $p(x) = x^n + 1$ with roots given by
  $$x_j = (-1)^\frac{1}{n} e^{2\pi i \frac{j}{n}}
    = (e^{i\pi})^\frac{1}{n} e^{2\pi i \frac{j}{n}}
    = e^{\frac{i\pi (2j + 1)}{n}}$$
  and according factorisation
  $$p(x) = \prod_{j=1}^{n} (x - e^{\frac{i\pi}{n} (2j + 1)})\,.$$

  Further letting $n = 2^{k-1}$ and observing that
  $$\gcd(2^k, l) = \begin{cases}
      1 & \text{if } l \text{ odd}  \\
      2 & \text{if } l \text{ even}
    \end{cases} \quad l, k \in \N$$
  since a number $2^k$ that can only be decomposed into multiples of $2$
  never shares a factor with an odd number, in accordance with \autoref{lemma:nth-roots-of-unity}
  we can conclude that the set of all odd roots of unity is exactly the set of all primitive roots (satisfying $\gcd(2^k, l) = 1$).

  Following from above,
  \begin{align*}
    p(x) = \prod_{j=1}^{2^{k-1}} (x - e^{\frac{i\pi}{n} (2j + 1)})
    = \prod_{\stackrel{l=1}{l \text{ odd}}}^{2^k} (x - e^{\frac{i\pi}{n} l})
    = \prod_{\stackrel{l=1}{\xi^l \text{ primitive}}}^{2^k} (x - \xi^l)
    = \Phi_{2^k}(x)
  \end{align*}
  we arrive exactly at the definition of a cyclotomic polynomial (\autoref{def:cyclotomic-poly}). \\
  \parencite{power-of-2-cyclo-poly}
\end{proof}

\section{Babystep-Giantstep Multiplication}
\label{proof:bsgs-matmul}
\begin{proof}[Proof of \autoref{thm:bsgs}]
  Starting from the adapted matrix-multiplication expression $P = (P_1, P_2, ..., P_t)^T \in \R^t$, we want to show that we indeed end up with an authentic matrix-vector product.
  \begin{align*}
    P = \bigg\{\sum_{k=0}^{t_2-1} \rot_{(kt_1)} \big(
    \sum_{j=0}^{t_1-1} \diag'_{(kt_1+j)}(M) \cdot \rot_j(\vec{x})
    \big)\bigg\}_i = \sum_{k=0}^{t_2-1} \sum_{j=0}^{t_1-1} m'_{kt_1+j,(i+kt_1)} x_{(i+kt_1)+j}
  \end{align*}
  with
  \begin{align*}
    m'_{p,i} = \big\{ \diag'_p(M)\big \}_i = \big\{ \rot_{-\lfloor p/t_1 \rfloor \cdot t_1}(\diag_p(M)) \big\}_i
    = M_{i-\lfloor\frac{p}{t_1}\rfloor t_1, i-\lfloor\frac{p}{t_1}\rfloor t_1 + p}
  \end{align*}
  and therefore
  \begin{align*}
    m'_{kt_1+j,i}        & = M_{i-\lfloor\frac{kt_1+j}{t_1}\rfloor t_1, i-\lfloor\frac{kt_1+j}{t_1}\rfloor t_1 + kt_1+j} \\
                         & = M_{i-kt_1-\lfloor\frac{j}{t_1}\rfloor t_1, i-kt_1-\lfloor\frac{j}{t_1}\rfloor t_1 + kt_1+j} \\
                         & = M_{i-kt_1-\lfloor\frac{j}{t_1}\rfloor t_1, i+j-\lfloor\frac{j}{t_1}\rfloor t_1}             \\
    m'_{kt_1+j,(i+kt_1)} & = M_{i+kt_1-kt_1-\lfloor\frac{j}{t_1}\rfloor t_1, i+kt_1+j-\lfloor\frac{j}{t_1}\rfloor t_1}   \\
                         & = M_{i-\lfloor\frac{j}{t_1}\rfloor t_1, i+kt_1+j-\lfloor\frac{j}{t_1}\rfloor t_1}
  \end{align*}
  leading to
  \begin{align*}
    P_i = \sum_{k=0}^{t_2-1} \sum_{j=0}^{t_1-1} m'_{kt_1+j,(i+kt_1)} x_{(i+kt_1)+j}
    = \sum_{k=0}^{t_2-1} \sum_{j=0}^{t_1-1} M_{i-\lfloor\frac{j}{t_1}\rfloor t_1, i+kt_1+j-\lfloor\frac{j}{t_1}\rfloor t_1} x_{(i+kt_1)+j} \,.
  \end{align*}
  Noticing that the downward rounded fraction $\lfloor\frac{j}{t_1}\rfloor$ vanishes
  in a sum with $j$ running from $0$ to $t_1-1$, we can simplify to
  \begin{align*}
    P_i = \sum_{k=0}^{t_2-1} \sum_{j=0}^{t_1-1} M_{i,i+kt_1+j} x_{i+kt_1+j}
  \end{align*}
  which contains two sums running to $t_1$ and $t_2$ respectively, containing an expression of the form $k \cdot t_1 + j$, which allows us to condense the nested sums into one single summation expression, as $$\sum_{k=0}^{t_2-1} \sum_{j=0}^{t_1-1} f(kt_1+j) = \sum_{l=0}^{t-1} f(l)$$ indeed catches every single value $l \in \{0, 1, 2, ..., t=t_1 \cdot t_2\}$ with $l = kt_1+j$. \\
  In summary, we obtain
  \begin{align*}
    P_i & = \sum_{k=0}^{t_2-1} \sum_{j=0}^{t_1-1} M_{i,i+kt_1+j} x_{i+kt_1+j} \\
        & = \sum_{l=0}^{t-1} M_{i,i+l} x_{i+l}
    = \sum_{\nu=0}^{t-1} M_{i,\nu} x_{\nu}                                    \\
        & = \big\{M \vec{x}\big\}_i
  \end{align*}
  which indeed equals the conventional definition of a matrix-vector product.
\end{proof}
