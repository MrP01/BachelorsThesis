\chapter{Introduction}
\label{chap:introduction}
The most well-known and widely used asymmetric ('public-key') cryptographic scheme, published by the trio \name{Rivest}-\name{Shamir}-\name{Adleman} in 1977 and known as \textit{RSA}, is based on the hardness assumption of the integer factorisation problem (factorising a large 2-composite number into its two prime factors $p$ and $q$ is hard).
As of today, this factorisation problem has not been proven to be in NP, yet it is suspected that it might indeed be NP-complete (i.e. \hyperref[def:np-hard]{NP-hard} while still being in NP) when modelled using a traditional Turing machine.
Since the advent of quantum computation, this situation changed as a whole with Peter \name{Shor}'s algorithm, threatening the security of many cryptosystems, for instance RSA which is still widely used today despite its known problems.

As it stands, lattice-based cryptography presents a solution to a politically and socially problematic situation in which few parties world-wide, with access to a sufficiently powerful quantum computer, may be able to decrypt most of today's digital communication.
\hyperref[subsec:lattice-crypto]{Lattice Cryptography} is based on other mathematical problems, shown to be sufficiently hard on quantum computers and traditional ones alike, most notably \hyperref[def:lwe-search-problem]{LWE} which this thesis will discuss in detail.

Many new cryptosystems have been developed on top of LWE, two of which this following thesis will focus on specifically: \hyperref[def:bfv-scheme]{BFV} and \hyperref[def:ckks-scheme]{CKKS};
whose security is still unaffected by efficient quantum algorithms.
Yet, it is not only their security prospect that makes these encryption schemes attractive, but first and foremost their defining \hyperref[def:ring-homomorphism]{homomorphic} property which allows for computations on the encrypted data.
A \textit{fully} homomorphic encryption (FHE) scheme was first introduced by Craig \name{Gentry} in 2009, using a bootstrapping approach.
The \textit{levelled} homomorphic \gls{bgv} encryption scheme is implemented in Microsoft SEAL and allows for integer arithmetic, up to a few multiplication 'levels' deep.
The \gls{bfv} scheme is very similar to it and described in a bit more detail in \autoref{sec:bfv}.
And finally, building upon concepts introduced in the former, the \gls{ckks} scheme allows for approximative floating-point arithmetic that finally facilitates machine-learning applications.

Machine Learning allows a computer to 'learn' from specifically structured data using linear regression or similar methods, and applying this 'knowledge' to new, unknown inputs.
In its simplest form, or even using a neural network, this only requires two different operations on numbers (or even better, vectors): addition and multiplication.
Using an \gls{he} scheme such as the ones mentioned above and described in \autoref{chap:homomorphic-encryption}, both are given and PPML (Privacy-Preserving Machine Learning) applications are born!

Considering the implications of mass surveilance, the importance of privacy-preserving/enhancing technologies should not be forgotten.

The present thesis not only focusses on theoretical remarks but also includes a publicly available implementation of an \gls{he} classification server written in C++ and a compact graphical user interface to interact with.
The following aims to introduce most of the necessary theory to understand the homomorphic encryption schemes used in practice today, as well as the simple machine learning approaches involved in securely classifying images as a service.

\begin{figure}[H]
  \centering
  \includegraphics[width=\linewidth]{figures/frontend.pdf}
  \caption{The user interface of the demonstrator.}
\end{figure}
