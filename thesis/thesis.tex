\documentclass[11pt,
  oneside,openany,    % for one-sided printing and screens
  %draft              % enable to compile faster and test layout
]{scrreprt}

% Settings
\usepackage[ngerman,english]{babel}
\usepackage[utf8]{inputenc}
\usepackage{geometry}
\usepackage{microtype}
\usepackage[all]{nowidow}
\usepackage[bsc,             % for Bachelor's Thesis titlepage
            %project,        % for Master's Project titlepage
            claim,iaik]{iaikthesis}
\addtokomafont{disposition}{\rmfamily}

% <INSERT YOUR CUSTOM MACROS AND PACKAGES HERE>

% Useful packages for complex content:
\usepackage{amsmath,amsfonts,amssymb} % typesetting math
%\usepackage{tikz}                    % typesetting diagrams, figures, ...
%\usepackage{siunitx}                 % typesetting SI-units and formatted numbers
%\usepackage{listings}                % typesetting source code
%\usepackage{booktabs,multirow}       % utils for complex/beautiful tables
%\usepackage{subcaption}              % placing multiple subfigures in a figure

% Useful utils:
\usepackage{blindtext}                % insert dummy text for this template
%\usepackage{todonotes}               % add ToDo markers (\todo{...})
%\usepackage{xspace}                  % define macros that don't eat the following space (add \xspace at the end)
%\usepackage[section]{placeins}       % prevent figures from floating into the wrong section (\FloatBarrier)

% Bibliography, referencing, and indexing
\usepackage{csquotes}                 % typesetting \enquote{text in quotes} correctly
\usepackage[backend=biber,
            style=alphabetic,
            maxbibnames=20]{biblatex} % to generate the bibliography
\addbibresource{thesis.bib}           % name of the bib-file

\usepackage[hidelinks]{hyperref}      % clickable links (but hide color frames around links)
%\usepackage{cleveref}                % named references (\Cref{chap:introduction}, ...)

\usepackage[toc,acronym,style=long3col]{glossaries} % List of acronyms and symbols  (optional)
\makenoidxglossaries
%
\newacronym{aes}{AES}{Advanced Encryption Standard}
\newglossaryentry{xor}{name={\ensuremath{\oplus}}, sort=xor,
                       description={exclusive-or (\textsc{Xor})}}
% <INSERT YOUR GLOSSARY ENTRIES HERE (or \input{} a dedicated file) >



\begin{document}

%--- INSERT INFORMATION FOR TITLEPAGE ------------------------------------------

% Your name + previous academic degrees:
\thesisauthor{Firstname Lastname}

% Title of your thesis:
\thesistitle{Title and Subtitle\\of the Thesis}

% Date of completion:
\thesisdate{Month Year}

% Supervisor:
\supervisortitle{Supervisors} % or Supervisor
\supervisor{%
  Firstname Lastname of first supervisor\\
  Firstname Lastname of second supervisor (if any)
  \smallskip

  Institute of Applied Information Processing and Communications\\
  Graz University of Technology
}

% Name of your degree programme according to your curriculum
\curriculum{CurriculumName}
%\curriculum{Information and Computer Engineering}
%\curriculum{Computer Science}
%\curriculum{Software Engineering and Management}



%--- FRONT MATTER --------------------------------------------------------------

\printthesistitle

\chapter*{Abstract}

Abstract of your thesis (at most one page)

\Blindtext[2]

\paragraph{Keywords:}
Some keywords...

%\cleardoublepage
%\tableofcontents  % optional



%--- MAIN CONTENT --------------------------------------------------------------

\chapter{Introduction}
\label{chap:introduction}

\Blindtext[3]
% You may want to keep your chapters in separate files ('chapterfilename.tex')
% and include them here using either
%\input{chapterfilename.tex}   or   \include{chapterfilename}


\chapter{Background}
\label{chap:background}

In this chapter, we provide some usage examples for
glossaries and acronym lists with \texttt{glossaries} (Section \ref{sec:gloss}),
bibliography and citations with \texttt{biblatex} (Section \ref{sec:bib}), and more.

\section{Notation and Acronyms}
\label{sec:gloss}

Symbols and acronyms are defined in the preamble, after loading the \texttt{glossaries} package, and used as follows.

In this chapter, we introduce the necessary background on the \gls{aes}.
We denote binary exclusive-or by \gls{xor}.

\section{Citations}
\label{sec:bib}

This is an example of how to specify and cite
a book \cite{AESbook},
a journal article \cite{bstjShannon49},
a conference article \cite{spKocherHFGGHHLM019},
and an informal report \cite{iacrSchneierFKR15}.
We can also add the authors' names to the citation:
\Gls{aes} is a block cipher defined by \textcite{AESbook}.


\chapter{Conclusion}
\label{chap:conclusion}

\Blindtext[3]



%--- INDEX and BIBLIOGRAPHY ----------------------------------------------------

%% Print List of Acronyms and Symbols  (optional)
%\printnoidxglossary[title={Notation}]
%\printnoidxglossary[type=acronym]

% Print bibliography and include it in the table of contents:
\printbibliography[heading=bibintoc]

\end{document}
