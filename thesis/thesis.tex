\documentclass[11pt,
  oneside,openany,    % for one-sided printing and screens
  %draft              % enable to compile faster and test layout
]{scrreprt}

% Settings
\usepackage[ngerman,english]{babel}
\usepackage[utf8]{inputenc}
\usepackage{geometry}
\usepackage{microtype}
\usepackage[all]{nowidow}
\usepackage[bsc,claim,iaik]{iaikthesis}
\addtokomafont{disposition}{\rmfamily}
\usepackage{amsmath,amsfonts,amssymb}
\usepackage{blindtext}
\usepackage{csquotes}
\usepackage[backend=biber,
            style=alphabetic,
            maxbibnames=20]{biblatex}
\addbibresource{thesis.bib}
\usepackage[hidelinks]{hyperref}
%\usepackage{cleveref}                % named references (\Cref{chap:introduction}, ...)
\usepackage[toc,acronym,style=long3col]{glossaries} % List of acronyms and symbols  (optional)

\thesistitle{Classification as a Service}
\thesisauthor{Peter Julius Waldert}
\thesisdate{Month 2020}
\supervisortitle{Supervisors}
\supervisor{
	Dipl.-Ing. Daniel Kales\\
	Dipl.-Ing. Roman Walch\\
	\smallskip
	Institute of Applied Information Processing and Communications\\
	Graz University of Technology
}
\curriculum{\textit{Physics} and \textit{Information and Computer Engineering}}

\title{\@thesistitle}
\author{\@thesisauthor}

\makenoidxglossaries
\newacronym{aes}{AES}{Advanced Encryption Standard}
\newglossaryentry{xor}{name={\ensuremath{\oplus}}, sort=xor, description={exclusive-or (\textsc{Xor})}}


\begin{document}
	\printthesistitle
	
	\chapter*{Abstract}
	Abstract of your thesis (at most one page)
	
	\Blindtext[2]
	
	\paragraph{Keywords:}
	Some keywords...

	\tableofcontents
	
	\chapter{Introduction}
	\label{chap:introduction}
	\Blindtext[3]
	
	\chapter{Background}
	\label{chap:background}
	In this chapter, we provide some usage examples for
	glossaries and acronym lists with \texttt{glossaries} (Section \ref{sec:gloss}),
	bibliography and citations with \texttt{biblatex} (Section \ref{sec:bib}), and more.
	
	\section{Notation and Acronyms}
	\label{sec:gloss}
	Symbols and acronyms are defined in the preamble, after loading the \texttt{glossaries} package, and used as follows.
	
	In this chapter, we introduce the necessary background on the \gls{aes}.
	We denote binary exclusive-or by \gls{xor}.
	
	\section{Citations}
	\label{sec:bib}
	This is an example of how to specify and cite
	a book \cite{AESbook},
	a journal article \cite{bstjShannon49},
	a conference article \cite{spKocherHFGGHHLM019},
	and an informal report \cite{iacrSchneierFKR15}.
	We can also add the authors' names to the citation:
	\Gls{aes} is a block cipher defined by \textcite{AESbook}.
	
	
	\chapter{Conclusion}
	\label{chap:conclusion}
	\Blindtext[3]
	
	\printnoidxglossary[title={Notation}]  % Symbols / Notation
	\printnoidxglossary[type=acronym]  % Acronyms
	
	\printbibliography[heading=bibintoc]
\end{document}
