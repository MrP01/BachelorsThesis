\usetikzlibrary{calc}
\begin{figure}
  \centering
  \begin{tikzpicture}
    \coordinate (Origin)   at (0,0);
    \coordinate (XAxisMin) at (-3,0);
    \coordinate (XAxisMax) at (5,0);
    \coordinate (YAxisMin) at (0,-2);
    \coordinate (YAxisMax) at (0,5);
    \draw [thin, gray,-latex] (XAxisMin) -- (XAxisMax);
    \draw [thin, gray,-latex] (YAxisMin) -- (YAxisMax);

    \clip (-3,-2) rectangle (6.5cm, 6.5cm);
    \pgftransformcm{1}{0.3}{0.7}{1}{\pgfpoint{0cm}{0cm}}
    \coordinate (Bone) at (0,2);
    \coordinate (Btwo) at (2,-2);
    \draw[style=help lines,dashed] (-14,-14) grid[step=2cm] (14,14);
    \foreach \x in {-7,-6,...,7}{
        \foreach \y in {-7,-6,...,7}{
            \node[draw,circle,inner sep=2pt,fill] at (2*\x,2*\y) {};
          }
      }
    \draw [ultra thick,-latex,themecolor] (Origin)
    -- (Bone) node [above left] {$\vec{b}_1$};
    \draw [ultra thick,-latex,themecolor] (Origin)
    -- (Btwo) node [below right] {$\vec{b}_2$};
  \end{tikzpicture}
  \caption{Illustration of a standard lattice $\lat$ over the integers $\Z$
    with two basis vectors $\vec{b}_1$ and $\vec{b}_2$, c.f. \autoref{def:lattice}.
    The shortest vector problem in this case is solved by $\vec{x} = 0 \vec{b}_1 \pm 1 \vec{b}_2$.}
  \label{fig:lattice}
\end{figure}
