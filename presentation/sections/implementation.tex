\section{Implementation Goal and Methods}
\begin{frame}{Goal: Classify MNIST Images of Handwritten Digits}
  \begin{itemize}
    \item Two main types of \gls{ml}: Supervised and Unsupervised Learning
    \item Popular dataset: \gls{mnist}. Encode as vector of 784 entries.
  \end{itemize}

  \begin{figure}[H]
    \centering
    \scalebox{0.8}{\inputtikz{figures/mnist}}
    \caption[Sample images of the MNIST dataset]{
      Sample images of the MNIST database of handwritten digits \parencite{mnist-original}.
      The dataset contains 70,000 images of $28 \times 28$ greyscale pixels valued from 0 to 255 as well as associated labels (as required for supervised learning).
    }
    \label{fig:mnist}
  \end{figure}
\end{frame}

\begin{frame}{Feedforward Neural Networks}
  \begin{figure}[H]
    \centering
    \scalebox{0.9}{\inputtikz{figures/neural-network}}
    \caption[Neural Network illustration resembling the one used in our demonstrator]{
      A simple neural network resembling the structure we use in our demonstrator with $\vec{h} = \cryptop{relu}(M_1 \vec{x} + \vec{b_1})$ and the output $\vec{y} = \cryptop{softmax}(M_2 \vec{h} + \vec{b_2})$.
    }
    \label{fig:neural-network}
  \end{figure}
\end{frame}

\newcommand{\matmulscale}{0.75}
\newcommand{\matmulhoffset}{-3cm}
\begin{frame}{Matrix Multiplication: The Na\"ive Method}
  \begin{figure}[H]
    \centering
    \hspace{\matmulhoffset}
    \scalebox{\matmulscale}{\inputtikz{figures/generated/matmul-naive}}
    \caption[Naïve matrix multiplication method]{The naïve method to multiply a matrix $M \in \R^{s \times t}$ with a vector $\vec{x} \in \R^t$ (adapted from \cite{2018-gazelle}).}
    \label{fig:naive-method}
  \end{figure}

  $$\{M \vec{x}\}_i = \sum_{j=1}^{t} M_{ij} x_j \,.$$
\end{frame}

\begin{frame}{Matrix Multiplication: The Diagonal Method}
  \begin{figure}[H]
    \centering
    \hspace{\matmulhoffset}
    \scalebox{\matmulscale}{\inputtikz{figures/generated/matmul-diagonal}}
    \caption[Diagonal matrix multiplication method]{The diagonal method to multiply a square matrix with a vector (adapted from \cite{2018-gazelle}).}
    \label{fig:diagonal-method}
  \end{figure}

  $$M \vec{x} = \sum_{j=0}^{t-1} \diag_j(M) \cdot \rot_j(\vec{x}) \,.$$
\end{frame}

\begin{frame}{Matrix Multiplication: The Hybrid Method}
  \begin{figure}[H]
    \centering
    \hspace{\matmulhoffset}
    \scalebox{\matmulscale}{\inputtikz{figures/generated/matmul-hybrid}}
    \caption[Hybrid matrix multiplication method]{The hybrid method to multiply an arbitrarily sized matrix with a vector (adapted from \cite{2018-gazelle}).}
    \label{fig:hybrid-method}
  \end{figure}

  $$M \vec{x} = (y_i)_{i \in \Z/s\Z} \;\text{with}\; \vec{y} = \sum_{k=1}^{t / s} \rot_{ks}\bigg(\sum_{j=1}^s \diag_j(M) \cdot \rot_j(\vec{x})\bigg) \,.$$
\end{frame}

\begin{frame}{Polynomial Evaluation}
  \begin{itemize}
    \item Fourth Method: The \gls{bsgs} Method, which has similar performance as the hybrid method.
    \item In between the dense layers, we need to evaluate the \cryptop{relu} function.
          \begin{itemize}
            \item Approximate it by a series expansion...
                  $$\cryptop{relu\_taylor}(x) = -0.006137 x^3 + 0.090189 x^2 + 0.59579 x + 0.54738 \,.$$
          \end{itemize}
    \item The \cryptop{softmax} activation at the end can be done by the client.
  \end{itemize}
\end{frame}
