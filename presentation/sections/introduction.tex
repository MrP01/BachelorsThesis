\section{Introduction}
\begin{frame}{\gls{ppml}}
  \begin{itemize}
    \item Development of new applications and solutions 'of numerical nature' in different fields
          \begin{itemize}
            \item Example: Health Care with highly sensitive medical data.
            \item For instance, RNA sequences, images of skin, lab data, medical records, etc.
            \item The results are even more volatile: disease indicators.
          \end{itemize}
    \item $\Rightarrow$ Demand for privacy-preserving solutions in \gls{ml} applications.
    \item By the way: Post-Quantum Secure Cryptosystems!
  \end{itemize}
\end{frame}

\begin{frame}{What about Long-Term Security?}
  \begin{columns}
    \begin{column}{0.6\linewidth}
      Quantum Computers affect Cryprography today:
      \begin{itemize}
        \item Problems believed to be \glstext{np}-hard on classical computers can be computed in polynomial time using a quantum computer.
        \item No hardness proof of the integer factorisation or RSA problems exist as of today.
        \item \name{Shor}'s, \name{Grover}'s and other algorithms can 'break' many cryptographic schemes used today.
        \item The existence of a sufficiently powerful quantum computer endangers the security of \glstext{tls}, etc.
      \end{itemize}
    \end{column}
    \begin{column}{0.4\linewidth}
      \vspace{-0.3cm}
      \begin{figure}[H]
        \centering
        \scalebox{0.7}{\inputtikz{figures/wave-function}}
        \caption{Illustration of a wave function $\psi$ as commonly used in quantum mechanics.}
        \label{fig:wave-function}
      \end{figure}
    \end{column}
  \end{columns}
  Our Webservice is (from the point of todays knowledge) still secure in the presence of a quantum computer.
\end{frame}

\begin{frame}{The \gls{rsa} Scheme}
  From the integers $\Z$, define the quotient ring $(\Z/q\Z, +, \cdot)$ for some modulus $q \in \N$.

  With unpadded \glstext{rsa} \parencite{1983-rsa}, $\mathcal{E}: \Z/q\Z \mapsto \Z/q\Z$
  $$\mathcal{E}(m) := m^r \mod q \quad r, q \in \N$$
  applied to the messages $m_1, m_2 \in \Z/q\Z$ respectively, the following holds:
  \begin{align*}
    \mathcal{E}(m_1) \cdot \mathcal{E}(m_2)
     & \equiv (m_1)^r (m_2)^r \mod q            \\
     & \equiv (m_1 m_2)^r \mod q                \\
     & \equiv \mathcal{E}(m_1 \cdot m_2) \mod q
  \end{align*}
  $\Rightarrow$ A Group Homomorphism!
\end{frame}
