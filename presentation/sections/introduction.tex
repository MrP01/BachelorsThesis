\section{Introduction}
\begin{frame}{Privacy for Medical Applications}
  \begin{itemize}
    \item Development of new applications and solutions 'of numerical nature' in health care, but: highly sensitive medical data.
    \item For instance, RNA sequences, images of skin, lab data, medical records, etc.
    \item The results are even more volatile: disease predictions
    \item $\Rightarrow$ Demand for privacy-preserving solutions in \gls{ml} applications.
  \end{itemize}
\end{frame}

\begin{frame}{Post-Quantum Security}
  \begin{figure}[H]
    \centering
    \inputtikz{figures/wave-function}
    \caption{Illustration of a wave function $\psi$ as commonly used in quantum mechanics.}
    \label{fig:wave-function}
  \end{figure}
\end{frame}

\begin{frame}{The \gls{rsa} Scheme}
  From the integers $\Z$, define the quotient ring $(\Z/q\Z, +, \cdot)$ for some modulus $q \in \N$.

  With unpadded \gls{rsa} \parencite{1983-rsa}, some arithmetic can be performed on the ciphertext - looking at the encrypted ciphertext $\mathcal{E}: \Z/q\Z \mapsto \Z/q\Z,\, \mathcal{E}(m) := m^r \mod q$ ($r, q \in \N$) of the message $m_1, m_2 \in \Z/q\Z$ respectively, the following holds:
  \begin{align*}
    \mathcal{E}(m_1) \cdot \mathcal{E}(m_2)
     & \equiv (m_1)^r (m_2)^r \mod q            \\
     & \equiv (m_1 m_2)^r \mod q                \\
     & \equiv \mathcal{E}(m_1 \cdot m_2) \mod q
  \end{align*}
\end{frame}
