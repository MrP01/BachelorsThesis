\section{The CKKS Scheme}
\begin{frame}{Overview of \gls{ckks}}
  \begin{figure}[H]
    \centering
    \scalebox{0.8}{\inputtikz{figures/ckks-schematic}}
    \caption[Schematic overview of the CKKS scheme]{
      Schematic overview of CKKS \parencite{2017-ckks-original}, adapted from \cite{2020-cryptotree}.
      A plain vector $\vec{z} \in \C^{N/2}$ is encoded to $m = \cryptop{CKKS.Encode}(\vec{z})$, encrypted to $\vec{c} = \cryptop{CKKS.Encrypt}(\vec{p}, m)$, decrypted and decoded to a new $\tilde{\vec{z}} = \cryptop{CKKS.Decode}(\cryptop{CKKS.Decrypt}(s, \tilde{\vec{c}}))$.
    }
    \label{fig:ckks-overview}
  \end{figure}
\end{frame}

\begin{frame}{Encoding and Decoding}
  \cryptop{CKKS.} \\
  \begin{tblr}{Q[l,h]Q[l,h,\textwidth - 3.5cm]}
    \cryptop{Encode}$(\vec{z})$ & {For a given input vector $\vec{z}$, output
        $m = (\underline{\sigma}\inv \circ \underline{\rho_\delta}\inv \circ \underline{\pi}\inv)(\vec{z}) = \underline{\sigma}\inv(\lfloor \delta \cdot \underline{\pi}\inv(\vec{z})\rceil_{\underline{\sigma}(R)})$ $\quad\rightarrow m$} \\
    \cryptop{Decode}$(m)$ & {Decode plaintext $m$ as
        $\vec{z} = (\underline{\pi} \circ \underline{\rho_\delta} \circ \underline{\sigma})(m) = (\underline{\pi} \circ \underline{\sigma})(\delta\inv m)$
        $\quad\rightarrow \vec{z}$} \\
  \end{tblr}
  \begin{itemize}
    \item Three transformations: $\underline{\sigma}\inv$, $\underline{\rho_\delta}\inv$ and $\underline{\pi}\inv$.
    \item Key idea: Homomorphic property, they preserve additivity and multiplicativity.
    \item Allows for homomorphic \gls{simd} operations.
  \end{itemize}
\end{frame}

\begin{frame}{Encryption and Decryption}
  \cryptop{CKKS.} \\
  \begin{tblr}{Q[l,h]Q[l,h,\textwidth - 3.5cm]}
    \cryptop{Encrypt}$(\vec{p}, m)$ & {
        Let $(b,a) = \vec{p}$, $u \leftarrow \chi_{enc}$, $\mu_1, \mu_2 \leftarrow \chi_{error}$,
        then the ciphertext is $\vec{c} = u \cdot \vec{p} + (m + \mu_1, \mu_2) = (m + bu + \mu_1, au + \mu_2)$
        $\quad\rightarrow \vec{c}$} \\
    \cryptop{Decrypt}$(s, \vec{c})$ & {
        Decrypt the ciphertext $\vec{c} = (c_0, c_1)$ as $m = \lbrack c_0 + c_1 s\rbrack_{q_L}$
        $\quad\rightarrow m$} \\
  \end{tblr}
  \begin{itemize}
    \item A public-key cryptosystem! Encrypt with $\vec{p}$, decrypt with $s$.
    \item Leaves the attacker with the \gls{rlwe} problem.
    \item Decrypts correctly under certain conditions...
  \end{itemize}
\end{frame}

\begin{frame}{Homomorphic Addition}
  \begin{tblr}{Q[l,h]Q[l,h,\textwidth - 3.5cm]}
    \cryptop{CKKS.Add}$(\vec{c}, \vec{c}')$ & {
        Output $\overline{\vec{c}} = \vec{c} + \vec{c}' = \begin{pmatrix}
            \delta (m + m') + b (u + u') + (\mu_1 + \mu_1') \\
            a (u + u') + (\mu_2 + \mu_2')
          \end{pmatrix}^T$} \\
  \end{tblr}

  Indeed, the ciphertext $\overline{\vec{c}}$ correctly decrypts back to $\overline{m} := m + m'$:
  \begin{align*}
    \cryptop{CKKS.Decrypt}(s, \overline{\vec{c}})
     & = \lfloor \delta\inv [\overline{c_0} + \overline{c_1} s]_t \rceil                                                                                                                                                         \\
     & = \big\lfloor \delta\inv [\delta \overline{m} + b \overline{u} + \overline{\mu_1} + (a \overline{u} + \overline{\mu_2}) s]_t \big\rceil                                                                                   \\
     & = \big\lfloor [(\delta\inv\delta) \overline{m} + \delta\inv b \overline{u} + \delta\inv \overline{\mu_1} + \delta\inv a s \overline{u} + \delta\inv \overline{\mu_2} s]_t \big\rceil                                      \\
     & = \big\lfloor [\overline{m} - \cancel{\delta\inv as \overline{u}} - \delta\inv \tilde{\mu} \overline{u} + \delta\inv \overline{\mu_1} + \cancel{\delta\inv as \overline{u}} + \delta\inv \overline{\mu_2} s]_t \big\rceil \\
     & = \big\lfloor [\overline{m} + \underbrace{\delta\inv (\overline{\mu_1} + \overline{\mu_2} s - \tilde{\mu} \overline{u})}_{:= \epsilon \,, ||\epsilon|| \ll 1}]_t \big\rceil
    \approx \big\lfloor [\overline{m}]_t \big\rceil = \lfloor \overline{m} \rceil \approx \overline{m}
  \end{align*}
\end{frame}
