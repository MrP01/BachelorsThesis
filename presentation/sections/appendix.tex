\appendix
\section{\appendixname}
\begin{frame}{Some more Details...}
  Additional Material omitted in main talk.

  \begin{itemize}
    \item \glsdesc{rlwe}
    \item Encoding and Decoding transformations
    \item The BabyStep-Giantstep method
    \item Proof of Diagonal, Hybrid method
    \item Shor's Algorithm
  \end{itemize}
\end{frame}

\begin{frame}{Polynomial Rings}
  \begin{columns}
    \begin{column}{0.6\linewidth}
      \begin{definition}[Cyclotomic Polynomial]
        Given the $n$\th roots of unity $\{\xi_k\}$, define $\Phi_n \in \Z[X]$ as
        $$\Phi_n(x) := \prod_{\stackrel{k=1}{\xi_k \mathrm{primitive}}}^{n} (x - \xi_k) \,.$$
        It is unique for each given $n \in \N$.
      \end{definition}
    \end{column}
    \begin{column}{0.32\linewidth}
      \begin{figure}
        \scalebox{0.64}{\hspace{-1.5cm}\inputtikz{figures/nth-roots-of-unity}}
        \caption{The 5\th roots of unity}
        \label{fig:nth-roots-of-unity}
      \end{figure}
    \end{column}
  \end{columns}
\end{frame}

\begin{frame}{Some Notation}
  \begin{itemize}
    \item $\Z[X] := \big\{p: \C \mapsto \C \,, p(x) = \sum_{k=0}^\infty a_k x^k, a_k \in \Z \;\forall k \geq 0\big\}$
          \begin{itemize}
            \item Complex-valued Polynomials with integer coefficients.
          \end{itemize}
    \item $\Z_q[X] := (\Z/q\Z)[X] = \Z[X] / q\Z[X]$
    \item $\Z_q[X] / \Phi_M(X) \underbrace{ = \Z_q[X]/(X^N+1)}_{\text{for } M=2N, \text{ powers of 2}}$ using the $M$\th cyclotomic polynomial $\Phi_M(X)$.
          \begin{itemize}
            \item Its elements are polynomials of degree $(N-1)$\footnote{For general $M$, degree $\varphi(M) - 1$} with integer coefficients mod $q$.
          \end{itemize}
  \end{itemize}
\end{frame}

\begin{frame}{The \gls{rlwe} Problem}
  \begin{definition}[RLWE-Distribution $B_{\vec{s}, \chi_{error}}$]
    Given a quotient ring $(R/qR, +, \cdot)$, choose a secret $s \in R/qR$.
    Sampling from the RLWE distribution $B_{s, \chi_{error}}$:
    \begin{itemize}
      \item Uniformly randomly draw an element $a \in R/qR$
      \item Sample 'noise' $\mu \in R/qR$ from $\chi_{error}$.
      \item Set $b = s \cdot a + \mu$, with $\cdot$ denoting the ring multiplication operation.
      \item Output the pair $(a, b) \in R/qR \times R/qR$.
    \end{itemize}
  \end{definition}

  Proven equivalent to LWE.

  Use Search-RLWE to construct a cryptosystem...
  Idea: Attacker needs to solve \gls{lwe} given the public key to recover the secret $s$.
\end{frame}

\begin{frame}{Encoding and Decoding}
  \cryptop{CKKS.} \\
  \begin{tblr}{Q[l,h]Q[l,h,\textwidth - 3.5cm]}
    \cryptop{Encode}$(\vec{z})$ & {For a given input vector $\vec{z}$, output
        $m = (\underline{\sigma}\inv \circ \underline{\rho_\delta}\inv \circ \underline{\pi}\inv)(\vec{z}) = \underline{\sigma}\inv(\lfloor \delta \cdot \underline{\pi}\inv(\vec{z})\rceil_{\underline{\sigma}(R)})$ $\quad\rightarrow m$} \\
    \cryptop{Decode}$(m)$ & {Decode plaintext $m$ as
        $\vec{z} = (\underline{\pi} \circ \underline{\rho_\delta} \circ \underline{\sigma})(m) = (\underline{\pi} \circ \underline{\sigma})(\delta\inv m)$
        $\quad\rightarrow \vec{z}$} \\
  \end{tblr}
  \begin{itemize}
    \item Three transformations: $\underline{\sigma}\inv$, $\underline{\rho_\delta}\inv$ and $\underline{\pi}\inv$.
    \item Key idea: Homomorphic property, they preserve additivity and multiplicativity.
    \item Allows for homomorphic \gls{simd} operations.
  \end{itemize}
\end{frame}

\begin{frame}[c]
  \begin{definition}[Canonical Embedding $\underline{\sigma}$]
    For a real-valued polynomial $p \in \mathcal{S}$, define the canonical embedding of $\mathcal{S}$ in $\C^N$ as a mapping $\underline{\sigma}: \mathcal{S} \mapsto \C^N$ with $$\underline{\sigma}(p) := \big(p(e^{-2\pi i j / N})\big)_{j \in \Z_d^*}$$ with $\Z_d^* := \{x \in \Z / d\Z \,|\, \gcd(x, d) = 1\}$ the set of all integers smaller than $d$ that do not share a factor $> 1$ with $d$.
    The image of $\underline{\sigma}$ given a set of inputs $R$ shall be denoted as $\underline{\sigma}(R) \subseteq \C^N$.
    Let the inverse of $\underline{\sigma}$ be denoted by $\underline{\sigma}\inv: \C^N \mapsto \mathcal{S}$.
  \end{definition}
\end{frame}

\begin{frame}[c]
  \begin{definition}[Discretisation to an element of $\underline{\sigma}(R)$]
    Using one of several round-off algorithms (cf. \cite{2013-rlwe-toolkit}), given an element of $\mathbb{H}$, define a rounding operation $\underline{\rho}\inv: \mathbb{H} \mapsto \underline{\sigma}(R)$ that maps an $\vec{h} \in \mathbb{H}$ to its closest element in $\underline{\sigma}(R) \subset \mathbb{H}$, also denoted as
    $$\underline{\rho}\inv(\vec{h}) := \lfloor \vec{h} \rceil_{\underline{\sigma}(R)}\,.$$
    Further let $\underline{\rho_\delta}\inv(\vec{h}) = \lfloor \delta \cdot \vec{h} \rceil_{\underline{\sigma}(R)}$ denote the same rounding operation but with prior scaling by a scalar factor $\delta$.
    Note that $\underline{\rho}$ is given directly as the identity operation because all elements of its domain are already elements of its image. Similarly, $\underline{\rho_\delta}(\vec{y}) = \delta\inv \cdot \vec{y}$.
  \end{definition}
\end{frame}

\begin{frame}[c]
  \begin{definition}[Natural Projection $\underline{\pi}$]
    Let $T$ be a multiplicative subgroup of $\Z_d^*$ with $\Z_d^* / T = \{\pm 1\} = \{1T, -1T\}$, then the natural projection $\underline{\pi}: \mathbb{H} \mapsto \C^{N/2}$ is defined as
    $$\underline{\pi}\big((z_j)_{j \in \Z_M^*}\big) := (z_j)_{j \in T}$$
    Let its inverse be denoted by $\underline{\pi}\inv: \C^{N/2} \mapsto \mathbb{H}$ and consequently defined as
    $$\underline{\pi}\inv\big((z_j)_{j \in T}\big) := \big(\nu(z_j)\big)_{j \in \Z_M^*} \; \mathrm{with} \; \nu(z_j) = \begin{cases}
        z_j            & \text{if } j \in T \\
        \overline{z_j} & \text{otherwise}
      \end{cases}$$
  \end{definition}
\end{frame}

\begin{frame}{Goal: Classify MNIST Images of Handwritten Digits}
  \begin{itemize}
    \item Two major types of \gls{ml}: Supervised and Unsupervised Learning
    \item Popular dataset: \gls{mnist}. Encode as vector of 784 entries.
  \end{itemize}

  \begin{figure}[H]
    \centering
    \scalebox{0.8}{\inputtikz{figures/mnist}}
    \caption[Sample images of the MNIST dataset]{
      Sample images of the MNIST database of handwritten digits \parencite{mnist-original}.
      The dataset contains 70,000 images of $28 \times 28$ greyscale pixels valued from 0 to 255 as well as associated labels (as required for supervised learning).
    }
    \label{fig:mnist}
  \end{figure}
\end{frame}

\begin{frame}[c]
  \begin{theorem}[Babystep-Giantstep Method]
    Given a matrix $M \in \R^{t \times t}$ and a vector $\vec{x} \in \R^t$, with $t = t_1 \cdot t_2$ split into two \gls{bsgs} parameters $t_1, t_2 \in \N$ and
    $$\diag'_p(M) = \rot_{-\lfloor p/t_1 \rfloor \cdot t_1}(\diag_p(M))\,,$$
    one can express a matrix-vector multiplication as follows:
    \begin{equation*}
      M \vec{x} = \sum_{k=0}^{t_2-1} \rot_{(kt_1)} \bigg(
      \sum_{j=0}^{t_1-1} \diag'_{(kt_1+j)}(M) \cdot \rot_j(\vec{x})
      \bigg)
    \end{equation*}
    where $\cdot$ denotes an element-wise multiplication of two vectors.
  \end{theorem}
\end{frame}

\begin{frame}[c]
  \begin{proof}[Proof (Diagonal Method)]
    For all indices $i \in \Z/t\Z$,
    $$\bigg\{\sum_{j=0}^{t-1} \diag_j(M) \cdot \rot_j(\vec{x})\bigg\}_i
      = \sum_{j=0}^{t-1} M_{i,(i+j)} x_{i+j}
      \overset{[k=i+j]}{=} \sum_{k=i}^{t+i-1} M_{ik} x_k
      = \sum_{k=0}^{t-1} M_{ik} x_k
      = \{M \vec{x}\}_i \,.$$
  \end{proof}
\end{frame}

\begin{frame}[c]
  \begin{proof}[Proof (Hybrid Method)]
    For all indices $i \in \Z/s\Z$,
    $$\{\vec{y}\}_i = \bigg\{\sum_{k=1}^{t / s} \rot_{ks}\bigg(\sum_{j=1}^s \diag_j(M) \cdot \rot_j(\vec{x})\bigg)\bigg\}_i = \sum_{k=1}^{t / s} \sum_{j=1}^{s} M_{i,(i+j) + ks} x_{(i+j) + ks} \,,$$
    substituting $l = i+j+ks$ and condensing the nested sums into one single summation expression since $\sum_{k=1}^{t / s} \sum_{j=1}^{s} f(j+ks) = \sum_{l=1}^{t} f(l)$, we obtain
    $$y_i = \sum_{l=1+i}^{t+i} M_{il} x_l = \sum_{l=1}^{t} M_{il} x_l = \{M \vec{x}\}_i \,.$$
  \end{proof}
\end{frame}

\begin{frame}{The \gls{rsa} Scheme}
  From the integers $\Z$, define the quotient ring $(\Z/q\Z, +, \cdot)$ for some modulus $q \in \N$.

  With unpadded \glstext{rsa} \parencite{1983-rsa}, $\mathcal{E}: \Z/q\Z \mapsto \Z/q\Z$
  $$\mathcal{E}(m) := m^r \mod q \quad r, q \in \N$$
  applied to the messages $m_1, m_2 \in \Z/q\Z$ respectively, the following holds:
  \begin{align*}
    \mathcal{E}(m_1) \cdot \mathcal{E}(m_2)
     & \equiv (m_1)^r (m_2)^r \mod q            \\
     & \equiv (m_1 m_2)^r \mod q                \\
     & \equiv \mathcal{E}(m_1 \cdot m_2) \mod q
  \end{align*}
  $\Rightarrow$ A Group Homomorphism!
\end{frame}

\begin{frame}[allowframebreaks]{\name{Shor}'s Algorithm}
  Peter \name{Shor}'s algorithm was published in 1994 \parencite{1997-shors-algorithm} and will be outlined here shortly as it is a core element to security considerations of modern cryptosystems.
  The core structure of the algorithm is
  \begin{enumerate}
    \item guessing some $g \in \N$ that we hope shares a factor with a large $N = p \cdot q$ ($p, q, N \in \N$),
    \item improving that guess $g$ by a quantum subroutine and
    \item applying \name{Euclid}'s algorithm to find $p$ and $q$ the factors of $N$.
  \end{enumerate}

  The core factorisation idea is the following, not specific to quantum computation: We know that for a pair $g, N \in \N$, we can always find some $r \in \N$ such that
  $$g^r = m N + 1,\, m \in \N\,,$$
  we are looking for a $g^r$ that is exactly one more than a multiple of $N$.
  Rearranging,
  $$g^r - 1 = m N \;\Longleftrightarrow\; (g^\frac{r}{2} + 1)(g^\frac{r}{2} - 1) = m N$$
  we have found two factors $g^\frac{r}{2} + 1$ and $g^\frac{r}{2} - 1$ (for even $r$) that share a common factor with $N$ and apply Euclid's algorithm to get $p$ and $q$.

  Thereby, we instruct the quantum computer to raise our guess $g$ by all possible powers $\in \N$ up to some boundary in order to obtain
  $$\ket{1, g^1} + \ket{2, g^2} + \ket{3, g^3}, ...$$
  which we then take modulo $N$, resulting in a superposition of remainders
  $$\ket{1, [g^1]_N} + \ket{2, [g^2]_N} + \ket{3, [g^3]_N} + ... \,.$$

  Here is where \name{Shor}'s key idea came in:
  The remainders in the above superposition expose repetitions at a period of exactly $r$ (which, by our definition fulfils $g^r \equiv 1 \mod N$)
  $$g^x \equiv g^{x + r} \equiv g^{x + 2r} \equiv ... \equiv g^{x + ar} \mod N$$
  the remainders are periodic with frequency $\frac{1}{r}$.

  The above can be quickly derived from $g^r = mN + 1$, therefore $$g^{x+r} = g^x g^r = (\tilde{m} N + [g^x]_N) (m N + 1) = (m \tilde{m} N + [g^x]_N m + \tilde{m}) N + [g^x]_N$$ is indeed congruent to $g^x \mod N$.

  From the output of
  $$\cryptop{QFT}\big(\ket{1, [g^1]_N} + \ket{2, [g^2]_N} + \ket{3, [g^3]_N} + ...\big)$$
  we obtain the dominant frequency $\frac{1}{r}$ yielding us our desired improved guess \parencite{1997-shors-algorithm}.
\end{frame}
