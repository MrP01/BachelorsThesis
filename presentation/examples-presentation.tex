\documentclass[table,aspectratio=43]{beamer}
% For a 16:9 example, set aspectratio=169

\usetheme[institute,webfont]{tugraz2018}
\institutelogo{beamerthemetugraz/institute/IAIK}

\usepackage[utf8]{inputenc}
\usepackage[english]{babel}

%%%%%%%%%%%%%%%%%%%%%%%%%%%%%%%%%%%%%%%%%%%%%%%%%%%%%%%%%%%%%%%%%%%%%%%%%%%%
% --- Optional packages for scientific content ---
\usepackage{listings}    % Source code listings
\usepackage{pgfplots}    % Plots and diagrams
\usepackage{tabu}        % Coloured tables -- requires [table] option for beamer
\usepackage{fontawesome} % Useful icons (\faName)
\usepackage[style=alphabetic,backend=biber]{biblatex} % Bibliography
\addbibresource{\jobname.bib}                         % Bibliography
\usepackage{filecontents}

%%%%%%%%%%%%%%%%%%%%%%%%%%%%%%%%%%%%%%%%%%%%%%%%%%%%%%%%%%%%%%%%%%%%%%%%%%%%
% --- Presentation metadata ---
\title[]{Presentation Examples}
\author[Jane Doe]{\textbf{Jane Doe} \and Mary Sue \and Gary Stu}
\date{SomeConf 2018}
\institute{IAIK}
\instituteurl{www.tugraz.at}


\begin{document}

%%%%%%%%%%%%%%%%%%%%%%%%%%%%%%%%%%%%%%%%%%%%%%%%%%%%%%%%%%%%%%%%%%%%%%%%%%%%
\begin{frame}[plain]
  \maketitle
\end{frame}

\begin{frame}
  \tableofcontents
\end{frame}

%%%%%%%%%%%%%%%%%%%%%%%%%%%%%%%%%%%%%%%%%%%%%%%%%%%%%%%%%%%%%%%%%%%%%%%%%%%%
\section{Theme variants}

%%%%%%%%%%%%%%%%%%%%%%%%%%%%%%%%%%%%%%%%%%%%%%%%%%%%%%%%%%%%%%%%%%%%%%%%%%%%
\switchtostandard
\title[]{Standard Titlepage\\Minimal Titlepage}
\begin{frame}[plain]
  \maketitle
\end{frame}

\switchtoinstitute
\title[]{Institute Titlepage}
\begin{frame}[plain]
  \maketitle
\end{frame}

%%%%%%%%%%%%%%%%%%%%%%%%%%%%%%%%%%%%%%%%%%%%%%%%%%%%%%%%%%%%%%%%%%%%%%%%%%%%
\switchtostandard
\begin{frame}{Standard style}
\end{frame}
\switchtoinstitute
\begin{frame}{Institute style}
\end{frame}
\switchtominimal
\begin{frame}{Minimal style}
\end{frame}
\switchtostandard

%%%%%%%%%%%%%%%%%%%%%%%%%%%%%%%%%%%%%%%%%%%%%%%%%%%%%%%%%%%%%%%%%%%%%%%%%%%%
\section{Text and colors}

%%%%%%%%%%%%%%%%%%%%%%%%%%%%%%%%%%%%%%%%%%%%%%%%%%%%%%%%%%%%%%%%%%%%%%%%%%%%
\begin{frame}{Color palette}
  \begin{columns}[t]
    \column{.25\textwidth}
      \begin{beamercolorbox}[colsep*=4pt]{tug}tug = main\end{beamercolorbox}
      \begin{beamercolorbox}[colsep*=4pt]{fore}fore\end{beamercolorbox}
      \begin{beamercolorbox}[colsep*=4pt]{back}back\end{beamercolorbox}
      \begin{beamercolorbox}[colsep*=4pt]{foot}foot\end{beamercolorbox}
      \begin{beamercolorbox}[colsep*=4pt]{dark}dark\end{beamercolorbox}
      \begin{beamercolorbox}[colsep*=4pt]{lite}lite\end{beamercolorbox}
      \begin{beamercolorbox}[colsep*=4pt]{head}head\end{beamercolorbox}
      \begin{beamercolorbox}[colsep*=4pt]{body}body\end{beamercolorbox}
    \column{.25\textwidth}
      \begin{beamercolorbox}[colsep*=4pt]{arch}arch\end{beamercolorbox}
      \begin{beamercolorbox}[colsep*=4pt]{bauw}bauw (bau)\end{beamercolorbox}
      \begin{beamercolorbox}[colsep*=4pt]{etec}etec (etit)\end{beamercolorbox}
      \begin{beamercolorbox}[colsep*=4pt]{mach}mach (mbww)\!\!\end{beamercolorbox}
      \begin{beamercolorbox}[colsep*=4pt]{chem}chem (tcvp)\end{beamercolorbox}
      \begin{beamercolorbox}[colsep*=4pt]{info}info (infbio)\end{beamercolorbox}
      \begin{beamercolorbox}[colsep*=4pt]{math}math (mpug)\end{beamercolorbox}
    \column{.25\textwidth}
      \begin{beamercolorbox}[colsep*=4pt]{colA}colA\end{beamercolorbox}
      \begin{beamercolorbox}[colsep*=4pt]{colB}colB\end{beamercolorbox}
      \begin{beamercolorbox}[colsep*=4pt]{colC}colC\end{beamercolorbox}
      \begin{beamercolorbox}[colsep*=4pt]{colD}colD\end{beamercolorbox}
      \begin{beamercolorbox}[colsep*=4pt]{colE}colE\end{beamercolorbox}
      \begin{beamercolorbox}[colsep*=4pt]{colF}colF\end{beamercolorbox}
      Also:
      \textcolor{tugblue}{tugblue},
      \textcolor{tugred}{tugred}, \dots
  \end{columns}
\end{frame}

%%%%%%%%%%%%%%%%%%%%%%%%%%%%%%%%%%%%%%%%%%%%%%%%%%%%%%%%%%%%%%%%%%%%%%%%%%%%
\begin{frame}{Structured text}

  \textbf{bold}, \textit{italics}, \textrm{roman}, \texttt{typewriter}, \underline{underlined},
  \bigskip

  \emph{Emphasis} (itshape)
  \bigskip

  \alert{Alert},  $a + \alert{b} = c$ (tug)
  \bigskip

  \structure{Structure},  $a + \structure{b} = c$ (main)
  \bigskip
  
  \textcolor{colD}{Color},  $a + {\color{colD} b} = c$
  \bigskip

  \highlight{Highlight (main)}
  \highlight[lite]{Highlight (lite)}
  \bigskip

  \url{www.tugraz.at} (urlA, ttfamily)
\end{frame}

%%%%%%%%%%%%%%%%%%%%%%%%%%%%%%%%%%%%%%%%%%%%%%%%%%%%%%%%%%%%%%%%%%%%%%%%%%%%
\begin{frame}[fragile]{Font sizes}
  \thefontsize[TINY]\TINY
  \thefontsize[Tiny]\Tiny
  \thefontsize[tiny]\tiny
  \thefontsize[scriptsize]\scriptsize
  \thefontsize[footnotesize]\footnotesize
  \thefontsize[small]\small
  \thefontsize[normalsize]\normalsize
  \thefontsize[large]\large
  \thefontsize[Large]\Large
  \thefontsize[LARGE]\LARGE
  \thefontsize[huge]\huge
  \thefontsize[Huge]\Huge
\end{frame}

%%%%%%%%%%%%%%%%%%%%%%%%%%%%%%%%%%%%%%%%%%%%%%%%%%%%%%%%%%%%%%%%%%%%%%%%%%%%
\section{Structuring the frame}

%%%%%%%%%%%%%%%%%%%%%%%%%%%%%%%%%%%%%%%%%%%%%%%%%%%%%%%%%%%%%%%%%%%%%%%%%%%%
\begin{frame}{Lists -- Basic}
  \texttt{itemize}:
  \begin{itemize}
    \item Item level 1
      \begin{itemize}
        \item Item level 2
          \begin{itemize}
            \item Item level 3
          \end{itemize}
      \end{itemize}
  \end{itemize}

  \texttt{enumerate}:
  \begin{enumerate}
    \item Item level 1
      \begin{enumerate}
        \item Item level 2
          \begin{enumerate}
            \item Item level 3
          \end{enumerate}
      \end{enumerate}
  \end{enumerate}
\end{frame}

%%%%%%%%%%%%%%%%%%%%%%%%%%%%%%%%%%%%%%%%%%%%%%%%%%%%%%%%%%%%%%%%%%%%%%%%%%%%
\begin{frame}{Lists -- Variations}
  \texttt{tugitemize} with tighter spacing:
  \begin{tugitemize}
    \item Item level 1
      \begin{tugitemize}
        \item Item level 2
          \begin{tugitemize}
            \item Item level 3
          \end{tugitemize}
      \end{tugitemize}
  \end{tugitemize}

  \texttt{boxenumerate}:
  \begin{boxenumerate}
    \item Item level 1
      \begin{boxenumerate}
        \item Item level 2
          \begin{boxenumerate}
            \item Item level 3
          \end{boxenumerate}
      \end{boxenumerate}
  \end{boxenumerate}
\end{frame}

%%%%%%%%%%%%%%%%%%%%%%%%%%%%%%%%%%%%%%%%%%%%%%%%%%%%%%%%%%%%%%%%%%%%%%%%%%%%
\newcommand{\checkyes}{\textcolor{tuggreen}{\faCheckSquareO}}
\newcommand{\checkno}{\textcolor{black}{\faSquareO\,}}
% Check out https://ctan.org/pkg/fontawesome
\begin{frame}[fragile]{Lists -- Custom Icons with package \texttt{fontawesome}}
  Checklist:
  \begin{itemize}
    \item[\checkyes] Item 1
    \item[\checkyes] Item 2
    \item[\checkno] Item 3
  \end{itemize}
  Advantages and Disadvantages:
  \begin{itemize}
    \item[\textcolor{tuggreen}{\faPlusCircle}] Advantage
    \item[\textcolor{tugred}{\faMinusCircle}]  Disadvantage
    \item[\textcolor{tugblue}{\faArrowCircleRight}] Conclusion
  \end{itemize}
\end{frame}

%%%%%%%%%%%%%%%%%%%%%%%%%%%%%%%%%%%%%%%%%%%%%%%%%%%%%%%%%%%%%%%%%%%%%%%%%%%%
\begin{frame}{Blocks}
  %
  \begin{block}{Block}
    dark+lite
  \end{block}
  %
  \begin{alertblock}{Alert Block}
    tug+lite
  \end{alertblock}
  %
  \begin{exampleblock}{Example Block}
    colE+lite
  \end{exampleblock}
\end{frame}

%%%%%%%%%%%%%%%%%%%%%%%%%%%%%%%%%%%%%%%%%%%%%%%%%%%%%%%%%%%%%%%%%%%%%%%%%%%%
\begin{frame}{Blocks -- Math}
  \begin{theorem}[Theorem or Definition name]
    Content
  \end{theorem}
  %
  %\begin{definition}[Definition name]
  %  Content
  %\end{definition}
  %
  \begin{example}[Example name]
    Content
  \end{example}
  %
  \begin{proof}[Proof name]
    Content
  \end{proof}
\end{frame}

%%%%%%%%%%%%%%%%%%%%%%%%%%%%%%%%%%%%%%%%%%%%%%%%%%%%%%%%%%%%%%%%%%%%%%%%%%%%
\begin{frame}{Side-by-side content}
	\begin{columns}[onlytextwidth]
		\begin{column}{0.5\textwidth}
			\begin{itemize}
				\item Lorem ipsum dolor sit amet, consectetur 
				\item adipisicing elit, sed do eiusmod tempor 
				\item incididunt ut labore et dolore magna aliqua. 
			\end{itemize}
		\end{column}
		\begin{column}{0.5\textwidth}
			\begin{center}
        \includegraphics[width=0.75\textwidth]{figures/photoexample-43}\\
      A figure
			\end{center}
		\end{column}
	\end{columns}
\end{frame}

%%%%%%%%%%%%%%%%%%%%%%%%%%%%%%%%%%%%%%%%%%%%%%%%%%%%%%%%%%%%%%%%%%%%%%%%%%%%
\begin{frame}[plain,c] % 'plain' hides theme elements
  \centering
  \Large Central standout text
\end{frame}

%%%%%%%%%%%%%%%%%%%%%%%%%%%%%%%%%%%%%%%%%%%%%%%%%%%%%%%%%%%%%%%%%%%%%%%%%%%%
\fullscreenfigure{figures/photoexample-43}

%%%%%%%%%%%%%%%%%%%%%%%%%%%%%%%%%%%%%%%%%%%%%%%%%%%%%%%%%%%%%%%%%%%%%%%%%%%%
\section{Structuring the presentation}

%%%%%%%%%%%%%%%%%%%%%%%%%%%%%%%%%%%%%%%%%%%%%%%%%%%%%%%%%%%%%%%%%%%%%%%%%%%%
\begin{frame}{Outline}
  \tableofcontents[currentsection] 
\end{frame}

%% Also try in the preamble:
%\AtBeginSection[]
%{
%  \begin{frame}{Outline}
%    \tableofcontents[currentsection]
%  \end{frame}
%}

%%%%%%%%%%%%%%%%%%%%%%%%%%%%%%%%%%%%%%%%%%%%%%%%%%%%%%%%%%%%%%%%%%%%%%%%%%%%
\sectionheader[Optional subtitle or figure]{Section Header}

\sectionheader[\huge\color{tug}\faBalanceScale]{Section Header}

%%%%%%%%%%%%%%%%%%%%%%%%%%%%%%%%%%%%%%%%%%%%%%%%%%%%%%%%%%%%%%%%%%%%%%%%%%%%
\section{Technical content}

%%%%%%%%%%%%%%%%%%%%%%%%%%%%%%%%%%%%%%%%%%%%%%%%%%%%%%%%%%%%%%%%%%%%%%%%%%%%
\newcommand{\blue}[1]{{\color{tugblue}{#1}}}
% Note that the standard CM fonts (the fallback for math symbols) are not scaled
% correctly with the [webfont] and default [] options. If this bothers you, try
%\usepackage{lmodern}
%\usepackage[scale=1.1]{variablelm}
\begin{frame}{Math and URLs}
  A multi-line equation about $\blue{x}$ and $\blue{t}$:
	\begin{align*}
    u(\blue{x},\blue{t}) &= \sum_{k=1}^{\infty} f_k \sin \frac{k \pi \blue{x}}{L} \cos \frac{k \pi \blue{t}}{aL} + \\
                         & + \sum_{k=1}^{\infty} g_k \sin \frac{k \pi \blue{x}}{L} \sin \frac{k \pi \blue{t}}{aL}
	\end{align*}
\end{frame}

%%%%%%%%%%%%%%%%%%%%%%%%%%%%%%%%%%%%%%%%%%%%%%%%%%%%%%%%%%%%%%%%%%%%%%%%%%%%
\begin{frame}[fragile]{Verbatim}
	\begin{semiverbatim}
Subject: Next meeting
\alert{X-TUGAntiSpamFlag: spam}
To: "Jane Doe" <jane.doe@tugraz.at>
From: "Mary Sue" <mary.sue@tugraz.at>
	\end{semiverbatim}
\end{frame}

%%%%%%%%%%%%%%%%%%%%%%%%%%%%%%%%%%%%%%%%%%%%%%%%%%%%%%%%%%%%%%%%%%%%%%%%%%%%
\begin{frame}[fragile]{Source code}
  \begin{lstlisting}[language=C,basicstyle=\ttfamily]
int find(int *A, int x, int n) {
  for(int i = 0; i < n; ++i) {
    if(A[i] == x) {
      return i;
    }
  }
  return -1;
}
  \end{lstlisting}
\end{frame}

%%%%%%%%%%%%%%%%%%%%%%%%%%%%%%%%%%%%%%%%%%%%%%%%%%%%%%%%%%%%%%%%%%%%%%%%%%%%
\section{Tables and Diagrams}

%%%%%%%%%%%%%%%%%%%%%%%%%%%%%%%%%%%%%%%%%%%%%%%%%%%%%%%%%%%%%%%%%%%%%%%%%%%%
\begin{frame}{Table -- bold}
  % requires the tabu package
  \tabulinesep=2mm % wider spacing (default: 1mm)
  \begin{tabu}{lll}
    \tugtabu
    No.  & Title & Due \\
    D1.1 & Data Management Plan (DMP) & M6 \\
    D1.2 & Intermediate Report & M18 \\
    D1.3 & Final Report on Data Management & M36 \\
  \end{tabu}
\end{frame}

%%%%%%%%%%%%%%%%%%%%%%%%%%%%%%%%%%%%%%%%%%%%%%%%%%%%%%%%%%%%%%%%%%%%%%%%%%%%
\begin{frame}{Table -- light}
  \begin{tabu}{lll}
    \tugtabulite
    No.  & Title & Due \\
    D1.1 & Data Management Plan (DMP) & M6 \\
    \hline
    D1.2 & Intermediate Report & M18 \\
    \hline
    D1.3 & Final Report on Data Management & M36 \\
    \hline
  \end{tabu}
\end{frame}

\begin{frame}{Table -- light}
  \footnotesize
  \begin{tabu}{X[L]X[L]X[L]}
    \tugtabulite
    Überschrift 1 & Überschrift 2 & Überschrift 3 \\
    Platzhaltertext Information dazu & Das ist Platzhaltertext & Platzhaltertext \\
    \hline
    Zweite Zeile & Und noch eine Zeile & Mehr Information \\
    \hline
    Zweite Zeile & Und noch eine Zeile & Mehr Information \\
    \hline
    Zweite Zeile & Und noch eine Zeile & Mehr Information \\
    \hline
  \end{tabu}
\end{frame}

%%%%%%%%%%%%%%%%%%%%%%%%%%%%%%%%%%%%%%%%%%%%%%%%%%%%%%%%%%%%%%%%%%%%%%%%%%%%
\begin{frame}{Color scheme for PGFplots}
  \centering
  \begin{tikzpicture}
    \begin{axis}[cycle list name=tug, no markers, every axis plot/.append style=thick,width=10cm,height=6cm]
      \addplot {rnd*.25-.25};
      \addplot {rnd*.25-0.5};
      \addplot {rnd*.25-0.75};
      \addplot {rnd*.25-1};
      \addplot {rnd*.25+x*0.05-0.75};
      \addplot {rnd*.25+x*0.05-0.75};
    \end{axis}
  \end{tikzpicture}
\end{frame}

%%%%%%%%%%%%%%%%%%%%%%%%%%%%%%%%%%%%%%%%%%%%%%%%%%%%%%%%%%%%%%%%%%%%%%%%%%%%
\begin{frame}{Another Plot}
  \centering
  \begin{tikzpicture}
    \begin{axis}[ymin=0,ymax=1,enlargelimits=false,width=8cm,height=6cm]
      \addplot [colE,fill=colE,fill opacity=0.65,very thick] coordinates {
        (0,0.4) (0.2,0.75) (1,0.75)
      } |- (0,0) -- cycle;
      \addplot [colD,fill=colD,fill opacity=0.65,very thick] coordinates {
        (0,0.1)
        (0.1,0.15) (0.2,0.5)
        (0.3,0.62)
        (0.4,0.56) (0.5,0.58) (0.6,0.65) (0.7,0.6)
        (0.8,0.58) (0.9,0.55) (1,0.52)
      } |- (0,0) -- cycle;
      \addplot [colA,fill=colA,opacity=0.65,very thick] coordinates {
        (0,0.25)
        (0.1,0.27) (0.2,0.24) (0.3,0.24)
        (0.4,0.26) (0.5,0.3)
        (0.6,0.23) (0.7,0.2)
        (0.8,0.15) (0.9,0.1)
        (1,0.1)
      } |- (0,0) -- cycle;
    \end{axis}
  \end{tikzpicture}
\end{frame}

%%%%%%%%%%%%%%%%%%%%%%%%%%%%%%%%%%%%%%%%%%%%%%%%%%%%%%%%%%%%%%%%%%%%%%%%%%%%
\begin{frame}{Bar Chart}
  \centering
  \begin{tikzpicture}
    \begin{axis}[
      %title={Diagram title},
      enlargelimits=0.15,
      font=\scriptsize,
      legend style={at={(0.5,-0.15)}, anchor=north,legend columns=-1,font=\scriptsize,draw=none},
      width=1.0\textwidth,height=.7\textheight,
      ybar,
      cycle list name=tugfill,
      bar width=5pt,
      ymajorgrids,
      symbolic x coords={Kategorie 1, Kategorie 2, Kategorie 3, Kategorie 4},
      xtick=data,
      xtick style={draw=none},xtick pos=left,xticklabel shift=-4pt,
      ymin=0,ymax=5.5,ytick distance=1,
      enlarge y limits=false,
      ]
      \addplot coordinates {
          (Kategorie 1,4.1)
          (Kategorie 2,2.5)
          (Kategorie 3,3.5)
          (Kategorie 4,4.5)
        };
      \addplot coordinates {
          (Kategorie 1,2.2)
          (Kategorie 2,4.5)
          (Kategorie 3,1.8)
          (Kategorie 4,2.8)
        };
      \addplot coordinates {
          (Kategorie 1,4.5)
          (Kategorie 2,3.5)
          (Kategorie 3,1.5)
          (Kategorie 4,2.5)
        };
      \addplot coordinates {
          (Kategorie 1,3.8)
          (Kategorie 2,2.8)
          (Kategorie 3,1.8)
          (Kategorie 4,4.8)
        };
      \addplot coordinates {
          (Kategorie 1,2.3)
          (Kategorie 2,3.3)
          (Kategorie 3,4.3)
          (Kategorie 4,1.3)
        };
      \addplot coordinates {
          (Kategorie 1,3.1)
          (Kategorie 2,2.1)
          (Kategorie 3,1.1)
          (Kategorie 4,4.1)
        };
      \legend{Reihe 1,
              Reihe 2,
              Reihe 3,
              Reihe 4,
              Reihe 5,
              Reihe 6}
    \end{axis}
  \end{tikzpicture}
\end{frame}

%%%%%%%%%%%%%%%%%%%%%%%%%%%%%%%%%%%%%%%%%%%%%%%%%%%%%%%%%%%%%%%%%%%%%%%%%%%%
\begin{frame}{Pie Chart}
  \centering
  \begin{tikzpicture}[thick]
    \pgfmathsetmacro{\globalsumperc}{0}

    \foreach \name/\col/\perc in {A/colA/10,
                                  B/colB/20,
                                  C/colC/30,
                                  D/colD/9,
                                  E/colE/9,
                                  F/colF/22} {
      \pgfmathsetmacro{\sumperc}{\globalsumperc+\perc}
      \draw[white,fill=\col] (0,0) -- (\globalsumperc*3.6:2cm) arc [start angle=\globalsumperc*3.6, end angle=\sumperc*3.6, radius=2cm] -- cycle;
      \draw[white] (\globalsumperc*3.6+\perc*1.8:1.25cm) node {\perc\%};
      \global\let\globalsumperc=\sumperc
    }
  \end{tikzpicture}
\end{frame}

%%%%%%%%%%%%%%%%%%%%%%%%%%%%%%%%%%%%%%%%%%%%%%%%%%%%%%%%%%%%%%%%%%%%%%%%%%%%
\section{Bibliography}

%%%%%%%%%%%%%%%%%%%%%%%%%%%%%%%%%%%%%%%%%%%%%%%%%%%%%%%%%%%%%%%%%%%%%%%%%%%%
\begin{frame}{Bibliographic references}
  Short citation: \cite{cryptoAuthor18}

  In their paper, \textcite{cryptoAuthor18} show that \dots % requires biblatex
\end{frame}

%%%%%%%%%%%%%%%%%%%%%%%%%%%%%%%%%%%%%%%%%%%%%%%%%%%%%%%%%%%%%%%%%%%%%%%%%%%%
\begin{frame}[allowframebreaks]{Bibliography}
  \printbibliography
\end{frame}

%%%%%%%%%%%%%%%%%%%%%%%%%%%%%%%%%%%%%%%%%%%%%%%%%%%%%%%%%%%%%%%%%%%%%%%%%%%%
\setbeamertemplate{bibliography item}[book]
\begin{frame}[allowframebreaks]{Bibliography}
  \printbibliography
\end{frame}

\begin{filecontents*}{\jobname.bib}
@inproceedings{cryptoAuthor18,
  author       = {First Author and
                  Second Author and
                  Third Author},
  title        = {On Publications and their Bibliographic Representation},
  booktitle    = {CONF 2018},
  publisher    = {Springer},
  year         = {2018},
}
\end{filecontents*}

\end{document}
