\PassOptionsToPackage{dvispnames, table}{xcolor}
\documentclass[aspectratio=169]{beamer}
\usetheme[institute]{tugraz2018}
\usepackage[beamer]{prettytex/base}
\usepackage{prettytex/math}
\usepackage{prettytex/gfx}
\usepackage{tabularray}
\usepackage{neuralnetwork}
\usepackage{cancel}
\usepackage{braket}
\usepackage[nameinlink]{cleveref}

\setlength{\headheight}{19.53pt}
\setlength{\headsep}{1.8em}
\setlength{\belowcaptionskip}{-12pt}

\makeatletter
\renewcommand\listoffigures{
  \section*{\listfigurename}
  \@starttoc{lof}
}
\renewcommand\listoftables{
  \section*{\listtablename}
  \@starttoc{lot}
}
\makeatother

\crefname{figure}{figure}{figures}
\crefname{tcb@cnt@definition}{definition}{definitions}
\crefname{tcb@cnt@theorem}{theorem}{theorems}
\crefname{tcb@cnt@lemma}{lemma}{lemmata}
\crefname{tcb@cnt@corollary}{corollary}{corollaries}
\crefname{tcb@cnt@remark}{remark}{remarks}


\tikzexternalize[prefix=tikz/]
\renewcommand{\inputtikz}[1]{
  \tikzsetnextfilename{#1}
  \input{../thesis/#1.tex}
}

\title[Secure Classification as a Service]{
  Secure Classification as a Service \\
  \small\normalfont\textcolor{black}{
    Levelled Homomorphic, Post-Quantum Secure Machine Learning Inference \\
    based on the CKKS Encryption Scheme
  }
}
\author{Peter Waldert}
\date{Bachelor Thesis Presentation, 01.08.2022}
\institute{IAIK}
\instituteurl{iaik.tugraz.at}

\institutelogo{beamerthemetugraz/institute/IAIK}
% \additionallogo{figures/logo}  % additional institute/department logo (footline; optional)
% \logobar{Supported by: ...}  % sponsors (titlepage; optional)

% Ungefähre Struktur:
% * Motivation aus zwei Richtungen: Privacy-Preserving Machine Learning möglich? (Anwendungen in der Medizin) -> HE
%   netter Side-Effect: Post-Quantum Security via Lattice Cryptography
% * Beispiel für HE und Public-Key Asymmetric Encryption zur Veranschaulichung: RSA
% * Learning With Errors und andere Lattice Probleme (nur ein paar Sätze)
% * Relation von CKKS zu LWE: Wie der Private Key geschützt ist und Hardness des LWE-Problems.
% * CKKS Scheme erklären, dazu: Polynomring Z_q[X]/(X^N+1) erläutern, wie wird encoded und verschlüsselt
% * Zeigen, dass Encrypt(Encode(z)) + Encrypt(Encode(z')) == Encrypt(Encode(z + z')) in CKKS
% * Implementation erklären (SEAL, Struktur in Server/Client, etc.), Demo der UI.
% * Verschiedene Matmul-Implementationen, Optimierungen
% * Ergebnisse: Performance des Netzwerks, Benchmarks, Ciphertext-Visualisierungen.

\setlength{\headheight}{19.53pt}

\makenoidxglossaries
\newacronym{he}{HE}{Homomorphic Encryption}
\newacronym{fhe}{FHE}{Fully Homomorphic Encryption}
\newacronym{bfv}{BFV}{Brakerski-Fan-Vercauteren}
\newacronym{bgv}{BGV}{Brakerski-Gentry-Vaikuntanathan}
\newacronym{ckks}{CKKS}{Cheon-Kim-Kim-Song}
\newacronym{rsa}{RSA}{Rivest-Shamir-Adleman}
\newacronym{aes}{AES}{Advanced Encryption Standard}
\newacronym{lwe}{LWE}{Learning With Errors}
\newacronym{dlwe}{DLWE}{Decision Learning With Errors}
\newacronym{rlwe}{RLWE}{Learning With Errors on Rings}
\newacronym{tls}{TLS}{Transport Layer Security}
\newacronym{ml}{ML}{Machine Learning}
\newacronym{gd}{GD}{Gradient Descent}
\newacronym{mse}{MSE}{Mean-Squared-Error}
\newacronym{dft}{DFT}{Discrete Fourier Transform}
\newacronym{fft}{FFT}{Fast Fourier Transform}
\newacronym{iff}{iff}{if and only if}
\newacronym{np}{NP}{Non-deterministic Polynomial time}
\newacronym{ppml}{PPML}{Privacy-Preserving Machine Learning}
\newacronym{mnist}{MNIST}{Modified National Institute of Standards and Technology database}
\newacronym{sis}{SIS}{Shortest Integer Solution}
\newacronym{svp}{SVP}{Shortest Vector Problem}
\newacronym{gapsvp}{GapSVP}{Decisional Approximate Shortest Vector Problem}
\newacronym{fhew}{FHEW}{Fastest Homomorphic Encryption in the West}
\newacronym{tfhe}{TFHE}{Torus Fully Homomorphic Encryption}
\newacronym{crt}{CRT}{Chinese Remainder Theorem}
\newacronym{rns}{RNS}{Residue Number System}

\newcommand{\cpp}[1]{\mintinline{cpp}{#1}}
\newcommand{\name}[1]{\textsc{#1}}
\newcommand{\cryptop}[1]{\text{\textcolor{darkpurple}{#1}}}
\newcommand{\inv}{^{-1}}
\newcommand{\inputtikz}[1]{
  % \tikzsetnextfilename{#1}
  % \input{#1.tex}
  \vspace{2cm}
  Placeholder
  \vspace{2cm}
}

\addbibresource{../library/sources.bib}

\begin{document}
  \begin{frame}[plain]
    \maketitle
  \end{frame}

  \begin{frame}{Outline}
    \tableofcontents
  \end{frame}

  \section{Introduction}
  \begin{frame}{Privacy for Medical Applications}
    \begin{itemize}
      \item Development of new applications and solutions in health care, but: very sensitive data.
      \item For instance, RNA sequences, images of skin, lab data, medical records, etc.
      \item The results are even more volatile: Disease predictions
      \item $\Rightarrow$ Demand for privacy-preserving solutions in machine learning applications.
    \end{itemize}
  \end{frame}

  \begin{frame}{Post-Quantum Security}
    \begin{figure}[H]
      \centering
      \inputtikz{figures/wave-function}
      \caption{Illustration of a wave function $\psi$ as commonly used in quantum mechanics.}
      \label{fig:wave-function}
    \end{figure}
  \end{frame}

  \begin{frame}{The \gls{rsa} Scheme}
    From the integers $\Z$, define the quotient ring $(\Z/q\Z, +, \cdot)$ for some modulus $q \in \N$.

    With unpadded RSA \parencite{1983-rsa}, some arithmetic can be performed on the ciphertext - looking at the encrypted ciphertext $\mathcal{E}: \Z/q\Z \mapsto \Z/q\Z,\, \mathcal{E}(m) := m^r \mod q$ ($r, q \in \N$) of the message $m_1, m_2 \in \Z/q\Z$ respectively, the following holds:
    \begin{align*}
      \mathcal{E}(m_1) \cdot \mathcal{E}(m_2)
       & \equiv (m_1)^r (m_2)^r \mod q            \\
       & \equiv (m_1 m_2)^r \mod q                \\
       & \equiv \mathcal{E}(m_1 \cdot m_2) \mod q
    \end{align*}
  \end{frame}

  \section{Lattice Cryptography and RLWE}
  \begin{frame}{The \gls{lwe} Problem}
    \begin{definition}[LWE-Distribution $A_{\vec{s}, \chi_{error}}$]
      Given a prime $q \in \N$ and $n \in \N$, we choose some secret $\vec{s} \in (\Z / q \Z)^n$.
      In order to sample a value from the LWE distribution $A_{\vec{s}, \chi_{error}}$:
      \begin{itemize}
        \item Draw a random vector $a \in (\Z/q\Z)^n$ from the multivariate uniform distribution with its domain in the integers up to $q$.
        \item Given another probability distribution $\chi_{error}$ over the integers modulo $q$, sample a scalar 'error term' $\mu \in \Z / q \Z$ from it, often also referred to as noise.
        \item Set $b = \vec{s} \cdot \vec{a} + \mu$, with $\cdot$ denoting the standard vector product.
        \item Output the pair $(\vec{a}, b) \in (\Z / q \Z)^n \times (\Z / q \Z)$.
      \end{itemize}
    \end{definition}

    Search-LWE-Problem:
    Given $m$ independent samples $(\vec{a}_i, b_i)_{0 < i \leq m}$ from $A_{\vec{s}, \chi_{error}}$, find $\vec{s}$.
  \end{frame}

  \begin{frame}{Polynomial Rings}
    \begin{columns}
      \begin{column}{0.5\linewidth}
        \begin{definition}[Cyclotomic Polynomial]
          Given the $n$\th roots of unity $\{\xi_k\}$, define $\Phi_n \in \Z[X]$ as
          $$\Phi_n(x) := \prod_{\stackrel{k=1}{\xi_k \mathrm{primitive}}}^{n} (x - \xi_k) \,.$$
          It is unique for each given $n \in \N$.
        \end{definition}
      \end{column}
      \begin{column}{0.32\linewidth}
        \begin{figure}
          \scalebox{0.7}{\inputtikz{figures/nth-roots-of-unity}}
          \caption{The 5\th roots of unity}
          \label{fig:nth-roots-of-unity}
        \end{figure}
      \end{column}
    \end{columns}
  \end{frame}

  \begin{frame}{Some Notation}
    \begin{itemize}
      \item $\Z[X] := \big\{p: \C \mapsto \C \,, p(x) = \sum_{k=0}^\infty a_k x^k, a_k \in \Z \;\forall k \geq 0\big\}$
      \item $\Z_q[X] := (\Z/q\Z)[X]$
      \item $\Z_q[X] / \Phi_M(X)$ using the $M$\th cyclotomic polynomial
      \item $\Z_q[X]/(X^N+1)$ for $N$ a power of 2. \pause
            \begin{itemize}
              \item Elements are polynomials of degree $N-1$ with integer coefficients modulo $q$.
            \end{itemize}
    \end{itemize}
  \end{frame}

  \begin{frame}{The \gls{rlwe} Problem}
    \begin{corollary}[RLWE-Distribution $B_{\vec{s}, \chi_{error}}$]
      Given a quotient $(R/qR, +, \cdot)$, we choose some secret $s \in R/qR$.
      In order to sample a value from the RLWE distribution $B_{s, \chi_{error}}$:
      \begin{itemize}
        \item Uniformly randomly draw an element $a \in R/qR$
        \item Given another probability distribution $\chi_{error}$ over the ring elements, sample an 'error term' $\mu \in R/qR$ from it, also referred to as noise.
        \item Set $b = s \cdot a + \mu$, with $\cdot$ denoting the ring multiplication operation.
        \item Output the pair $(a, b) \in R/qR \times R/qR$.
      \end{itemize}
    \end{corollary}

    Use it to construct a cryptosystem...
    Idea: Attacker needs to solve \gls{lwe} given the ciphertext and public key.
  \end{frame}

  \section{The CKKS Scheme}
  \begin{frame}{Overview of \gls{ckks}}
    \begin{figure}[H]
      \centering
      \scalebox{0.8}{\inputtikz{figures/ckks-schematic}}
      \caption[Schematic overview of the CKKS scheme]{
        Schematic overview of CKKS, adapted from \cite{2020-cryptotree}.
        A plain vector $\vec{z} \in \C^{N/2}$ is encoded to $m = \cryptop{CKKS.Encode}(\vec{z})$, encrypted to $\vec{c} = \cryptop{CKKS.Encrypt}(\vec{p}, m)$, decrypted and decoded to a new $\tilde{\vec{z}} = \cryptop{CKKS.Decode}(\cryptop{CKKS.Decrypt}(s, \tilde{\vec{c}}))$.
      }
      \label{fig:ckks-overview}
    \end{figure}
    \cite{2017-ckks-original}
  \end{frame}

  \begin{frame}{Encoding and Decoding}
    \cryptop{CKKS.} \\
    \begin{tblr}{Q[l,h]Q[l,h,\textwidth - 3.5cm]}
      \cryptop{Encode}$(\vec{z})$ & {For a given input vector $\vec{z}$, output
          $m = (\underline{\sigma}\inv \circ \underline{\rho_\delta}\inv \circ \underline{\pi}\inv)(\vec{z}) = \underline{\sigma}\inv(\lfloor \delta \cdot \underline{\pi}\inv(\vec{z})\rceil_{\underline{\sigma}(R)})$ $\quad\rightarrow m$} \\
      \cryptop{Decode}$(m)$ & {Decode plaintext $m$ as
          $\vec{z} = (\underline{\pi} \circ \underline{\rho_\delta} \circ \underline{\sigma})(m) = (\underline{\pi} \circ \underline{\sigma})(\delta\inv m)$
          $\quad\rightarrow \vec{z}$} \\
    \end{tblr}
    \begin{itemize}
      \item Three transformations: $\underline{\sigma}\inv$, $\underline{\rho_\delta}\inv$ and $\underline{\pi}\inv$.
      \item Key idea: Homomorphic property, they preserve additivity and multiplicativity.
    \end{itemize}
  \end{frame}

  \begin{frame}{Encryption and Decryption}
    \cryptop{CKKS.} \\
    \begin{tblr}{Q[l,h]Q[l,h,\textwidth - 3.5cm]}
      \cryptop{Encrypt}$(\vec{p}, m)$ & {
          Let $(b,a) = \vec{p}$, $u \leftarrow \chi_{enc}$, $\mu_1, \mu_2 \leftarrow \chi_{error}$,
          then the ciphertext is $\vec{c} = u \cdot \vec{p} + (m + \mu_1, \mu_2) = (m + bu + \mu_1, au + \mu_2)$
          $\quad\rightarrow \vec{c}$} \\
      \cryptop{Decrypt}$(s, \vec{c})$ & {
          Decrypt the ciphertext $\vec{c} = (c_0, c_1)$ as $m = \lbrack c_0 + c_1 s\rbrack_{q_L}$
          $\quad\rightarrow m$} \\
    \end{tblr}
    \begin{itemize}
      \item A public-key cryptosystem! Encrypt with $\vec{p}$, decrypt with $s$.
      \item Leaves the attacker with the \gls{rlwe} problem.
      \item Decrypts correctly under certain conditions...
    \end{itemize}
  \end{frame}

  \begin{frame}{Homomorphic Addition}
    \begin{tblr}{Q[l,h]Q[l,h,\textwidth - 3.5cm]}
      \cryptop{CKKS.Add}$(\vec{c}_1, \vec{c}_2)$ & {
          Output $\vec{c}_3 = \vec{c}_1 + \vec{c}_2$
          $\quad\rightarrow \vec{c}_3$} \\
    \end{tblr}

    Decrypts correctly?
    \begin{align*}
      \cryptop{BFV.Decrypt}(s, \overline{\vec{c}})
       & = \lfloor \delta\inv [\overline{c_0} + \overline{c_1} s]_t \rceil                                                                                                                                                         \\
       & = \big\lfloor \delta\inv [\delta \overline{m} + b \overline{u} + \overline{\mu_1} + (a \overline{u} + \overline{\mu_2}) s]_t \big\rceil                                                                                   \\
       & = \big\lfloor [(\delta\inv\delta) \overline{m} + \delta\inv b \overline{u} + \delta\inv \overline{\mu_1} + \delta\inv a s \overline{u} + \delta\inv \overline{\mu_2} s]_t \big\rceil                                      \\
       & = \big\lfloor [\overline{m} - \cancel{\delta\inv as \overline{u}} - \delta\inv \tilde{\mu} \overline{u} + \delta\inv \overline{\mu_1} + \cancel{\delta\inv as \overline{u}} + \delta\inv \overline{\mu_2} s]_t \big\rceil \\
       & = \big\lfloor [\overline{m} + \underbrace{\delta\inv (\overline{\mu_1} + \overline{\mu_2} s - \tilde{\mu} \overline{u})}_{:= \epsilon \,, ||\epsilon|| \ll 1}]_t \big\rceil
      \approx \big\lfloor [\overline{m}]_t \big\rceil = \lfloor \overline{m} \rceil \approx \overline{m}
    \end{align*}
  \end{frame}

  \section{Implementation Goal and Methods}
  \begin{frame}{Goal: Classify MNIST}
    \begin{itemize}
      \item Two main types of \gls{ml}: Supervised and Unsupervised Learning
      \item Popular dataset: \gls{mnist}. Encode as vector of 784 entries.
    \end{itemize}

    \begin{figure}[H]
      \centering
      \scalebox{0.8}{\inputtikz{figures/mnist}}
      \caption[Sample images of the MNIST dataset]{
        Sample images of the MNIST dataset of handwritten digits \parencite{mnist-original}.
        The dataset contains 70,000 images of $28 \times 28$ greyscale pixels valued from 0 to 255 as well as associated labels (as required for supervised learning).
      }
      \label{fig:mnist}
    \end{figure}
  \end{frame}

  \begin{frame}{Neural Networks}
    \begin{figure}[H]
      \centering
      \scalebox{0.9}{\inputtikz{figures/neural-network}}
      \caption[Neural Network illustration resembling the one used in our demonstrator]{
        A simple neural network resembling the structure we use in our demonstrator with $\vec{h} = \cryptop{relu}(M_1 \vec{x} + \vec{b_1})$ and the output $\vec{y} = \cryptop{softmax}(M_2 \vec{h} + \vec{b_2})$.
      }
      \label{fig:neural-network}
    \end{figure}
  \end{frame}

  \begin{frame}{Matrix Multiplication: The Na\"ive Method}
    \begin{figure}[H]
      \centering
      \hspace{-3cm}
      \scalebox{0.9}{\inputtikz{figures/generated/matmul-naive}}
      \caption[Naïve matrix multiplication method]{The naïve method to multiply a square matrix with a vector (adapted from \cite{2018-gazelle}).}
      \label{fig:naive-method}
    \end{figure}
  \end{frame}

  \begin{frame}{Matrix Multiplication: The Diagonal Method}
    \begin{figure}[H]
      \centering
      \hspace{-3cm}
      \scalebox{0.9}{\inputtikz{figures/generated/matmul-diagonal}}
      \caption[Diagonal matrix multiplication method]{The diagonal method to multiply a square matrix with a vector (adapted from \cite{2018-gazelle}).}
      \label{fig:diagonal-method}
    \end{figure}
  \end{frame}

  \begin{frame}{Matrix Multiplication: The Hybrid Method}
    \begin{figure}[H]
      \centering
      \hspace{-3cm}
      \scalebox{0.9}{\inputtikz{figures/generated/matmul-hybrid}}
      \caption[Hybrid matrix multiplication method]{The hybrid method to multiply an arbitrarily sized matrix with a vector (adapted from \cite{2018-gazelle}).}
      \label{fig:hybrid-method}
    \end{figure}
    Similar performance: The BabyStep-Giantstep Method.
  \end{frame}

  \section{Live Demo of the WebApp}
  \begin{frame}{Demo: Secure Handwritten Digit Classification as a Service}
    \qrcode[nolink,height=2cm]{https://secure-classification.peter.waldert.at/}
    \url{https://secure-classification.peter.waldert.at/}
  \end{frame}

  \section{Results: Network Analysis and Performance Benchmarks}
  \begin{frame}{Chaos everywhere: The Confusion Matrix}
    \centering
    \scalebox{0.7}{\inputtikz{figures/generated/confusion-matrix}}
  \end{frame}

  \begin{frame}{Runtime Benchmarks}
    \begin{table}[H]
      \centering
      \caption[Performance Benchmarks / Communication Overhead]{
        Performance benchmarks and communication overhead of the classification procedure on an Intel\textregistered \, i7-5600U CPU, including the encoding and decoding steps.
      }
      % \captionsetup{margin=10pt}
      % \caption*{
      %   $\bm{B_1}$ ... Coefficient Moduli start bits (also equal to the last) \\
      %   $\bm{B_2}$ ... Coefficient Moduli middle bits, also defines the scale as $2^{B_2}$ \\
      %   $\bm{N}$ ... Polynomial Modulus Degree, found in the exponent of $p(X) = X^N + 1$ \\
      %   $\bm{T}$ ... Runtime of encryption + classification + decryption \\
      %   $\bm{M}$ ... Message Size (Relin Keys + Galois Keys + Request Ciphertext + Response Ciphertext) \\
      %   $\bm{\Delta}$ ... Mean Max-Relative Error compared to the exact result, i.e. $\bm{\Delta} := \frac{\langle |\bm{y}_{prediction} - \bm{y}_{exact}| \rangle}{\max |\bm{y}_{exact}|}$
      % }
      \SetTblrInner{rowsep=1pt}
      \scalebox{0.8}{
        \begin{tblr}{
          colspec={ccrrrrrrr},
          row{2,3,4} = {bg=azure9},
          row{5,6,7} = {bg=violet9},
          row{8,9,10} = {bg=blue9},
          row{11,12,13} = {bg=azure9},
            }
          \hline
          \bf Mode & \bf SecLevel & $\bm{B_1}$ & $\bm{B_2}$ & $\bm{N}$ & \bf MatMul & $\bm{T}$ / \si{\second} & $\bm{M}$ / \si{\mebi\byte} & $\bm{\Delta}$ / 1 \\
          \hline
          Release & tc128 & 34 & 25 & 8192 & Diagonal & 8.39 & 132.72 & 0.0364 \\
          & & & & & Hybrid & 1.35 & 132.72 & 0.0362 \\
          & & & & & BSGS & 1.66 & 132.72 & 0.1433 \\
          \hline
          & tc128 & 60 & 40 & 16384 & Diagonal & 17.24 & 286.51 & 0.0363 \\
          & & & & & Hybrid & 3.05 & 286.51 & 0.0364 \\
          & & & & & BSGS & 3.66 & 286.51 & 0.1399 \\
          \hline
          & tc256 & 60 & 40 & 32768 & Diagonal & 35.24 & 615.16 & 0.0363 \\
          & & & & & Hybrid & 5.99 & 615.16 & 0.0364 \\
          & & & & & BSGS & 7.34 & 615.16 & 0.1399 \\
        \end{tblr}
      }
      \label{tab:performance-benchmarks}
    \end{table}
  \end{frame}

  \begin{frame}{Ciphertext Visualisations}
    \begin{figure}[H]
      \centering
      \scalebox{0.9}{\inputtikz{figures/ciphertext-visualisation}}
      \caption[Visualisation of the plain input images compared to their ciphertext]{Ciphertext Visualisation: The first row corresponds to the images in plain, the second row depicts an encrypted version, namely the reconstructed polynomial coefficients $a_k$ of the ciphertext polynomial.}
      \label{fig:ciphertext-visualisation}
    \end{figure}
  \end{frame}

  \section*{}
  \begin{frame}{Conclusion}
    Crypto is good for us
  \end{frame}

  %\begin{frame}[c,plain]
  \begin{frame}[c]
    \centering
    \Large Questions?
  \end{frame}

  \begin{frame}[allowframebreaks]{Glossary}
    \printnoidxglossary[type=acronym]
  \end{frame}

  \begin{frame}[allowframebreaks]{Bibliography}
    \printbibliography
  \end{frame}
\end{document}
