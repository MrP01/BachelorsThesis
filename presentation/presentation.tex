\PassOptionsToPackage{dvispnames, table}{xcolor}
\documentclass[aspectratio=169, onlytextwidth]{beamer}
\usetheme[institute]{tugraz2018}
\usepackage[beamer]{prettytex/base}
\usepackage{prettytex/math}
\usepackage{prettytex/gfx}
\usepackage{tabularray}
\usepackage{neuralnetwork}
\usepackage{cancel}
\usepackage{braket}
\usepackage[backend=biber, style=numeric, maxbibnames=20]{biblatex}
\usepackage[acronym, style=long3col, indexonlyfirst=true, nogroupskip=true]{glossaries}
\usepackage{csquotes}
\usepackage{qrcode}

\addtobeamertemplate{frametitle}{}{\vspace*{-1.5em}}


\tikzexternalize[prefix=tikz/]
\renewcommand{\inputtikz}[1]{
  \tikzsetnextfilename{#1}
  \input{../thesis/#1.tex}
}

% TODO: Galois Rotations
% TODO: Überleitungen! (zu FHE), etc.
% TODO: Sprachnachricht nochmal anhören

\title[Secure Classification as a Service]{
  Secure Classification as a Service \\
  \small\normalfont\textcolor{black}{
    Levelled Homomorphic, Post-Quantum Secure Machine Learning Inference \\
    based on the CKKS Encryption Scheme
  }
}
\author{Peter Waldert}
\date{Bachelor Thesis Presentation, 01.08.2022}
\institute{IAIK}
\instituteurl{iaik.tugraz.at}

\institutelogo{beamerthemetugraz/institute/IAIK}
% \additionallogo{figures/logo}  % additional institute/department logo (footline; optional)
% \logobar{Supported by: ...}  % sponsors (titlepage; optional)

% Ungefähre Struktur:
% * Motivation aus zwei Richtungen: Privacy-Preserving Machine Learning möglich? (Anwendungen in der Medizin) -> HE
%   netter Side-Effect: Post-Quantum Security via Lattice Cryptography
% * Beispiel für HE und Public-Key Asymmetric Encryption zur Veranschaulichung: RSA
% * Learning With Errors und andere Lattice Probleme (nur ein paar Sätze)
% * Relation von CKKS zu LWE: Wie der Private Key geschützt ist und Hardness des LWE-Problems.
% * CKKS Scheme erklären, dazu: Polynomring Z_q[X]/(X^N+1) erläutern, wie wird encoded und verschlüsselt
% * Zeigen, dass Encrypt(Encode(z)) + Encrypt(Encode(z')) == Encrypt(Encode(z + z')) in CKKS
% * Implementation erklären (SEAL, Struktur in Server/Client, etc.), Demo der UI.
% * Verschiedene Matmul-Implementationen, Optimierungen
% * Ergebnisse: Performance des Netzwerks, Benchmarks, Ciphertext-Visualisierungen.

\setlength{\headheight}{19.53pt}

\makenoidxglossaries
\newacronym{he}{HE}{Homomorphic Encryption}
\newacronym{fhe}{FHE}{Fully Homomorphic Encryption}
\newacronym{bfv}{BFV}{Brakerski-Fan-Vercauteren}
\newacronym{bgv}{BGV}{Brakerski-Gentry-Vaikuntanathan}
\newacronym{ckks}{CKKS}{Cheon-Kim-Kim-Song}
\newacronym{rsa}{RSA}{Rivest-Shamir-Adleman}
\newacronym{aes}{AES}{Advanced Encryption Standard}
\newacronym{lwe}{LWE}{Learning With Errors}
\newacronym{dlwe}{DLWE}{Decision Learning With Errors}
\newacronym{rlwe}{RLWE}{Learning With Errors on Rings}
\newacronym{tls}{TLS}{Transport Layer Security}
\newacronym{ml}{ML}{Machine Learning}
\newacronym{gd}{GD}{Gradient Descent}
\newacronym{mse}{MSE}{Mean-Squared-Error}
\newacronym{dft}{DFT}{Discrete Fourier Transform}
\newacronym{fft}{FFT}{Fast Fourier Transform}
\newacronym{qft}{QFT}{Quantum Fourier Transform}
\newacronym{iff}{iff}{if and only if}
\newacronym{np}{NP}{Non-deterministic Polynomial time}
\newacronym{ppml}{PPML}{Privacy-Preserving Machine Learning}
\newacronym{mnist}{MNIST}{Modified National Institute of Standards and Technology}
\newacronym{sis}{SIS}{Shortest Integer Solution}
\newacronym{svp}{SVP}{Shortest Vector Problem}
\newacronym{gapsvp}{GapSVP}{Decisional Approximate Shortest Vector Problem}
\newacronym{fhew}{FHEW}{Fastest Homomorphic Encryption in the West}
\newacronym{tfhe}{TFHE}{Torus Fully Homomorphic Encryption}
\newacronym{crt}{CRT}{Chinese Remainder Theorem}
\newacronym{rns}{RNS}{Residue Number System}

% \newcommand{\cpp}[1]{\mintinline{cpp}{#1}}
\newcommand{\cpp}[1]{Placeholder}
\newcommand{\name}[1]{\textsc{#1}}
\newcommand{\cryptop}[1]{\text{\textcolor{darkpurple}{#1}}}
\newcommand{\inv}{^{-1}}
\newcommand{\inputtikz}[1]{
  % \tikzsetnextfilename{#1}
  % \input{#1.tex}
  \vspace{2cm}
  Placeholder
  \vspace{2cm}
}

\addbibresource{../library/sources.bib}

\begin{document}
  \begin{frame}[plain]
    \maketitle
  \end{frame}

  \begin{frame}{Outline}
    \tableofcontents
  \end{frame}

  \section{Introduction}
\begin{frame}{What do we want?}
  \begin{itemize}
    \item Encrypted \gls{ml} as a service
    \item Using homomorphic encryption
    \item That is Post-Quantum secure
    \item Which is somewhat fast
  \end{itemize}
\end{frame}

\begin{frame}{What did I do?}
  \begin{itemize}
    \item Fix stuff
  \end{itemize}
\end{frame}

\begin{frame}{\gls{ppml}}
  \begin{itemize}
    \item Development of new applications and solutions 'of numerical nature' in different fields
          \begin{itemize}
            \item Example: Health Care with highly sensitive medical data.
            \item For instance, RNA sequences, images of skin, lab data, medical records, etc.
            \item The results are even more volatile: disease indicators.
          \end{itemize}
    \item $\Rightarrow$ Demand for privacy-preserving solutions in \gls{ml} applications.
    \item By the way: Post-Quantum Secure Cryptosystems!
  \end{itemize}
\end{frame}

\begin{frame}{What about Long-Term Security?}
  \begin{columns}
    \begin{column}{0.6\linewidth}
      Quantum Computers affect Cryprography today:
      \begin{itemize}
        \item Problems believed to be \glstext{np}-hard on classical computers can be computed in polynomial time using a quantum computer.
        \item No hardness proof of the integer factorisation or RSA problems exist as of today.
        \item \name{Shor}'s, \name{Grover}'s and other algorithms can 'break' many cryptographic schemes used today.
        \item The existence of a sufficiently powerful quantum computer endangers the security of \glstext{tls}, etc.
      \end{itemize}
    \end{column}
    \begin{column}{0.4\linewidth}
      \vspace{-0.3cm}
      \begin{figure}[H]
        \centering
        \scalebox{0.7}{\inputtikz{figures/wave-function}}
        % \caption{Illustration of a wave function $\psi$ as commonly used in quantum mechanics.}
        \label{fig:wave-function}
      \end{figure}
    \end{column}
  \end{columns}
  Our Webservice is (from the point of todays knowledge) still secure in the presence of a quantum computer.
\end{frame}

  \section{Lattice Cryptography, LWE and RLWE}
\begin{frame}[c]
  \begin{columns}
    \begin{column}{0.32\linewidth}
      \begin{figure}
        \centering
        \scalebox{0.5}{\inputtikz{figures/lattice}}
        % \caption[Illustration of a standard lattice]{
        %   Illustration of a standard lattice $\lat$ with two basis vectors $\vec{b}_1$ and $\vec{b}_2$.
        % }
        \label{fig:lattice}
      \end{figure}
    \end{column}
    \begin{column}{0.6\linewidth}
      \begin{definition}[Lattice]
        A lattice $(\lat, +, \cdot)$ is a vector field over the integers $(\Z, +, \cdot)$, defined using a set of $n$ basis vectors $\vec{b_1}, \vec{b_2}, ..., \vec{b_n} \in \R^n$, that can be introduced as a set
        $$\lat := \bigg\{\sum_{i=1}^n c_i \vec{b}_i \,\bigg|\, c \in \Z\bigg\} \subseteq \R^n$$
        equipped with at least vector addition $+: \lat \times \lat \mapsto \lat$ and scalar multiplication $\cdot: \Z \times \lat \mapsto \lat$.
      \end{definition}
    \end{column}
  \end{columns}
\end{frame}

\begin{frame}{The \gls{lwe} Problem}
  \begin{definition}[LWE-Distribution $A_{\vec{s}, \chi_{error}}$]
    Given a prime $q \in \N$ and $n \in \N$, choose a secret $\vec{s} \in (\Z / q \Z)^n$.
    Sampling from $A_{\vec{s}, \chi_{error}}$:
    \begin{itemize}
      \item Sample a uniformly random vector $a \in (\Z/q\Z)^n$.
      \item Sample a scalar 'error term' $\mu \in \Z / q \Z$ from $\chi_{error}$, often also referred to as noise.
      \item Set $b = \vec{s} \cdot \vec{a} + \mu$, with $\cdot$ denoting the standard vector product.
      \item Output the pair $(\vec{a}, b) \in (\Z / q \Z)^n \times (\Z / q \Z)$.
    \end{itemize}
  \end{definition}

  Search-LWE-Problem:
  Given $m$ independent samples $(\vec{a}_i, b_i)_{0 < i \leq m}$ from $A_{\vec{s}, \chi_{error}}$, find $\vec{s}$.

  Published by \name{Regev} in 2005 \cite{2005-lwe-original}.
  Lead to the \glstext{fhe} scheme by \name{Gentry} in 2009 \cite{2009-gentry-fhe-original}.
\end{frame}

  \section{The CKKS Scheme}
\begin{frame}{Overview of \gls{ckks}}
  \begin{figure}[H]
    \centering
    \scalebox{0.8}{\inputtikz{figures/ckks-schematic}}
    \caption[Schematic overview of the CKKS scheme]{
      Schematic overview of CKKS \parencite{2017-ckks-original}, adapted from \cite{2020-cryptotree}.
      A plain vector $\vec{z} \in \C^{N/2}$ is encoded to $m = \cryptop{CKKS.Encode}(\vec{z})$, encrypted to $\vec{c} = \cryptop{CKKS.Encrypt}(\vec{p}, m)$, decrypted and decoded to a new $\tilde{\vec{z}} = \cryptop{CKKS.Decode}(\cryptop{CKKS.Decrypt}(s, \tilde{\vec{c}}))$.
    }
    \label{fig:ckks-overview}
  \end{figure}
\end{frame}

\begin{frame}{Encoding and Decoding}
  \cryptop{CKKS.} \\
  \begin{tblr}{Q[l,h]Q[l,h,\textwidth - 3.5cm]}
    \cryptop{Encode}$(\vec{z})$ & {For a given input vector $\vec{z}$, output
        $m = (\underline{\sigma}\inv \circ \underline{\rho_\delta}\inv \circ \underline{\pi}\inv)(\vec{z}) = \underline{\sigma}\inv(\lfloor \delta \cdot \underline{\pi}\inv(\vec{z})\rceil_{\underline{\sigma}(R)})$ $\quad\rightarrow m$} \\
    \cryptop{Decode}$(m)$ & {Decode plaintext $m$ as
        $\vec{z} = (\underline{\pi} \circ \underline{\rho_\delta} \circ \underline{\sigma})(m) = (\underline{\pi} \circ \underline{\sigma})(\delta\inv m)$
        $\quad\rightarrow \vec{z}$} \\
  \end{tblr}
  \begin{itemize}
    \item Three transformations: $\underline{\sigma}\inv$, $\underline{\rho_\delta}\inv$ and $\underline{\pi}\inv$.
    \item Key idea: Homomorphic property, they preserve additivity and multiplicativity.
    \item Allows for homomorphic \gls{simd} operations.
  \end{itemize}
\end{frame}

\begin{frame}{Encryption and Decryption}
  \cryptop{CKKS.} \\
  \begin{tblr}{Q[l,h]Q[l,h,\textwidth - 3.5cm]}
    \cryptop{Encrypt}$(\vec{p}, m)$ & {
        Let $(b,a) = \vec{p}$, $u \leftarrow \chi_{enc}$, $\mu_1, \mu_2 \leftarrow \chi_{error}$,
        then the ciphertext is $\vec{c} = u \cdot \vec{p} + (m + \mu_1, \mu_2) = (m + bu + \mu_1, au + \mu_2)$
        $\quad\rightarrow \vec{c}$} \\
    \cryptop{Decrypt}$(s, \vec{c})$ & {
        Decrypt the ciphertext $\vec{c} = (c_0, c_1)$ as $m = \lbrack c_0 + c_1 s\rbrack_{q_L}$
        $\quad\rightarrow m$} \\
  \end{tblr}
  \begin{itemize}
    \item A public-key cryptosystem! Encrypt with $\vec{p}$, decrypt with $s$.
    \item Leaves the attacker with the \gls{rlwe} problem.
    \item Decrypts correctly under certain conditions...
  \end{itemize}
\end{frame}

\begin{frame}{Homomorphic Addition}
  \begin{tblr}{Q[l,h]Q[l,h,\textwidth - 3.5cm]}
    \cryptop{CKKS.Add}$(\vec{c}_1, \vec{c}_2)$ & {
        Output $\vec{c}_3 = \vec{c}_1 + \vec{c}_2$
        $\quad\rightarrow \vec{c}_3$} \\
  \end{tblr}

  Decrypts correctly?
  \begin{align*}
    \cryptop{CKKS.Decrypt}(s, \overline{\vec{c}})
     & = \lfloor \delta\inv [\overline{c_0} + \overline{c_1} s]_t \rceil                                                                                                                                                         \\
     & = \big\lfloor \delta\inv [\delta \overline{m} + b \overline{u} + \overline{\mu_1} + (a \overline{u} + \overline{\mu_2}) s]_t \big\rceil                                                                                   \\
     & = \big\lfloor [(\delta\inv\delta) \overline{m} + \delta\inv b \overline{u} + \delta\inv \overline{\mu_1} + \delta\inv a s \overline{u} + \delta\inv \overline{\mu_2} s]_t \big\rceil                                      \\
     & = \big\lfloor [\overline{m} - \cancel{\delta\inv as \overline{u}} - \delta\inv \tilde{\mu} \overline{u} + \delta\inv \overline{\mu_1} + \cancel{\delta\inv as \overline{u}} + \delta\inv \overline{\mu_2} s]_t \big\rceil \\
     & = \big\lfloor [\overline{m} + \underbrace{\delta\inv (\overline{\mu_1} + \overline{\mu_2} s - \tilde{\mu} \overline{u})}_{:= \epsilon \,, ||\epsilon|| \ll 1}]_t \big\rceil
    \approx \big\lfloor [\overline{m}]_t \big\rceil = \lfloor \overline{m} \rceil \approx \overline{m}
  \end{align*}
\end{frame}

  \chapter{Implementation}
\label{chap:implementation}

\section{Chosen Software Architecture}
In the given setting, the most accessible frontend is commonly a JavaScript web application.

To still make the classification run as quickly and efficiently as possible, a C++ binary runs
in the backend providing an HTTP API to the frontend application.
In order to allow for more flexibility of the HTTP server, the initial approach was to
pipe requests through a dedicated web application framework with database access
that would allow, for instance, user management next to the basic classification.
However, the resulting communication and computation overhead, even when running with very
efficient protocols such as ZeroMQ, was too high.

Extending the accessibility argument to reproducibility, Docker is a very solid choice \cite{using-docker-in-science}.
To run the attached demo project, simply execute
\begin{minted}{bash}
  docker-compose build
  docker-compose up
\end{minted}
in the 'code' folder and point your browser to \url{http://locahost}.

\section{The MNIST dataset}
The MNIST dataset \cite{mnist-original} contains X train and Y test images with corresponding labels.
In order to stick to the traditional feedforward technique with data represented
in vector format, therefore it is common to reshape data from $(28, 28)$ images (represented as grayscale values in a matrix)
into a $784$ element vector.

\section{Matrix-Vector Multiplication}
The dot product that is required as part of the neural network evaluation process
needs to be implemented on SEAL ciphertexts as well.

There are multiple methods to achieve a syntactically correct dot product (matrix-vector multiplication)
as described in \cite{2018-gazelle} for square matrices.

\begin{enumerate}
  \item Naive
  \item Diagonal
  \item Hybrid
\end{enumerate}

\begin{figure}
  \centering
  \includegraphics[width=0.4\linewidth]{figures/matrix-vector-multiplication-techniques.png}
  \caption[Image source: \cite{2018-gazelle}]{Different techniques to compute a dot product between a matrix and a vector,
    each having their up- and downsides.}
\end{figure}

\subsection{Adapting to non-square matrices}
The weight matrices in the given classification setting
are by no means square, on the contrary their output dimension tends
to be much lower than the input dimension as the goal is to reduce it from
$28^2 = 784$ to $10$ overall.

However, that also means one cannot directly apply the diagonal method
as described in the proceedings above.
This 'flaw' can be mitigated by a simple zero-padding approach
in order to make the matrix square, filling in zeros until
the lower dimension reaches the higher one.


  \section{Live Demo of the WebApp}
  \begin{frame}{Demo: Secure Handwritten Digit Classification as a Service}
    \begin{columns}[c]
      \begin{column}{0.7\linewidth}
        \begin{figure}[H]
          \centering
          \includegraphics[width=0.75\linewidth]{../thesis/figures/frontend.pdf}
          \vspace{-0.3cm}
          \caption{\url{https://secure-classification.peter.waldert.at/}.}
        \end{figure}
      \end{column}
      \begin{column}{0.24\linewidth}
        Scan the QR-Code:
        \qrcode[nolink,height=3.1cm]{https://secure-classification.peter.waldert.at/}
      \end{column}
    \end{columns}
  \end{frame}

  \chapter{Results}
\label{chap:results}

\section{The Training Process}
\begin{figure}[H]
  \centering
  \pgfplotsset{/pgfplots/group/.cd,vertical sep=1.6cm}
  \inputtikz{figures/generated/training-history}
  \caption[Classification accuracy and loss development during training]{
    Development of the classification accuracy and the mean squared error during the training process of our neural network.
    Training and validation set metrics are plotted separately.
    When the validation accuracy starts to drop, the training process halts with the next epoch to prevent overfitting.
  }
  \label{fig:training-history}
\end{figure}

The machine learning framework behind the project, Tensorflow, splits its training process into \textit{epochs}, which can be found on the x-axis in the plot above.
For each training epoch, we find the progress that has been made in a single epoch by looking at the new accuracy (which percentage of the images has been classified correctly) and the loss function (\gls{mse} in this case).
Per training run, we make a differentiation between training metrics and validation metrics, illustratively shown above for the given network.
The validation data is not involved with the training process, it is only used to find a point in the process when training accuracy still rises while validation accuracy starts to drop (confer \cref{fig:training-history}).
At this point we will likely find the network's learning process in an \textit{overfitting} situation, so the training process terminates.

\section{Accuracy, Precision, Recall}
\label{sec:accuracy-precision-recall}
\begin{figure}[H]
  \centering
  \inputtikz{figures/generated/confusion-matrix}
  \caption[Confusion Matrix of the trained network]{
    The Confusion Matrix of the trained network, showing digit-wise correct classifications in the diagonal and misclassifications, per digit-pair, in the off-diagonals.
    The matrix values were visually enhanced by mapping them to their logarithm base 2.
  }
  \label{fig:confusion-matrix}
\end{figure}

As we can see in \cref{fig:confusion-matrix}, the majority of all images are classified correctly (visible in the diagonal).
What makes the confusion matrix so interesting is identifying frequently mixed up digits, for instance, 3 and 5, 2 and 7 or 4 and 9.
Judging with a human eye, this is somewhat reasonable, even more so when looking at the actual set of misclassified images (\cref{fig:misclassifications}).

\begin{figure}[H]
  \centering
  \inputtikz{figures/misclassifications}
  \caption[Misclassified images of the test set]{
    \gls{mnist} test images with true labels $0, 1, 2, 3, 4, 5, 6, 7, 8, 9$ that were misclassified as $6, 8, 9, 8, 2, 6, 0, 9, 2, 3$ by the neural network.
  }
  \label{fig:misclassifications}
\end{figure}

The plain network classifies \SI{97.62}{\percent} of the 10,000 test images correctly.
Running the encrypted classification on the full test set too, the encrypted network classifies \SI{97.31}{\percent} of the test images correctly in $3\frac{1}{2}$ hours at full CPU utilisation, using the hybrid matrix multiplication method.
For a binary classification, two further metrics of interest are
$$\text{Precision} := \frac{\text{tp}}{\text{tp} + \text{fp}} \quad\quad
  \text{Recall} := \frac{\text{tp}}{\text{tp} + \text{fn}}$$

with
$\text{tp}$ ... True Positives,
$\text{fp}$ ... False Positives,
$\text{fn}$ ... False Negatives.

Precision (also referred to as PPV, positive predictive value) refers to the ability of the network to classify positive samples correctly, while Recall explains the completeness of the classified samples (i.e. how few true positives have been left out).

\begin{table}[H]
  \centering
  \caption[Precision and recall of each digit]{Precision and Recall of the trained network for each digit individually, above for the plain network evaluation (P) and below for the encrypted evaluation (E).}
  \begin{tblr}{r|cccccccccc}
    \textbf{Digit}     & 0     & 1     & 2     & 3     & 4     & 5     & 6     & 7     & 8     & 9     \\
    \hline
    \textbf{Precision (P)} & 0.978 & 0.990 & 0.959 & 0.960 & 0.985 & 0.968 & 0.977 & 0.976 & 0.963 & 0.978 \\
    \textbf{Recall (P)}    & 0.986 & 0.989 & 0.975 & 0.977 & 0.975 & 0.964 & 0.980 & 0.964 & 0.967 & 0.955 \\
    \hline
    \textbf{Precision (E)} & 0.987 & 0.976 & 0.979 & 0.946 & 0.961 & 0.960 & 0.981 & 0.974 & 0.975 & 0.956 \\
    \textbf{Recall (E)}    & 0.989 & 0.990 & 0.958 & 0.981 & 0.975 & 0.966 & 0.972 & 0.963 & 0.937 & 0.959 \\
  \end{tblr}
\end{table}

Averaged over all digits, the mean precision (in plain) amounts to \SI{97.37}{\percent} while the average recall is similarly high at \SI{97.36}{\percent}.
The encrypted network evaluation of the full test set yields an average Mean Max-Relative Error (see \cref{tab:performance-benchmarks}) of \SI{2.97}{\percent} using the hybrid method.
And although it is numerically less accurate than the plain computation, only a handful of the 10,000 images were classified incorrectly, that the plain network managed to classify correctly.

\section{Performance Benchmarks}
\label{sec:performance-benchmarks}
This section finally gives a runtime and communication overhead analysis of the project.
For fair comparisons, the preparations required for each method individually were not taken into account, as they can be performed once at the beginning of the program.

\vspace{6pt}
\begin{table}[H]
  \centering
  \caption[Performance Benchmarks / Communication Overhead]{
    Performance benchmarks and communication overhead of the classification procedure on an Intel\textregistered \, i7-5600U CPU, including the encoding and decoding steps.
    Different parameter sets $\bm{B_1}, \bm{B_2}, \bm{N}$ are compared for each of the implemented matrix multiplication methods \textit{diagonal}, \textit{hybrid} and \textit{Babystep-Giantstep} (\glstext{bsgs}), looking at the averaged runtime, message size and also the accuracy when compared to the plain result ($\bm{\Delta}$).
    \vspace{6pt}
  }
  \captionsetup{margin=10pt}
  \caption*{
    $\bm{B_1}$ ... Coefficient Moduli start bits (also equal to the last) \\
    $\bm{B_2}$ ... Coefficient Moduli middle bits, also defines the scale as $2^{B_2}$ \\
    $\bm{N}$ ... Polynomial Modulus Degree, found in the exponent of $p(X) = X^N + 1$ \\
    $\bm{T}$ ... Runtime of encryption + classification + decryption \\
    $\bm{M}$ ... Message Size (Relin Keys + Galois Keys + Request Ciphertext + Response Ciphertext) \\
    $\bm{\Delta}$ ... Mean Max-Relative Error compared to the exact result, i.e. $\frac{\langle |\bm{y}_{prediction} - \bm{y}_{exact}| \rangle}{\max |\bm{y}_{exact}|}$
  }
  \SetTblrInner{rowsep=1pt}
  \begin{tblr}{
    colspec={ccrrrrrrr},
    row{2,3,4} = {bg=azure9},
    row{5,6,7} = {bg=violet9},
    row{8,9,10} = {bg=blue9},
    row{11,12,13} = {bg=azure9},
      }
    \hline
    \bf Mode & \bf SecLevel & $\bm{B_1}$ & $\bm{B_2}$ & $\bm{N}$ & \bf MatMul & $\bm{T}$ / \si{\second} & $\bm{M}$ / \si{\mebi\byte} & $\bm{\Delta}$ / 1 \\
    \hline
    Release & tc128 & 34 & 25 & 8192 & Diagonal & 8.39 & 132.72 & 0.0364 \\
    & & & & & Hybrid & 1.35 & 132.72 & 0.0362 \\
    & & & & & BSGS & 1.66 & 132.72 & 0.1433 \\
    \hline
    & tc128 & 60 & 40 & 16384 & Diagonal & 17.24 & 286.51 & 0.0363 \\
    & & & & & Hybrid & 3.05 & 286.51 & 0.0364 \\
    & & & & & BSGS & 3.66 & 286.51 & 0.1399 \\
    \hline
    & tc256 & 60 & 40 & 32768 & Diagonal & 35.24 & 615.16 & 0.0363 \\
    & & & & & Hybrid & 5.99 & 615.16 & 0.0364 \\
    & & & & & BSGS & 7.34 & 615.16 & 0.1399 \\
    \hline
    Debug & tc128 & 34 & 25 & 8192 & Diagonal & 7.80 & 132.72 & 0.0358 \\
    & & & & & Hybrid & 1.33 & 132.72 & 0.0370 \\
    & & & & & BSGS & 1.66 & 132.72 & 0.1434 \\
  \end{tblr}
  \label{tab:performance-benchmarks}
\end{table}

The above benchmarks (\cref{tab:performance-benchmarks}) were accumulated on an Intel\textregistered \, i7-5600U CPU running at \SI{2.6}{\giga\hertz} as the average over 3 individual runs with different test vectors, consistent accross different parameter runs.

Without any encryption, the neural network classifies the full 10,000 image dataset in \SI{515}{\milli\second} on the same machine, as compared to $3 \frac{1}{2}$ hours for the encrypted evaluation.

Obviously, smaller parameters $B_1, B_2, N$ yield smaller polynomials, in the number of coefficients as well as the coefficient representations, and therefore cause less computation and communication overhead.
The slowest method is the diagonal method, followed by \gls{bsgs} and the winner is the hybrid method in this case, though not by far in terms of speed!
Looking at accuracy, the diagonal method wins, although it is negligibly close to the numeric accuracy of the hybrid method.

The most secure tested parameters are $60, 40, 32768$ which already lead to quite long classification times while ensuring a security level of 256 bits.
For the web-based demonstrator (where speed matters), the natural choice is the first parameter set of $34, 25, 8192$ which allows for the most efficient computations and short response times in combination with the hybrid matrix multiplication method, while still guaranteeing 128-bit security.

\section{Ciphertext Visualisations}
In order to visually demonstrate the encryption, visualisations of the ciphertext polynomial $c_0$ (refer to \cref{sec:ckks}) were generated using a decomposition of the \gls{rns} representation of $c_0$.
This is required because \gls{seal} optimizes the storage of the ciphertext polynomials by splitting them up into sub-polynomials using the \gls{crt}.
The full modulus $q$, required to be huge for a secure system, is split up into multiple moduli $q_1, q_2, ...$, depending on the system parameters, so that $q = q_1 \cdot q_2 \cdot ... = \prod_{i=1} q_i$.
Each pixel in the image below then corresponds to a coefficient $a \in \Z / q\Z$ scaled down by the total modulus $q$ to obtain a brightness value between $0$ and $1$.

\vspace{8pt}
\begin{figure}[H]
  \centering
  \inputtikz{figures/ciphertext-visualisation}
  \caption[Visualisation of the plain input images compared to their ciphertext]{Ciphertext Visualisation: The first row corresponds to the images in plain, the second row depicts an encrypted version, namely the reconstructed polynomial coefficients $a_k$ of the ciphertext polynomial.}
  \label{fig:ciphertext-visualisation}
\end{figure}


  \section*{}
  \begin{frame}{Conclusion}
    \begin{itemize}
      \item Schemes like \gls{rsa} become problematic due to \name{Shor}'s Algorithm $\Rightarrow$ Lattice Crypto.
      \item New Cryptosystems constructed based on \name{Regev}'s \gls{lwe}-problem, e.g. \gls{ckks}.
      \item Encryption is homomorphic with respect to addition (and multiplication).
      \item The Encoding and Decoding procedures of CKKS allow for \gls{simd} operations needed for efficient computations.
      \item Image Classification of the handwritten digits can be done using a neural network.
      \item The required operations can be translated to \gls{he}.
      \item For better performance, improved matrix multiplication methods are utilised.
      \item Our Demonstrator: \url{https://secure-classification.peter.waldert.at/}.
    \end{itemize}
  \end{frame}

  %\begin{frame}[c,plain]
  \begin{frame}[c]
    \centering
    \Large Questions?
  \end{frame}

  \begin{frame}[allowframebreaks]{Glossary}
    \printnoidxglossary[type=acronym]
  \end{frame}

  \begin{frame}[allowframebreaks]{Bibliography}
    \printbibliography
  \end{frame}

  \appendix
\section{\appendixname}
\begin{frame}{Details...}
  Additional Material omitted in main talk.

  \begin{itemize}
    \item Encoding and Decoding transformations
    \item Proof of Diagonal, Hybrid method
    \item Shor's Algorithm
  \end{itemize}
\end{frame}

\end{document}
